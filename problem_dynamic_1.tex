\subsubsection{Постановка задачі}
Розглядається пружне прямокутне тіло (Рис: \ref{geom_gen}), яке займає облась,
що описується у декартовій системі координат співвідношеннями $0 \le x \le a$, $0 \le y \le b$.

До грані $y=b$ додане нормальне навантаження
\begin{equation}\label{bound_1_dynamic_1}
    \sigma_y(x, y, t) |_{y=b} = -p(x, t), \quad  \tau_{xy}(x,y,t) |_{y=b} =0
\end{equation}
де $p(x,t)$ відома функція.
На бічних та нижній гранях виконуються умови ідеального контакту
\begin{equation}\label{bound_2_dynamic_1}
    u(x,y,t) |_{x=0} = 0, \quad \tau_{xy}(x,y,t) |_{x=0} =0
\end{equation}
\begin{equation}\label{bound_3_dynamic_1}
    u(x,y,t) |_{x=a} = 0, \quad \tau_{xy}(x,y,t) |_{x=a} =0
\end{equation}
\begin{equation}\label{bound_4_dynamic_1}
    v(x,y,t) |_{y=0} = 0, \quad \tau_{xy}(x,y,t) |_{y=0} =0
\end{equation}
Потрібно відшукати розв'язок рівняннь Ламе
\begin{equation}
    \begin{cases}
        \frac{\partial^2 u(x,y,t)}{\partial x^2} + \frac{\partial^2 u(x,y,t)}{\partial y^2} + \mu_0 (\frac{\partial^2 u(x,y,t)}{\partial x^2} + \frac{\partial^2 v(x,y,t)}{\partial x\partial y}) = \frac{1}{c_1^2} \frac{\partial^2 u(x,y,t)}{\partial t^2} \\
        \frac{\partial^2 v(x,y,t)}{\partial x^2} + \frac{\partial^2 v(x,y,t)}{\partial y^2} + \mu_0 (\frac{\partial^2 u(x,y,t)}{\partial x \partial y} + \frac{\partial^2 v(x,y,t)}{\partial y^2}) = \frac{1}{c_2^2} \frac{\partial^2 v(x,y,t)}{\partial t^2} \\
    \end{cases}
\end{equation}
за умови виконання крайових умов \eqref{bound_1_dynamic_1} - \eqref{bound_4_dynamic_1}

Тут і далі розглянуто постановкe задачі у випадку гармонічних коливань
\begin{equation}\label{garm_dynamic_1}
    u(x,y,t) = u(x,y) e^{i \omega t}, \quad v(x,y,t) = v(x,y) e^{i \omega t}, \quad p(x, y, t) = p(x, y) e^{i \omega t}
\end{equation}
З урахуванням подання переміщеннь \eqref{garm_dynamic_1} рівняння Ламе переформульовано:
\begin{equation}\label{lame_dynamic_1}
    \begin{cases}
        \frac{\partial^2 u(x,y)}{\partial x^2} + \frac{\partial^2 u(x,y)}{\partial y^2} + \mu_0 (\frac{\partial^2 u(x,y)}{\partial x^2} + \frac{\partial^2 v(x,y)}{\partial x\partial y}) = -\frac{\omega^2}{c_1^2}  u(x,y) \\
        \frac{\partial^2 v(x,y)}{\partial x^2} + \frac{\partial^2 v(x,y)}{\partial y^2} + \mu_0 (\frac{\partial^2 u(x,y)}{\partial x \partial y} + \frac{\partial^2 v(x,y)}{\partial y^2}) = -\frac{\omega^2}{c_2^2} v(x,y) \\
    \end{cases}
\end{equation}
Граничні умови набувають вигляду:
\begin{equation}\label{bound_dynamic_1}
    \begin{cases}
        \sigma_y(x, y) |_{y=b} = -p(x, t), \quad  \tau_{xy}(x,y) |_{y=b} = 0 \\
        v(x,y) |_{y=0} = 0, \quad \tau_{xy}(x,y) |_{y=0} = 0 \\
        u(x,y) |_{x=0} = 0, \quad \tau_{xy}(x,y) |_{x=0} = 0 \\
        u(x,y) |_{x=a} = 0, \quad \tau_{xy}(x,y) |_{x=a} = 0 
    \end{cases}
\end{equation}

Треба знайти хвильове поле пружного прямокутника,
що задовольняє крайову задачу \eqref{lame_dynamic_1}, \eqref{bound_dynamic_1}.

\subsubsection{Побудова точного розв'язку вихідної задачі}
Для того, щоби звести задачу до одновимірної задачі у просторі трансформант, використано інтегральне перетворення Фур'є за змінною $x$ до рівнянь (\ref{lame_static_1}):
\begin{equation}\label{int_trans_dynamic_1}
    \begin{pmatrix}
        u_n(y) \\
        v_n(y)
    \end{pmatrix} = \int_{0}^{a} 
    \begin{pmatrix}
        u(x,y) sin(\alpha_n x) \\
        v(x,y) cos(\alpha_n x)
    \end{pmatrix} dx, \quad \alpha_n = \frac{\pi n}{a}
\end{equation}

Після інтегрування за частинами обох рівнянь Ламе оримаємо наступні рівняння у просторі трансформант
\begin{equation}\label{transf_dynamic_1}
    \begin{cases}
        u_n^{''}(y) - \alpha_n \mu_0 v_n^{'}(y) + (-\alpha_n^2 -\alpha_n^2 \mu_0 + \frac{\omega^2}{c_1^2}) u_n(y) = 0 \\
        (1 + \mu_0) v_n^{''}(y) + \alpha_n \mu_0 u_n^{'}(y) + (- \alpha_n^2 + \frac{\omega^2}{c_2^2}) v_n(y) = 0 \\
    \end{cases}
\end{equation}
Застосовуючи інтегральне перетворення \eqref{int_trans_dynamic_1} до крайових умов \eqref{bound_dynamic_1},
отримаємо крайові умови задачі у просторі трансформант
\begin{equation}\label{transf_bound_dynamic_1}
    \begin{cases}
        \left( (2G + \lambda)v_n^{'}(y) + \alpha_n \lambda u_n(y) \right)|_{y=b} = -p_n \\
        \left(u_n^{'}(y) - \alpha_n v_n(y)  \right)|_{y=b} = 0 \\
        v_n(y)|_{y=0} = 0 \\
        \left(u_n^{'}(y) - \alpha_n v_n(y)  \right)|_{y=0} = 0
    \end{cases}
\end{equation}
де $p_n = \int_{0}^{a} p(x) cos(\alpha_n x) dx$

Для того щоб розв'язати задачу у простосторі трансформант, її переписано у векторній формі.
Рівняння рівноваги (\ref{transf_dynamic_1}) запишемо у наступному вигляді:
\begin{equation}\label{transf_mat_dynamic_1}
    L_2\left[ Z_n(y) \right] = 0
\end{equation}
\begin{equation}
    L_2\left[ Z_n(y) \right] = A * Z_n^{''}(y) + B * Z_n^{'}(y) + C * Z_n(y)
\end{equation}
де
\begin{equation*}
    A = \begin{pmatrix}
        1 & 0 \\
        0 & 1 + \mu_0
    \end{pmatrix}, \quad
    B = \begin{pmatrix}
        0 & -\alpha_n \mu_0 \\
        \alpha_n \mu_0 & 0
    \end{pmatrix}
\end{equation*}
\begin{equation*}
    C = \begin{pmatrix}
        -\alpha_n^2 -\alpha_n^2 \mu_0 + \frac{\omega^2}{c_1^2} & 0 \\
        0 & -\alpha_n^2 + \frac{\omega^2}{c_2^2}
    \end{pmatrix}, \quad
    Z_n(y) = \begin{pmatrix}
        u_n(y) \\
        v_n(y)
    \end{pmatrix}
\end{equation*}
Граничні умови (\ref{transf_bound_dynamic_1}) запишемо у наступному вигляді:
\begin{equation}\label{transf_bound_mat_dynamic_1}
    U_i\left[ Z_n(y) \right] = D_i
\end{equation}
\begin{equation}
    U_i\left[ Z_n(y) \right] = E_i * Z_n^{'}(b_i) + F_i * Z_n(b_i)
\end{equation}
де $i = \overline{0, 1}$, $b_0 = b$, $b_1 = 0$,
\begin{equation*}
    E_0 = \begin{pmatrix}
        1 & 0 \\
        0 & 2G + \lambda
    \end{pmatrix}, \quad
    F_0 = \begin{pmatrix}
        0 & -\alpha_n \\
        \alpha_n \lambda & 0
    \end{pmatrix}, \quad
\end{equation*}
\begin{equation*}
    E_1 = \begin{pmatrix}
        1 & 0 \\
        0 & 0
    \end{pmatrix}, \quad
    F_1 = \begin{pmatrix}
        0 & -\alpha_n \\
        0 & 1
    \end{pmatrix}, \quad
\end{equation*}
\begin{equation*}
    D_0 = \begin{pmatrix}
        0 \\
        -p_n
    \end{pmatrix}, \quad
    D_1 = \begin{pmatrix}
        0 \\
        0
    \end{pmatrix}, \quad
\end{equation*}

Будемо шукати однорідний розв'язок рівняння,
для цього спочатку побудуємо фундаментальну матрицю рівняння \eqref{transf_mat_static_1}.
Її будемо шукати у наступному поданні \cite{gantmaher}:
\begin{equation}
    Y(y) = \frac{1}{2\pi i} \oint_C e^{sy} M^{-1}(s)ds
\end{equation}
Де $M(s)$ - характерестична матриця рівняння (\ref{transf_mat_dynamic_1}), а $C$ - замкнений контур який містить усі особливі точки.
Матрицю $M(s)$ будемо шукати з наступної умови
\begin{equation}
    L_2\left[ e^{sy}*I \right] = e^{sy} * M(s), \quad I = \begin{pmatrix} 1 & 0 \\ 0 & 1 \end{pmatrix}
\end{equation}
де матриця $M(s)$ має вигляд:
\begin{equation}
    M(s) = \begin{pmatrix}
        s^2 - \alpha_n^2 - \alpha_n^2\mu_0 + \frac{\omega^2}{c_1^2} & -\alpha_n \mu_0 s \\
        \alpha_n \mu_0 s & s^2 (1 + \mu_0) -\alpha_n^2 + \frac{\omega^2}{c_2^2}
     \end{pmatrix}
\end{equation}

Знайдемо тепер $M^{-1}(s)$, яку побудовано у наступній формі $M^{-1}(s) = \frac{\widetilde{M(s)}}{det[M(s)]}$, де $\widetilde{M(s)}$ - транспонована матриця алгебричних доповнень,
$det[M(s)]$ - детермінант матриці
\begin{equation}
    \widetilde{M(s)} = \begin{pmatrix}
        s^2 (1 + \mu_0) -\alpha_n^2 + \frac{\omega^2}{c_2^2} & \alpha_n \mu_0 s \\
        -\alpha_n \mu_0 s & s^2 - \alpha_n^2 - \alpha_n^2\mu_0 + \frac{\omega^2}{c_1^2}
     \end{pmatrix}
\end{equation}
\begin{align}
    &det[M(s)] = \begin{vmatrix}
        s^2 - \alpha_n^2 - \alpha_n^2\mu_0 + \frac{\omega^2}{c_1^2} & -\alpha_n \mu_0 s \\
        \alpha_n \mu_0 s & s^2 (1 + \mu_0) -\alpha_n^2 + \frac{\omega^2}{c_2^2}
     \end{vmatrix} = \nonumber \\
    &=(s - s_1)(s + s_1)(s - s_2)(s + s_2),
\end{align}
де $s_1$, $s_2$, $-s_1$, $-s_2$ корені $det[M(s)]=0$, детальне знаходження яких наведено у Додатку B.

Враховучи це, тепер знайдемо значення фундаментальної матриці за допомогою теореми про лишки:
\begin{align*}
    &\frac{1}{2\pi i} \oint_C e^{sy} M^{-1}(s)ds = \frac{2 \pi i}{2 \pi i (1 + \mu_0)} \sum_{i=1}^{4} Res\left[ e^{sy} \frac{\widetilde{M(s)}}{det[M(s)]} \right] = \\
    & = \left(Y_0(y) + Y_1(y) + Y_2(y) + Y_3(y) \right)
\end{align*}
де
\begin{align}
    &Y_0(y) =  \left( \frac{e^{sy}}{(s+s_1)(s - s_2)(s + s_2)} \widetilde{M(s)} \right) \Big|_{s=s_1} = \nonumber \\
    &=\frac{e^{s_1 y}}{2s_1 (s_1^2 - s_2^2)} \begin{pmatrix}
        s_1^2 (1 + \mu_0) -\alpha_n^2 + \frac{\omega^2}{c_2^2} & \alpha_n \mu_0 s_1 \\
        -\alpha_n \mu_0 s_1 & s_1^2 - \alpha_n^2 - \alpha_n^2\mu_0 + \frac{\omega^2}{c_1^2}
    \end{pmatrix}
\end{align}
\begin{align}
    &Y_1(y) =  \left( \frac{e^{sy}}{(s-s_1)(s - s_2)(s + s_2)} \widetilde{M(s)} \right) \Big|_{s=-s_1} = \nonumber \\
    &=-\frac{e^{-s_1 y}}{2s_1 (s_1^2 - s_2^2)} \begin{pmatrix}
        s_1^2 (1 + \mu_0) -\alpha_n^2 + \frac{\omega^2}{c_2^2} & -\alpha_n \mu_0 s_1 \\
        \alpha_n \mu_0 s_1 & s_1^2 - \alpha_n^2 - \alpha_n^2\mu_0 + \frac{\omega^2}{c_1^2}
    \end{pmatrix}
\end{align}
\begin{align}
    &Y_2(y) =  \left( \frac{e^{sy}}{(s+s_2)(s - s_1)(s + s_1)} \widetilde{M(s)} \right) \Big|_{s=s_2} = \nonumber \\
    &=\frac{e^{s_2 y}}{2s_2 (s_2^2 - s_1^2)} \begin{pmatrix}
        s_2^2 (1 + \mu_0) -\alpha_n^2 + \frac{\omega^2}{c_2^2} & \alpha_n \mu_0 s_2 \\
        -\alpha_n \mu_0 s_2 & s_2^2 - \alpha_n^2 - \alpha_n^2\mu_0 + \frac{\omega^2}{c_1^2}
    \end{pmatrix}
\end{align}
\begin{align}
    &Y_3(y) =  \left( \frac{e^{sy}}{(s-s_2)(s - s_1)(s + s_1)} \widetilde{M(s)} \right) \Big|_{s=-s_2} = \nonumber \\
    &=-\frac{e^{-s_2 y}}{2s_2 (s_2^2 - s_1^2)} \begin{pmatrix}
        s_2^2 (1 + \mu_0) -\alpha_n^2 + \frac{\omega^2}{c_2^2} & -\alpha_n \mu_0 s_2 \\
        \alpha_n \mu_0 s_2 & s_2^2 - \alpha_n^2 - \alpha_n^2\mu_0 + \frac{\omega^2}{c_1^2}
    \end{pmatrix}
\end{align}

Отримаємо розв'зок однорідного рівняння у просторі трансформант, який подано у формі
\begin{equation}
    Z_n(y) =\left( Y_0(y) +  Y_1(y)  \right) * \begin{pmatrix} c_1 \\ c_2 \end{pmatrix} + \left( Y_2(y) +  Y_3(y) \right) * \begin{pmatrix} c_3 \\ c_4 \end{pmatrix}
\end{equation}
Для того, щоб відшукати невідомі коєфіцієнти $c_i$, $i=\overline{1, 4}$ потрібно задовольнити граничні умови \eqref{transf_bound_mat_dynamic_1} (\nameref{ap_E}).
Тепер можна записати фінальний розв'язок векторної крайової задачі у просторі трансформант:
\begin{align}\label{transf_sol_u_dynamic_1}
    &u_n(y) = \frac{( s_1^2 (1 + \mu_0) -\alpha_n^2 + \frac{\omega^2}{c_2^2})(e^{s_1y} - e^{-s_1y})}{2s_1(s_1^2 - s_2^2)(1 + \mu_0)}c_1 + \nonumber \\
    & + \frac{( s_2^2 (1 + \mu_0) -\alpha_n^2 + \frac{\omega^2}{c_2^2})(e^{s_2y} - e^{-s_2y})}{2s_2(s_2^2 - s_1^2)(1 + \mu_0)}c_3 + \nonumber \\
    & + \frac{( s_1 \alpha_n y)(e^{s_1y} + e^{-s_1y})}{2s_1(s_1^2 - s_2^2)(1 + \mu_0)}c_2 + \frac{(s_2 \alpha_n y)(e^{s_2y} + e^{-s_2y})}{2s_2(s_2^2 - s_1^2)(1 + \mu_0)}c_4
\end{align}
\begin{align}\label{transf_sol_v_dynamic_1}
    &v_n(y) = \frac{(s_1^2 - \alpha_n^2 - \alpha_n^2\mu_0 + \frac{\omega^2}{c_1^2})(e^{s_1y} - e^{-s_1y})}{2s_1(s_1^2 - s_2^2)(1 + \mu_0)}c_2 + \nonumber \\
    & +\frac{(s_2^2 - \alpha_n^2 - \alpha_n^2\mu_0 + \frac{\omega^2}{c_1^2})(e^{s_2y} - e^{-s_2y})}{2s_2(s_2^2 - s_1^2)(1 + \mu_0)}c_4 - \nonumber \\
    & - \frac{(s_1 \alpha_n y)(e^{s_1y} + e^{-s_1y})}{2s_1(s_1^2 - s_2^2)(1 + \mu_0)}c_1 - \frac{(s_2 \alpha_n y)(e^{s_2y} + e^{-s_2y})}{2s_2(s_2^2 - s_1^2)(1 + \mu_0)}c_3
\end{align}

Використовуючи оберненне інтегральне перетворення Фур'є для трансформант переміщень \eqref{transf_sol_u_dynamic_1}, \eqref{transf_sol_v_dynamic_1} завершує побудову розв'язку вихідної задачі
\begin{equation}
    u(x,y) = \frac{2}{a} \sum_{n=1}^{\infty} u_n(y) sin(\alpha_n x), \quad \alpha_n = \frac{\pi n}{a}
\end{equation}
\begin{equation}
    v(x,y) = \frac{v_0(y)}{a} + \frac{2}{a} \sum_{n=1}^{\infty} v_n(y) cos(\alpha_n x), \quad \alpha_n = \frac{\pi n}{a}
\end{equation}

Останній крок - це знаходження $v_0(y)$ у випадку коли $n=0$, $\alpha_0 =0$.
Для цього повернемся до другого рівняння (\ref{transf_dynamic_1}), та запишем його для цього випадку:
\begin{equation}\label{transf_v_0_dynamic_1}
    (1 + \mu_0) v_0^{''}(y) + \frac{\omega^2}{c_2^2}v_0(y) = 0
\end{equation}
Та запишимо відповідні граничні умови:
\begin{equation}\label{transf_bound_v_0_dynamic_1}
    \begin{cases}
        (2G + \lambda)v_0^{'}(y)|_{y=b} = -p_0 \\
        v_0(y)|_{y=0} = 0
    \end{cases}
\end{equation}
де $p_0 = \int_{0}^{a}p(x)dx$

Розв'язок рівняння (\ref{transf_v_0_dynamic_1}) буде мати вигляд:
\begin{equation}
    v_0(y) = c_1 cos \left(y \sqrt{\frac{\omega^2}{c_2^2(1 + \mu_0)}} \right) + c_2 sin \left( y \sqrt{\frac{\omega^2}{c_2^2(1 + \mu_0)}} \right)
\end{equation}
Отже
\begin{equation}
    v_0(y) = \frac{-p_0}{(2G + \lambda) \sqrt{\frac{\omega^2}{c_2^2(1 + \mu_0)}} sin \left(b \sqrt{\frac{\omega^2}{c_2^2(1 + \mu_0)}} \right) } sin \left(y \sqrt{\frac{\omega^2}{c_2^2(1 + \mu_0)}} \right)
\end{equation}

\subsubsection{Числові розрахунки}
Наведені чисельні експеренти розглядаються для сталі ($E=200$ ГПА, $\mu=0.25$).

Розглянута прямокунта область $0 \le x \le 10$, $0 \le y \le 15$, при функції навантаження $p(x)=(x-2.5)^2$ та частоті коливань $\omega=0.75$.
На малюнках (Рис: \ref{static_1_u_1}), (Рис: \ref{static_1_v_1}), (Рис: \ref{static_1_sigma_x_1}), (Рис: \ref{static_1_sigma_y_1})
представлені функіі переміщень $u(x,y)$, $v(x,y)$ та напружень $\sigma_x(x,y)$, $\sigma_y(x,y)$ відповідно.
