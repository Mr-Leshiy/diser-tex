Помножим перше та друге рівняння \eqref{lame_gen} на $sin(\alpha_n x)$ та $cos(\alpha_n x)$ відповідно та проінтегруєм по змінній $x$ на інтервалі $0 \le x \le a$.
Скористаємося введенною заміною $\chi_1(y) = u(0, y)$, $\chi_2(y) = v(0, y)$, $\chi_3(y) = u(a, y)$, $\chi_4(y) = v(a, y)$ та 
$\frac{\partial u(0, y)}{\partial x}=-\frac{\alpha_1}{\beta_1} \chi_1(y)$,
$\frac{\partial v(0, y)}{\partial x}=-\frac{\alpha_1}{\beta_1} \chi_2(y)$,
$\frac{\partial u(a, y)}{\partial x}=-\frac{\alpha_2}{\beta_2} \chi_3(y)$,
$\frac{\partial v(a, y)}{\partial x}=-\frac{\alpha_2}{\beta_2} \chi_4(y)$,
та враховуючи граничні умови \eqref{bound_gen} знайдем вигляд задачі у просторі трансформант.
\newline

Розглянемо перше рівнняня
\begin{align*}
    &\int_{0}^{a} \frac{\partial^2 u(x,y)}{\partial x^2} sin(\alpha_n x)dx + \int_{0}^{a} \frac{\partial^2 u(x,y)}{\partial y^2} sin(\alpha_n x)dx + \\ 
    & + \mu_0 \left( \int_{0}^{a} \frac{\partial^2 u(x,y)}{\partial x^2} sin(\alpha_n x)dx +  \int_{0}^{a} \frac{\partial^2 v(x,y)}{\partial x \partial y} sin(\alpha_n x) dx\right) + \\
    & + \frac{\omega^2}{c_1^2} \int_{0}^{a} u(x,y) sin(\alpha_n x)dx = 0
\end{align*}

Розглянемо
\begin{align*}
    &\int_{0}^{a} \frac{\partial^2 u(x,y)}{\partial x^2} sin(\alpha_n x)dx = \frac{\partial u(x,y)}{\partial x} sin(\alpha_n x) |_{x=0}^{x=a} - \alpha_n \int_{0}^{a} \frac{\partial u(x,y)}{\partial x} cos(\alpha_n x)dx = \\
    &= \frac{\partial u(x,y)}{\partial x} sin(\alpha_n x) |_{x=0}^{x=a} - \alpha_n \left( u(x,y) cos(\alpha_n x) |_{x=0}^{x=a} + \alpha_n \int_{0}^{a} u(x,y) sin(\alpha_n x) dx \right) = \\
    &=-\alpha_n(\chi_3(y) cos(\alpha_n a) - \chi_1(y)) -\alpha_n^2 u_n(y)
\end{align*}

Розглянемо
\begin{align*}
    &\int_{0}^{a} \frac{\partial^2 u(x,y)}{\partial y^2} sin(\alpha_n x)dx = \frac{\partial^2}{\partial y^2} \int_{0}^{a} u(x,y) sin(\alpha_n x)dx = u_n^{''}(y)
\end{align*}

Розглянемо
\begin{align*}
    &\int_{0}^{a} \frac{\partial^2 v(x,y)}{\partial x \partial y} sin(\alpha_n x) dx = \frac{\partial v(x,y)}{\partial y} sin(\alpha_n x) |_{x=0}^{x=a} - \alpha_n \int_{0}^{a} \frac{\partial v(x,y)}{\partial y} cos(\alpha_n x) dx = \\
    &= -\alpha_n \frac{\partial}{\partial y} \int_{0}^{a} v(x,y) cos(\alpha_n x) dx = -\alpha_n v_n^{'}(y)
\end{align*}

Тоді перше рівняння у просторі трансформант прийме вигляд:
\begin{align*}
    &u_n^{''}(y) - \alpha_n \mu_0 v_n^{'}(y) -(\alpha_n^2 + \alpha_n^2 \mu_0 - \frac{\omega^2}{c_1^2}) u_n(y) = \\
    &= \alpha_n(1 + \mu_0)(\chi_3(y) cos(\alpha_n a) - \chi_1(y))
\end{align*}

Розлянемо друге рівняння
\begin{align*}
    &\int_{0}^{a} \frac{\partial^2 v(x,y)}{\partial x^2} cos(\alpha_n x)dx + \int_{0}^{a} \frac{\partial^2 v(x,y)}{\partial y^2} cos(\alpha_n x)dx + \\ 
    & + \mu_0 \left( \int_{0}^{a} \frac{\partial^2 u(x,y)}{\partial x \partial y} cos(\alpha_n x)dx +  \int_{0}^{a} \frac{\partial^2 v(x,y)}{\partial y^2} cos(\alpha_n x) dx\right) + \\
    & + \frac{\omega^2}{c_2^2} \int_{0}^{a} v(x,y) cos(\alpha_n x)dx = 0
\end{align*}

Розглянемо
\begin{align*}
    &\int_{0}^{a} \frac{\partial^2 v(x,y)}{\partial x^2} cos(\alpha_n x)dx = \frac{\partial v(x,y)}{\partial x} cos(\alpha_n x) |_{x=0}^{x=a} + \alpha_n \int_{0}^{a} \frac{\partial v(x,y)}{\partial x} sin(\alpha_n x) dx = \\
    &=\frac{\partial v(x,y)}{\partial x} cos(\alpha_n x) |_{x=0}^{x=a} + \alpha_n \left(v(x,y) sin(\alpha_n x)|_{x=0}^{x=a} - \alpha_n \int_{0}^{a} v(x,y) cos(\alpha_n x) dx  \right) = \\
    &= -(\frac{\alpha_2}{\beta_2}\chi_4(y) cos(\alpha_n a) - \frac{\alpha_1}{\beta_1}\chi_2(y)) -\alpha_n^2 v_n(y)
\end{align*}

Розглянемо
\begin{align*}
    &\int_{0}^{a} \frac{\partial^2 v(x,y)}{\partial y^2} cos(\alpha_n x)dx = \frac{\partial^2}{\partial y^2} \int_{0}^{a} v(x,y) cos(\alpha_n x)dx = v_n^{''}(y)
\end{align*}

Розглянемо
\begin{align*}
    &\int_{0}^{a} \frac{\partial^2 u(x,y)}{\partial y \partial x} cos(\alpha_n x)dx = \frac{\partial u(x,y)}{\partial y} cos(\alpha_n x) |_{x=0}^{x=a} + \alpha_n \int_{0}^{a} \frac{\partial u(x,y)}{\partial y} sin(\alpha_n x) dx = \\
    &=\frac{\partial u(x,y)}{\partial y} cos(\alpha_n x) |_{x=0}^{x=a} + \alpha_n \frac{\partial}{\partial y} \int_{0}^{a} u(x,y) sin(\alpha_n x) dx = \alpha_n u_n^{'}(y) + \\
    &+(\chi_3^{'}(y) cos(\alpha_n a) -\chi_1^{'}(y))
\end{align*}

Тоді друге рівняння у просторі трансформант прийме вигляд:
\begin{align*}
    &(1 + \mu_0) v_n^{''}(y) + \alpha_n \mu_0 u_n^{'}(y)  - (\alpha_n^2 -  \frac{\omega^2}{c_2^2}) v_n(y) = \\ 
    &= (\frac{\alpha_2}{\beta_2}\chi_4(y) cos(\alpha_n a) - \frac{\alpha_1}{\beta_1}\chi_2(y)) - \mu_0 (\chi_3^{'}(y) cos(\alpha_n a) -\chi_1^{'}(y))
\end{align*}

У випадку статичної задачі \eqref{lame_static_1} та умов ідеального контаку на бічних гранях \eqref{bound_1_static_1}, \eqref{bound_2_static_1}
отримаємо наступні рівняння у просторі трансформант:
\begin{equation*}
    \begin{cases}
        u_n^{''}(y) - \alpha_n \mu_0 v_n^{'}(y) -(\alpha_n^2 + \alpha_n^2 \mu_0) u_n(y) = 0 \\
        (1 + \mu_0) v_n^{''}(y) + \alpha_n \mu_0 u_n^{'}(y)  - \alpha_n^2 v_n(y) = 0
    \end{cases}
\end{equation*}

У випадку динамічної задачі \eqref{lame_dynamic_1} та умов ідеального контаку на бічних гранях \eqref{bound_dynamic_1}
отримаємо наступні рівняння у просторі трансформант:
\begin{equation*}
    \begin{cases}
        u_n^{''}(y) - \alpha_n \mu_0 v_n^{'}(y) -(\alpha_n^2 + \alpha_n^2 \mu_0 - \frac{\omega^2}{c_1^2}) u_n(y) = 0 \\
        (1 + \mu_0) v_n^{''}(y) + \alpha_n \mu_0 u_n^{'}(y)  - (\alpha_n^2 -  \frac{\omega^2}{c_2^2}) v_n(y) = 0
    \end{cases}
\end{equation*}

У випадку статичної задачі \eqref{lame_static_2} та другої основної задачі теорії пружності на бічних гранях \eqref{bound_1_static_2}, \eqref{bound_2_static_2}
отримаємо наступні рівняння у просторі трансформант:
\begin{equation*}
    \begin{cases}
        u_n^{''}(y) - \alpha_n \mu_0 v_n^{'}(y) -(\alpha_n^2 + \alpha_n^2 \mu_0) u_n(y) = 0 \\
        (1 + \mu_0) v_n^{''}(y) + \alpha_n \mu_0 u_n^{'}(y)  - \alpha_n^2 v_n(y) = -cos(\alpha_n) \frac{\partial v(x,y)}{\partial x}|_{x=a}
    \end{cases}
\end{equation*}

