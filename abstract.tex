\textit{Pozhylenkov O. V.}
Plane mixed problems of elasticity for a rectangular domain - Manuscript.

A dissertation submitted for the degree of Doctor of Philosophy in the field of Applied Mathematics (11 - Mathematics and Statistics) - Odessa I.I. Mechnikov National University, Odessa, 2023.

The flat mixed problem of elasticity theory for a finite rectangular domain subjected to various types of loads was considered.
By applying the integral semibound sin- and cos-transformations of Fourier, the initial problem was reduced to a one-dimensional vector boundary problem.
The problem in the transform space was reformulated as a vector boundary problem.
The solution to this problem was constructed as a superposition of the general solution to the homogeneous vector equation and the particular solution to the nonhomogeneous equation.
The solution to the homogeneous vector equation was obtained using matrix differential calculus and represented through the fundamental matrix solution of the corresponding homogeneous matrix equation.
To obtain the particular solution to the nonhomogeneous vector equation, the Green's matrix-function was found using the method of matrix integral transformations. The Green's matrix-function was constructed in the form of a bilinear expansion, which simplifies further computations.
After applying the inverse Fourier transform to the explicit solution of the one-dimensional boundary problem in the transform space and summing weakly convergent integrals in the displacement formulas, only one unknown displacement function along the short end of the semibounded strip remains.
To find this function, a singular integral equation was obtained.
The singular integral equation was solved using the method of orthogonal polynomials.
An investigation of the stress state of the half-strip was conducted for different types of loads and sizes of the rectangular domain.


\textit{Key words:}
rectangular domain, dynamic problem, Fourier transformation, matrix Green's function, singular integral equation, method of orthogonal polynomials.