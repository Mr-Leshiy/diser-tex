\documentclass[a4paper,14pt]{extarticle}

\usepackage{amsmath}
\usepackage[english,ukrainian]{babel}
\usepackage{mathptmx}

\title{Diseration}
\author{}
\date{}

\numberwithin{equation}{section}

\begin{document}
\maketitle

\section*{\centering Перелік умовних позначень}
$G$ - коєфіцієент Ламе \newline
$E$ - молуль Юнга \newline
$\mu$ - коєфіцієнт Пуасона \newline
$\mu_0 = \frac{1}{1 - 2\mu}$ \newline
$U_x(x,y) = u(x,y)$ - переміщення по осі $x$ \newline
$U_y(x,y) = v(x,y)$ - переміщення по осі $y$

\section{Напруженний стан прямокутної області}
\subsection{Постановка задачі}
Розглядається пружна прямокутна область, яка займає облась,
що описується у декартовій системі координат співвідношенням $0 \le x \le a$, $0 \le y \le b$.

До прямокутної області на грані $y=b$ додане навантаження
\begin{equation}
    V(x, y) |_{y=b} = -p(x), \quad  \tau_{xy}(x,y) |_{y=b} =0
\end{equation}
де $p(x)$ відома функція.
На бічних гранях виконується умова ідеального контакту
\begin{equation}
    u(x,y) |_{x=0}, \quad \tau_{xy}(x,y) |_{x=0} =0
\end{equation}
\begin{equation}
    u(x,y) |_{x=a}, \quad \tau_{xy}(x,y) |_{x=a} =0
\end{equation}
На нижній грані виконуються наступні умови
\begin{equation}
    v(x,y) |_{y=0}, \quad \tau_{xy}(x,y) |_{y=0} =0
\end{equation}
Розглядаються наступні рівняння рівноваги Ламе:
\begin{equation}\label{lame_1}
    \begin{cases}
        \frac{\partial^2 u(x,y)}{\partial x^2} + \frac{\partial^2 u(x,y)}{\partial y^2} + \mu_0 (\frac{\partial^2 u(x,y)}{\partial x^2} + \frac{\partial^2 v(x,y)}{\partial x\partial y}) = 0 \\
        \frac{\partial^2 v(x,y)}{\partial x^2} + \frac{\partial^2 v(x,y)}{\partial y^2} + \mu_0 (\frac{\partial^2 u(x,y)}{\partial x \partial y} + \frac{\partial^2 v(x,y)}{\partial y^2}) = 0 \\
    \end{cases}
\end{equation}

\subsection{Зведеня задачі до одновимірної задачі у просторі трансформант}
Для того, щоб звести задачу до одновимірної задачі, використаєм інтегральне перетворення Фур'є по змінній $x$ у до рівнянь (\ref{lame_1}) наступному вигляді:
\begin{equation}
    \begin{pmatrix}
        u_n(y) \\
        v_n(y)
    \end{pmatrix} = \int_{0}^{a} 
    \begin{pmatrix}
        u(x,y) sin(\alpha_n x) \\
        v(x,y) cos(\alpha_n x)
    \end{pmatrix} dx, \quad \alpha_n = \frac{\pi n}{a}, n=\overline{0, \infty}
\end{equation}
Для цього помножим перше та друге рівняння (\ref{lame_1}) на $sin(\alpha_n x)$ та $cos(\alpha_n x)$ відповідно та проінтегруєм по змінній $x$.

\section{Додаток А. }

\end{document}