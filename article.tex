\documentclass[a4paper,14pt]{extarticle}

\usepackage{amsmath}
\usepackage[english,ukrainian]{babel}
\usepackage{mathptmx}
\usepackage{nameref}

\title{Diseration}
\author{}
\date{}

\numberwithin{equation}{section}

\begin{document}
\maketitle

\newpage

\renewcommand{\contentsname}{\centering Зміст}
\tableofcontents

\newpage

\section*{\centering Перелік умовних позначень}
$G$ - коєфіцієент Ламе \newline
$E$ - молуль Юнга \newline
$\mu$ - коєфіцієнт Пуасона \newline
$\mu_0 = \frac{1}{1 - 2\mu}$ \newline
$U_x(x,y) = u(x,y)$ - переміщення по осі $x$ \newline
$U_y(x,y) = v(x,y)$ - переміщення по осі $y$

\section{Напруженний стан прямокутної області}
\subsection{Постановка задачі}
Розглядається пружна прямокутна область, яка займає облась,
що описується у декартовій системі координат співвідношенням $0 \le x \le a$, $0 \le y \le b$.

До прямокутної області на грані $y=b$ додане нормальне навантаження
\begin{equation}
    \sigma_y(x, y) |_{y=b} = -p(x), \quad  \tau_{xy}(x,y) |_{y=b} =0
\end{equation}
де $p(x)$ відома функція.
На бічних гранях виконується умова ідеального контакту
\begin{equation}
    u(x,y) |_{x=0}, \quad \tau_{xy}(x,y) |_{x=0} =0
\end{equation}
\begin{equation}
    u(x,y) |_{x=a}, \quad \tau_{xy}(x,y) |_{x=a} =0
\end{equation}
На нижній грані виконуються наступні умови
\begin{equation}
    v(x,y) |_{y=0}, \quad \tau_{xy}(x,y) |_{y=0} =0
\end{equation}
Розглядаються наступні рівняння рівноваги Ламе:
\begin{equation}\label{lame_1}
    \begin{cases}
        \frac{\partial^2 u(x,y)}{\partial x^2} + \frac{\partial^2 u(x,y)}{\partial y^2} + \mu_0 (\frac{\partial^2 u(x,y)}{\partial x^2} + \frac{\partial^2 v(x,y)}{\partial x\partial y}) = 0 \\
        \frac{\partial^2 v(x,y)}{\partial x^2} + \frac{\partial^2 v(x,y)}{\partial y^2} + \mu_0 (\frac{\partial^2 u(x,y)}{\partial x \partial y} + \frac{\partial^2 v(x,y)}{\partial y^2}) = 0 \\
    \end{cases}
\end{equation}

\subsection{Зведеня задачі до одновимірної у просторі трансформант}
Для того, щоб звести задачу до одновимірної задачі, використаєм інтегральне перетворення Фур'є по змінній $x$ у до рівнянь (\ref{lame_1}) наступному вигляді:
\begin{equation}
    \begin{pmatrix}
        u_n(y) \\
        v_n(y)
    \end{pmatrix} = \int_{0}^{a} 
    \begin{pmatrix}
        u(x,y) sin(\alpha_n x) \\
        v(x,y) cos(\alpha_n x)
    \end{pmatrix} dx, \quad \alpha_n = \frac{\pi n}{a}, n=\overline{1, \infty}
\end{equation}

Для цього помножим перше та друге рівняння (\ref{lame_1}) на $sin(\alpha_n x)$ та $cos(\alpha_n x)$ відповідно та проінтегруєм по змінній $x$ на інтервалі $0 \le x \le a$.
Покрокове інтегрування рівняння (\ref{lame_1}) наведено у (\nameref{ap_A_1}).
Отримана система рівнянь задачі у просторі трансформант:
\begin{equation}\label{transf_1}
    \begin{cases}
        u_n^{''}(y) - \alpha_n \mu_0 v_n^{'}(y) - \alpha_n^2 (1 + \mu_0) u_n(y) = 0 \\
        (1 + \mu_0) v_n^{''}(y) + \alpha_n \mu_0 u_n^{'}(y)  - \alpha_n^2 v_n(y) = 0 \\
    \end{cases}
\end{equation}

Застосовуючи інтегральне перетворення до граничних умов,
отримаємо наступні умови задачі у просторі трансформант
\begin{equation}\label{transf_bound_1}
    \begin{cases}
        \left( (2G + \lambda)v_n^{'}(y) + \alpha_n \lambda u_n(y) \right)|_{y=b} = -p_n \\
        \left(u_n^{'}(y) - \alpha_n v_n(y)  \right)|_{y=b} = 0 \\
        v_n(y)|_{y=0} = 0 \\
        \left(u_n^{'}(y) - \alpha_n v_n(y)  \right)|_{y=0} = 0
    \end{cases}
\end{equation}
Де $p_n = \int_{0}^{a} p(x) cos(\alpha_n x) dx$

\subsection{Зведення задачі у просторі трансформант до матрично-векторної форми}
Для того щоб розв'язати задачу у простосторі трансформант, перепишмо її у матрично-векторній формі.
Рівняння рівноваги (\ref{transf_1}) запишемо у наступному вигляді:
\begin{align}
    &L_2\left[ Z_n(y) \right] = A * Z_n^{''}(y) + B * Z_n^{'}(y) + C * Z_n(y) \nonumber \\
    & L_2\left[ Z_n(y) \right] = 0
\end{align}
Де
\begin{equation*}
    A = \begin{pmatrix}
        1 & 0 \\
        0 & 1 + \mu_0
    \end{pmatrix}, \quad
    B = \begin{pmatrix}
        0 & -\alpha_n \mu_0 \\
        \alpha_n \mu_0 & 0
    \end{pmatrix}, \quad
    C = \begin{pmatrix}
        -\alpha_n^2(1 + \mu_0) & 0 \\
        0 & -\alpha_n^2
    \end{pmatrix}
\end{equation*}
\begin{equation*}
    Z_n(y) = \begin{pmatrix}
        u_n(y) \\
        v_n(y)
    \end{pmatrix}
\end{equation*}

Граничні умови (\ref{transf_bound_1}) запишемо у наступному вигляді:
\begin{align}
    &U_i\left[ Z_n(y) \right] = E_i * Z_n^{'}(b_i) + F_i * Z_n(b_i) \nonumber \\
    & U_i\left[ Z_n(y) \right] = D_i
\end{align}
Де $i = \overline{0, 1}$, $b_0 = b$, $b_1 = 0$,
\begin{equation*}
    E_0 = \begin{pmatrix}
        1 & 0 \\
        0 & 2G + \lambda
    \end{pmatrix}, \quad
    F_0 = \begin{pmatrix}
        0 & -\alpha_n \\
        \alpha_n \lambda & 0
    \end{pmatrix}, \quad
\end{equation*}
\begin{equation*}
    E_1 = \begin{pmatrix}
        1 & 0 \\
        0 & 0
    \end{pmatrix}, \quad
    F_1 = \begin{pmatrix}
        0 & -\alpha_n \\
        0 & 1
    \end{pmatrix}, \quad
\end{equation*}
\begin{equation*}
    D_0 = \begin{pmatrix}
        0 \\
        -p_n
    \end{pmatrix}, \quad
    D_1 = \begin{pmatrix}
        0 \\
        0
    \end{pmatrix}, \quad
\end{equation*}


\section{Додаток А}\label{ap_A_1}
Помножим перше та друге рівняння (\ref{lame_1}) на $sin(\alpha_n x)$ та $cos(\alpha_n x)$ відповідно та проінтегруєм по змінній $x$ на інтервалі $0 \le x \le a$.

Розглянемо перше рівнняня
\begin{align*}
    &\int_{0}^{a} \frac{\partial^2 u(x,y)}{\partial x^2} sin(\alpha_n x)dx + \int_{0}^{a} \frac{\partial^2 u(x,y)}{\partial y^2} sin(\alpha_n x)dx + \\ 
    & + \mu_0 \left( \int_{0}^{a} \frac{\partial^2 u(x,y)}{\partial x^2} sin(\alpha_n x)dx +  \int_{0}^{a} \frac{\partial^2 v(x,y)}{\partial x \partial y} sin(\alpha_n x) dx\right)
\end{align*}

Розглянемо
\begin{align*}
    &\int_{0}^{a} \frac{\partial^2 u(x,y)}{\partial x^2} sin(\alpha_n x)dx = \frac{\partial u(x,y)}{\partial x} sin(\alpha_n x) |_{x=0}^{x=a} - \alpha_n \int_{0}^{a} \frac{\partial u(x,y)}{\partial x} cos(\alpha_n x)dx = \\
    &= \frac{\partial u(x,y)}{\partial x} sin(\alpha_n x) |_{x=0}^{x=a} - \alpha_n \left( u(x,y) cos(\alpha_n x) |_{x=0}^{x=a} + \alpha_n \int_{0}^{a} u(x,y) sin(\alpha_n x) dx \right) = \\
    &= -\alpha_n^2 u_n(y)
\end{align*}

Розглянемо
\begin{align*}
    &\int_{0}^{a} \frac{\partial^2 u(x,y)}{\partial y^2} sin(\alpha_n x)dx = \frac{\partial^2}{\partial y^2} \int_{0}^{a} u(x,y) sin(\alpha_n x)dx = u_n^{''}(y)
\end{align*}

Розглянемо
\begin{align*}
    &\int_{0}^{a} \frac{\partial^2 v(x,y)}{\partial x \partial y} sin(\alpha_n x) dx = \frac{\partial v(x,y)}{\partial y} sin(\alpha_n x) |_{x=0}^{x=a} - \alpha_n \int_{0}^{a} \frac{\partial v(x,y)}{\partial y} cos(\alpha_n x) dx = \\
    &= -\alpha_n \frac{\partial}{\partial y} \int_{0}^{a} v(x,y) cos(\alpha_n x) dx = -\alpha_n v_n^{'}(y)
\end{align*}

Тоді перше рівняння у просторі трансформант прийме вигляд:
\begin{align*}
    u_n^{''}(y) - \alpha_n \mu_0 v_n^{'}(y) - \alpha_n^2 (1 + \mu_0) u_n(y) = 0
\end{align*}

Розлянемо друге рівняння
\begin{align*}
    &\int_{0}^{a} \frac{\partial^2 v(x,y)}{\partial x^2} cos(\alpha_n x)dx + \int_{0}^{a} \frac{\partial^2 v(x,y)}{\partial y^2} cos(\alpha_n x)dx + \\ 
    & + \mu_0 \left( \int_{0}^{a} \frac{\partial^2 u(x,y)}{\partial x \partial y} cos(\alpha_n x)dx +  \int_{0}^{a} \frac{\partial^2 v(x,y)}{\partial y^2} cos(\alpha_n x) dx\right)
\end{align*}

Розглянемо
\begin{align*}
    &\int_{0}^{a} \frac{\partial^2 v(x,y)}{\partial x^2} cos(\alpha_n x)dx = \frac{\partial v(x,y)}{\partial x} cos(\alpha_n x) |_{x=0}^{x=a} + \alpha_n \int_{0}^{a} \frac{\partial v(x,y)}{\partial x} sin(\alpha_n x) dx = \\
    &=\frac{\partial v(x,y)}{\partial x} cos(\alpha_n x) |_{x=0}^{x=a} + \alpha_n \left(v(x,y) sin(\alpha_n x)|_{x=0}^{x=a} - \alpha_n \int_{0}^{a} v(x,y) cos(\alpha_n x) dx  \right) = \\
    &= -\alpha_n^2 v_n(y)
\end{align*}

Розглянемо
\begin{align*}
    &\int_{0}^{a} \frac{\partial^2 v(x,y)}{\partial y^2} cos(\alpha_n x)dx = \frac{\partial^2}{\partial y^2} \int_{0}^{a} v(x,y) cos(\alpha_n x)dx = v_n^{''}(y)
\end{align*}

Розглянемо
\begin{align*}
    &\int_{0}^{a} \frac{\partial^2 u(x,y)}{\partial y \partial x} cos(\alpha_n x)dx = \frac{\partial u(x,y)}{\partial y} cos(\alpha_n x) |_{x=0}^{x=a} + \alpha_n \int_{0}^{a} \frac{\partial u(x,y)}{\partial y} sin(\alpha_n x) dx = \\
    &=\frac{\partial u(x,y)}{\partial y} cos(\alpha_n x) |_{x=0}^{x=a} + \alpha_n \frac{\partial}{\partial y} \int_{0}^{a} u(x,y) sin(\alpha_n x) dx = \alpha_n u_n^{'}(y)\\
\end{align*}

Тоді друге рівняння у просторі трансформант прийме вигляд:
\begin{align*}
    (1 + \mu_0) v_n^{''}(y) + \alpha_n \mu_0 u_n^{'}(y)  - \alpha_n^2 v_n(y) = 0
\end{align*}

В результаті отримаємо наступну систему рівнянь у просторі трансформант:
\begin{equation*}
    \begin{cases}
        u_n^{''}(y) - \alpha_n \mu_0 v_n^{'}(y) - \alpha_n^2 (1 + \mu_0) u_n(y) = 0 \\
        (1 + \mu_0) v_n^{''}(y) + \alpha_n \mu_0 u_n^{'}(y)  - \alpha_n^2 v_n(y) = 0 \\
    \end{cases}
\end{equation*}

\end{document}