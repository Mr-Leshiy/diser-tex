\documentclass[a4paper,14pt]{extarticle}
\usepackage{amsmath}
\usepackage[english,ukrainian]{babel}
\usepackage{nameref}
\usepackage{graphicx}
\usepackage[a4paper, left=30mm, right=30mm, top=20mm, bottom=20mm]{geometry}

\title{Diseration}
\author{}
\date{}

\numberwithin{equation}{section}
\counterwithin{figure}{section}

\begin{document}

\section*{\centering АНОТАЦІЯ}
\textit{Пожиленков О. В.}
Плоскі мішані задачі теорії пружності для прямокутної області. - Кваліфікаційна наукова праця на правах рукопису.

Дисертація на здобуття наукового доктора філософії за спеціальністю 113 «Прикладна математика» (11 – «Математика та статистика»). – Одеський національний університет імені І. І. Мечникова, Одеса, 2023.

Розв'язано плоскі мішані задачі теорії пружності для пружної прямокутної області яка піддається впливу статичних та динамічних навантажень.
Шляхом застосування інтегрального скінченного sin- та cos-перетворення Фур'є вихідну задачу зведено до одновимірної крайової задачі.
Яку у просторі трансформант переформульовано у вигляді векторної крайової задачі.
Розв'язок цієї задачі побудовано як суперпозицію загального розв'язку однорідного векторного рівняння та частинного розв'язку неоднорідного рівняння.
Розв'язок однорідного векторного рівняння отримано за допомогою матричного диференціального числення і зображено за допомогою фундаментальної матричної системи розв'язків відповідного однорідного матричного рівняння.
Частковий розв'язок неоднорідного векторного рівняння знайдено за допомогою зображення матриці-функція Гріна.
Застосування оберненного перетворення Фур'є та реалізація відокремлення слабко-збіжних частин інтегралу подає поле переміщень та напружень через невідому функцію - граничне значення переміщень по торцю прямокутною області.
Для її знаходження за умови виконання крайової умови отримано сингулярне інтегральне рівняння яке розв'язано за допомогою метода ортогональних поліномів.
Було проведено дослідження напруженого стану середовища за різних типів навантаження та різних геометричних розмірів прямокутної області.

\textit{Ключові слова:}
прямокутна область, динамічна задача, перетворення Фур’є, матриця-функція Гріна, сингулярне інтегральне рівняння, метод ортогональних поліномів.


\section*{\centering ABSTRACT}
\textit{Pozhylenkov O. V.}
Plane mixed problems of elasticity for a rectangular domain - Manuscript.

A dissertation submitted for the degree of Doctor of Philosophy in the field of 113 «Applied Mathematics» (11 - «Mathematics and Statistics») - Odessa I.I. Mechnikov National University, Odessa, 2023.

The plane mixed boundary value problems of theory elasticity were solved for an elastic rectangular domain subjected to static and dynamic loads.
By applying the integral finite sin-cos Fourier transformation, the original problem was reduced to a one-dimensional boundary value problem, which was reformulated in the form of a vector boundary value problem.
The solution to this problem was constructed as a superposition of the general solution of the homogeneous vector equation and the particular solution of the nonhomogeneous equation.
The solution of the homogeneous vector equation was obtained using matrix differential calculus and represented using the fundamental matrix solution system of the corresponding homogeneous matrix equation.
The particular solution of the nonhomogeneous vector equation was found using the Green's matrix function representation.
The application of inverse Fourier transformation and the realization of the separation of weakly convergent parts of the integral provide the field of displacements and stresses through an unknown function - the boundary value of displacements along the end of the rectangular region.
To find this function under the boundary condition, a singular integral equation was derived and solved using the method of orthogonal polynomials.
The stress state of the medium was investigated for various types of loading and different geometric dimensions of the rectangular region.

\textit{Key words:}
rectangular domain, dynamic problem, Fourier transformation, matrix Green's function, singular integral equation, method of orthogonal polynomials.

\section*{\centering СПИСОК ОПУБЛІКОВАНИХ ПРАЦЬ ЗА ТЕМОЮ ДИСЕРТАЦІЇ}
\begin{enumerate}
    \item D. Nerukh, O. Pozhylenkov, N. Vaysfeld (2019) Mixed plain boundary value problem of elasticity for a rectangular domain.
    25-th International Conference Engineering Mechanics. 2019, May 13-16, Svratka, Czech Republic. p. 255
    \item O. V. Pozhylenkov (2019) The stress state of a rectangular elastic domain. Researches in Mathematics and Mechanics, Volume 24, Issue 2(34), pp. 88-96
    \item Пожиленков О. В. Вайсфельд Н. Д. (2019) Мішана крайова задача теорії пружності для прямокутної області. Математичні проблеми механіки неоднорідних структур, випуск 5, Львів, ст. 30-32
    \item O. Pozhyenkov, N. Vaysfeld (2020) Stress state of a rectangular domain with the mixed boundary conditions. Procedia Structural Integrity, Volume 28, pp. 458-463
    \item O. Pozhyenkov, N. Vaysfeld (2021) Stress state of an elastic rectangular domain under steady load. Procedia Structural Integrity, Volume 33, pp. 385-390
    \item O. Pozhylenkov, N. Vaysfeld (2022) Dynamic mixed problem of elasticity for a rectangular domain. Recent trends in Wave Mechanics and Vibrations, pp. 211-218
\end{enumerate}

\newpage

\renewcommand{\contentsname}{\centering Зміст}
\tableofcontents

\newpage

\addcontentsline{toc}{section}{\protect\numberline{}Перелік умовних позначень}
\section*{\centering Перелік умовних позначень}
$G$ - коєфіцієент Ламе \newline
$E$ - молуль Юнга \newline
$\mu$ - коєфіцієнт Пуасона \newline
$c_1$, $c_2$ - швидкості хвилі \newline
$\omega$ - частота \newline
$\mu_0 = \frac{1}{1 - 2\mu}$ \newline
$U_x(x,y) = u(x,y)$ - переміщення по осі $x$ \newline
$U_y(x,y) = v(x,y)$ - переміщення по осі $y$
\newpage

\addcontentsline{toc}{section}{\protect\numberline{}Вступ}
\section*{\centering Вступ}
\textbf{Актуальність роботи.} 
Прямокутна пружна область є однією з найбільш простих для аналізу і моделювання у механіці пружного тіла.
Але для багатьох застосувань прямокутна форма може бути використана як апроксімація більш складних об'єктів, наприклад:
в інженерних розрахунках прямокутні пластини використовують для моделювання деталів конструкцій з більш складними формами,
такими як пластини з отворами та вирізами. Прямокутна форма може бути застосована для різних типів задач, а саме:
для моделювання пружного деформування у твердих тілах, дослідженню розподілу напружень та деформації у матеріалах,
а також для розв'язання задач пов'язанних з пружністю і біологічних та геологічних системах.
Саме тому розробка математичних аналітичних методів розв'язання задач для прямокутної пружної області залишаеться актуальної за своєю простотою,
універсальністю, практичністю на широкому спектрі застосувань, незважаючи на те, що існують більш складні та реалістичні моделі.

У даній роботі запропоновано використання інтегральних перетворень безпосередньо до рівнянь руху.
Цей підхід дозволяє побудувати аналітичний розв'язок вихідної крайової задачі у просторі трансформант
відносно шуканих переміщень. Для спрощення обчислень побудовано матрицю-функцію Гріна,
яку зображено як комбінацію фундаментальних базісних матриць.
Цей підхід був використано під час розв'язання динамічних та статичних мішаних задач теорії пружності для прямокутної області,
яка є модельним об'єктом для вивчення напружено-деформованого стану пружних матеріалів.

Незважаючи на те, що згідно з аналізом літератури, було встановлено досить широке коло дослідженних мішаних задач теорії пружності,
залишається наявність невирішених питань, що пов'язані з поведінкою напружень у кутових точках та їх розподілам у прямокутній області.
Залишається необхідним отримати загальну якісну картину поведінки хвильових полів, за умов динамічного навантаження.
Тому розробка нового аналітичного підходу до розв'язання плоских мішаних задач для прямокутної області залишається актуальною задачою.сті для прямокутної області.

\textbf{Зв’язок роботи з науковими програмами, планами, темами.}
Дисертаційна робота виконана в рамках держбюджетних тем Одеського національного університету імені І. І. Мечникова
«Статичні та динамічні задачі для тіл канонічної форми з дефектами»
(2021-2024 рр., реєстраційний номер 0121U111664).

\textbf{Мета і задачі дослідження.}
Головною метою цього дослідження є визначення взаємозв'язку між напружено-деформованим станом прямокутної області та різними крайовими умовами на його торцях.

Для досягнення поставленої мети необхідно виконати наступні завдання:
\begin{enumerate}
    \item розробка методики розв'язання плоских задач для прямокутної області, яка використовує безпосередні перетворення рівнянь рівноваги,
    спрямована на отримання розв'язку поставленої задачі для реальних механічних характеристик;
    \item побудові аналітичного розв'язку для задачі прямокутної області, яка піддається зовнішньому навантаженню різної конфігурації.
    Цей розв'язок отримується шляхом перетворення початкової задачі до одновимірної крайової задачі та використання матричної функції Гріна для її розв'язання.
    Основною метою є встановлення якісних та кількісних закономірностей для полів переміщень та напружень в цій задачі;
\end{enumerate}

\textit{\textbf{Об’єктом дослідження є}}
пружна прямокутна область під впливом зовнішнього навантаження різної природи (статичного та динамічного).

\textit{\textbf{Предметом дослідження є}}
закономірності зміни напружено-деформований стану та хвильового поля прямокутної області в залежності від видів навантаження.

\textbf{Методи дослідження.}
У даній дисертаційній роботі було розглянуто розв'язання мішаних динамічних та статичних задач теорії пружності для прямокутної області з використанням методу інтегральних перетворень,
що безпосередньо застосовується до рівнянь рівноваги. Цей підхід дозволяє зведення вихідної задачі до одновимірної крайової задачі у просторі трансформант,
відносно невідомих трансформант переміщень. Для розв'язання цієї задачі використовується матричне диференціальне числення та матриці-функції Гріна.
Задачу можна звести до одного сингулярного інтегрального рівняння. Для розв'язання якого застосовується спеціальний аналітично-числовий метод, який дозволяє враховувати рухомі та нерухомі особливості у ядрі інтегрального рівняння.

\textbf{Обґрунтованість та достовірність отриманих результатів} забезпечується:
використання точних математичних формулювань задач у лінійній механіці суцільного тіла та механіці руйнування;
використання перевірених і строгих аналітичних методів для отримання розв'язків сформульованих задач;
фізичною інтерпретацією результатів розрахунків задач.
Отримані результати збігаються з відомими результатами теоретичних досліджень.

\textbf{Наукова новизна} отриманих результатів полягає в наступному:
\begin{itemize}
    \item вперше була застосована нова методика розв'язання мішаних динамічних плоских задач теорії пружності для прямокутної області,
    що ґрунтується на безпосередньому перетворенні рівнянь рівноваги. Цей підхід дозволяє отримати аналітичні вирази для шуканих механічних характеристик;
    \item за допомогою побудови матриці-функції Гріна у формі комбінації фундаментальних базісних матриць та використання методу матричного диференціального числення,
    були отримані аналітичні розв'язки для мішаних плоских задач. В цих розв'язках задача була зведена до одного сингулярного інтегрального рівняння, що залежить від невідомого стрибка переміщень.
    Були встановлені особливості залежності полів переміщень та напружень від параметрів навантаження на короткому торці прямокутної області;
\end{itemize}

\textbf{Теоретичне і практичне значення одержаних результатів.} 
Запропонована методика для розв'язання мішаних динамічних задач теорії пружності для скінченної прямокутної області має важливе теоретичне значення для подальшого розвитку математичних методів вирішення плоских задач теорії пружності.
Отримані результати стали складовою частиною курсів "Теорія пружності" та "Додаткові глави методів математичної фізики" і були використані студентами, які навчаються за спеціальністю "Прикладна математика", у написанні їх магістерських і дипломних робіт.
Отримані результати також можуть знайти застосування в геомеханіці, будівництві конструкцій, вивченні міцності елементів транспортних засобів та визначенні їх безпечності та в інших сферах.

\textbf{Особистий внесок здобувача.}
Основні результати дисертаційної роботи отримано здобувачем самостійно.
У роботах у співавторстві \cite{pozhylenkov_1, pozhylenkov_2, pozhylenkov_3, pozhylenkov_4, pozhylenkov_5, pozhylenkov_6},
науковому керівнику належить постановка задач, вибір методики їх розв’язання.
Дисертантом проведено огляд літератури, виконано усі математичні перетворення при побудові розв’язків,
здійснено програмну реалізацію та проведено аналіз отриманих результатів.

\textbf{Апробація результатів дисертації.}

Результати досліджень, які були включені до дисертаційної роботи, були представлені і обговорені на різних міжнародних наукових конференціях:
\begin{itemize}
    \item конференція  <<Актуальные вопросы и перспективы развития транспортного и строительного комплексов>> (Білорусь, Гомель, 2018);
    \item Х Міжнародна наукова конференція <<Математичні проблеми механіки неоднорідних структур>> (Львів, 2019);
    \item 25-th international conference <<Engineering Mechanics 2019>> (Czech Republic, Svratka, 2019);
    \item <<1st Virtual European Conference on Fracture>> (Italy, 2020);
    \item <<26th International Conference on Fracture and Structural Integrity>> (Italy, Turin, 2021);
    \item <<10th International Conference on Wave Mechanics and Vibrations>> (Portugal, Lisbon, 2022)
\end{itemize}

\textbf{Публікації.}
Основні наукові положення дисертаційного дослідження відображено у 6 публікаціях,
дві статті \cite{pozhylenkov_2,pozhylenkov_3} опубліковано у провідних фахових виданнях України, що входить у перелік ДАК України,
статті \cite{pozhylenkov_1,pozhylenkov_4,pozhylenkov_5,pozhylenkov_6} прореферовано у міжнародній наукометричній базі Scopus.

\textbf{Структура і обсяг дисертації.}
???
Приблизний текст (Робота складається зі вступу, 5 розділів, висновків, списку використаної літератури, що включає 173 найменування. Загальний обсяг дисертації становить 160 сторінок, із них 119 сторінок основного тексту. Робота містить 68 рисунків та 1 таблицю.)

\textbf{Подяка.}
Автор висловлює глибоку подяку своєму першому вчителеві професору та науковому керівнику Н. Д. Вайсфельд, який сприяв виникненню його інтересу до математичних проблем механіки та визначив напрям наукових досліджень.
Дисертант висловлює щиру вдячність кандидату фізико-математичних наук, доценту Ю. С. Процерову за цінні наукові поради, що допомогли успішному проведенню досліджень.
\newpage

\addcontentsline{toc}{section}{\protect\numberline{}Огляд літератури}
\section*{\centering Огляд літератури}

Дослідження напружено-деформованого стану пружних тіл почало активно розвиватися у XIX столітті і залишається актуальним до нашого часу.
Це пояснюється широким спектром застосування в різноманітних інженерних галузях.
Класична лінійна теорія пружності є основою для більшості міцностних розрахунків в техніці.
Під час експлуатації будівель та інших конструкцій вони піддаються механічним, температурним та іншим впливам.
Тому при проектуванні необхідно враховувати міцність таких конструкцій. Характеристики міцності виробів можна отримати шляхом аналізу напружено-деформованого стану їх пружних моделей.
Одним з таких моделей є скінченна прямокутна область.
Тому актуальною проблемою є розробка аналітично-числових методів для дослідження її напружено-деформованого стану.
\newpage

\section{МЕТОДИКА ПОБУДОВИ РОЗВ’ЗКІВ МІШАНИХ ЗАДАЧ ТЕОРІЇ ПРУЖНОСТІ ДЛЯ ПРЯМОКУТНОЇ ОБЛАСТІ}
У даному розділі наведено опис аналітичного апарату, який використовується для розв'язання мішаних задач теорії пружності для прямокутної області.
Цей підхід базується на результати раніше проведених досліджень, зокрема робіт \cite{popov_1} і \cite{popov_2}.
Розглянута методика розв'язання мішаних плоских задач ґрунтується на застосуванні інтегральних перетворень безпосередньо до системи рівнянь рівноваги Ламе та крайових умов.
Це дозволяє зводити вихідну задачу до векторної одновимірної крайової задачі.
Векторна одновимірна крайова задача точно розв'язується за допомогою матричного диференційного числення та матричної функції Гріна.
Що призводить у результаті до сингулярного інтегрального рівнняння яке розв'язане за допомогою методу ортогональних многочленів описанного \cite{popov_3}.

\subsection{Постановка задачі}
\begin{figure}[h]
    \begin{center}
        \includegraphics[scale=1]{images/geometry/image_4.jpg}
    \end{center}
    \caption{Геометрія проблеми}\label{geom_gen}
\end{figure}
Розглядається пружна прямокутна область (Рис: \ref{geom_gen}), яка займає область, що у декартовій системі координат описується співвідношенням $0 \le x \le a$, $0 \le y \le b$.

До прямокутної області на грані $y=b$ додане нормальне навантаження
\begin{equation}
    \sigma_y(x, y, t) |_{y=b} = -p(x, t), \quad  \tau_{xy}(x,y,t) |_{y=b} =0, \quad 0 \le x \le a
\end{equation}
де $p(x, t)$ відома функція.
На нижній грані виконуються наступні умови
\begin{equation}
    v(x,y,t) |_{y=0}, \quad \tau_{xy}(x,y,t) |_{y=0} =0
\end{equation}
На бічних гранях $x=0$ та $x=a$ граничні умови запишемо у формі
\begin{equation}\label{gen_bound_gen}
    U_1[f(x,y,t)]=0, \quad U_2[f(x,y,t)]=0 , \quad 0 \le y \le b
\end{equation}
Де 
\begin{align*}
    &U_1[f(x,y,t)]=\left[\alpha_1f(x,y,t) + \beta_1 \frac{\partial f(x,y,t)}{\partial x} \right]|_{x=0} \\
    &U_2[f(x,y,t)]=\left[\alpha_2f(x,y,t) + \beta_2 \frac{\partial f(x,y,t)}{\partial x} \right]|_{x=a} \\
\end{align*}
граничні функціонали у загальному виді (для кожної конкретної задачі вони будуть деталізовані), $f(x,y,t)=(u(x,y,t), v(x,y,t))^T$ - вектор переміщеннь.

Розглядаються наступні рівняння рівноваги Ламе:
\begin{equation}
    \begin{cases}
        \frac{\partial^2 u(x,y,t)}{\partial x^2} + \frac{\partial^2 u(x,y,t)}{\partial y^2} + \mu_0 (\frac{\partial^2 u(x,y,t)}{\partial x^2} + \frac{\partial^2 v(x,y,t)}{\partial x\partial y}) = \frac{1}{c_1^2} \frac{\partial^2 u(x,y,t)}{\partial t^2} \\
        \frac{\partial^2 v(x,y,t)}{\partial x^2} + \frac{\partial^2 v(x,y,t)}{\partial y^2} + \mu_0 (\frac{\partial^2 u(x,y,t)}{\partial x \partial y} + \frac{\partial^2 v(x,y,t)}{\partial y^2}) = \frac{1}{c_2^2} \frac{\partial^2 v(x,y,t)}{\partial t^2} \\
    \end{cases}
\end{equation}

Будемо розглядати випадок гармонічних коливань, тому можемо предствавити функції у наступному вигляді:
\begin{equation}
    u(x,y,t) = u(x,y) e^{i \omega t}, \quad v(x,y,t) = v(x,y) e^{i \omega t}, \quad p(x,t) = p(x) e^{i \omega t}
\end{equation}
Таким чином отримаємо наступні рівняння рівноваги:
\begin{equation}\label{lame_gen}
    \begin{cases}
        \frac{\partial^2 u(x,y)}{\partial x^2} + \frac{\partial^2 u(x,y)}{\partial y^2} + \mu_0 (\frac{\partial^2 u(x,y)}{\partial x^2} + \frac{\partial^2 v(x,y)}{\partial x\partial y}) = -\frac{\omega^2}{c_1^2}  u(x,y) \\
        \frac{\partial^2 v(x,y)}{\partial x^2} + \frac{\partial^2 v(x,y)}{\partial y^2} + \mu_0 (\frac{\partial^2 u(x,y)}{\partial x \partial y} + \frac{\partial^2 v(x,y)}{\partial y^2}) = -\frac{\omega^2}{c_2^2} v(x,y) \\
    \end{cases}
\end{equation}
Та граничні умови:
\begin{equation}\label{bound_gen}
    \begin{cases}
        \sigma_y(x, y) |_{y=b} = -p(x), \quad  \tau_{xy}(x,y) |_{y=b} =0 \\
        v(x,y) |_{y=0}, \quad \tau_{xy}(x,y) |_{y=0} =0 \\
        U_1[f(x,y)]=0, \quad U_2[f(x,y)]=0
    \end{cases}
\end{equation}

Введемо невідомі функції $\chi_1(y) = u(0, y)$, $\chi_2(y) = v(0, y)$, $\chi_3(y) = u(a, y)$, $\chi_4(y) = v(a, y)$.
Враховучи умову \eqref{gen_bound_gen}, отримаємо, що 
$\frac{\partial u(0, y)}{\partial x}=-\frac{\alpha_1}{\beta_1} \chi_1(y)$,
$\frac{\partial v(0, y)}{\partial x}=-\frac{\alpha_1}{\beta_1} \chi_2(y)$,
$\frac{\partial u(a, y)}{\partial x}=-\frac{\alpha_2}{\beta_2} \chi_3(y)$,
$\frac{\partial v(a, y)}{\partial x}=-\frac{\alpha_2}{\beta_2} \chi_4(y)$.
Отже умова \eqref{gen_bound_gen} виконується автоматично.

\subsection{Зведення задачі до одновимірної у просторі трансформант та її розв'язання}\label{to_one_dimensional}
Для того, щоб звести задачу до одновимірної задачі, використаємо інтегральне перетворення Фур'є по змінній $x$ до рівнянь \eqref{lame_gen} в наступному вигляді:
\begin{equation}
    \begin{pmatrix}
        u_n(y) \\
        v_n(y)
    \end{pmatrix} = \int_{0}^{a} 
    \begin{pmatrix}
        u(x,y) sin(\alpha_n x) \\
        v(x,y) cos(\alpha_n x)
    \end{pmatrix} dx, \quad \alpha_n = \frac{\pi n}{a}
\end{equation}

Для цього помножимо перше та друге рівняння \eqref{lame_gen} на $sin(\alpha_n x)$ та $cos(\alpha_n x)$ відповідно та проінтегруємо по змінній $x$ на інтервалі $0 \le x \le a$.

Розглянемо перше рівняння:
\begin{align*}
    &\int_{0}^{a} \frac{\partial^2 u(x,y)}{\partial x^2} sin(\alpha_n x)dx + \int_{0}^{a} \frac{\partial^2 u(x,y)}{\partial y^2} sin(\alpha_n x)dx + \\ 
    & + \mu_0 \left( \int_{0}^{a} \frac{\partial^2 u(x,y)}{\partial x^2} sin(\alpha_n x)dx +  \int_{0}^{a} \frac{\partial^2 v(x,y)}{\partial x \partial y} sin(\alpha_n x) dx\right) + \\
    & + \frac{\omega^2}{c_1^2} \int_{0}^{a} u(x,y) sin(\alpha_n x)dx = 0
\end{align*}

Розглянемо
\begin{align*}
    &\int_{0}^{a} \frac{\partial^2 u(x,y)}{\partial x^2} sin(\alpha_n x)dx = \frac{\partial u(x,y)}{\partial x} sin(\alpha_n x) |_{x=0}^{x=a} - \alpha_n \int_{0}^{a} \frac{\partial u(x,y)}{\partial x} cos(\alpha_n x)dx = \\
    &= \frac{\partial u(x,y)}{\partial x} sin(\alpha_n x) |_{x=0}^{x=a} - \alpha_n \left( u(x,y) cos(\alpha_n x) |_{x=0}^{x=a} + \alpha_n \int_{0}^{a} u(x,y) sin(\alpha_n x) dx \right) = \\
    &=-\alpha_n(\chi_3(y) cos(\alpha_n a) - \chi_1(y)) -\alpha_n^2 u_n(y)
\end{align*}

Розглянемо
\begin{align*}
    &\int_{0}^{a} \frac{\partial^2 u(x,y)}{\partial y^2} sin(\alpha_n x)dx = \frac{\partial^2}{\partial y^2} \int_{0}^{a} u(x,y) sin(\alpha_n x)dx = u_n^{''}(y)
\end{align*}

Розглянемо
\begin{align*}
    &\int_{0}^{a} \frac{\partial^2 v(x,y)}{\partial x \partial y} sin(\alpha_n x) dx = \frac{\partial v(x,y)}{\partial y} sin(\alpha_n x) |_{x=0}^{x=a} - \alpha_n \int_{0}^{a} \frac{\partial v(x,y)}{\partial y} cos(\alpha_n x) dx = \\
    &= -\alpha_n \frac{\partial}{\partial y} \int_{0}^{a} v(x,y) cos(\alpha_n x) dx = -\alpha_n v_n^{'}(y)
\end{align*}

Тоді перше рівняння у просторі трансформант прийме вигляд:
\begin{align*}
    &u_n^{''}(y) - \alpha_n \mu_0 v_n^{'}(y) -(\alpha_n^2 + \alpha_n^2 \mu_0 - \frac{\omega^2}{c_1^2}) u_n(y) = \\
    &= \alpha_n(1 + \mu_0)(\chi_3(y) cos(\alpha_n a) - \chi_1(y))
\end{align*}

Розлянемо друге рівняння
\begin{align*}
    &\int_{0}^{a} \frac{\partial^2 v(x,y)}{\partial x^2} cos(\alpha_n x)dx + \int_{0}^{a} \frac{\partial^2 v(x,y)}{\partial y^2} cos(\alpha_n x)dx + \\ 
    & + \mu_0 \left( \int_{0}^{a} \frac{\partial^2 u(x,y)}{\partial x \partial y} cos(\alpha_n x)dx +  \int_{0}^{a} \frac{\partial^2 v(x,y)}{\partial y^2} cos(\alpha_n x) dx\right) + \\
    & + \frac{\omega^2}{c_2^2} \int_{0}^{a} v(x,y) cos(\alpha_n x)dx = 0
\end{align*}

Розглянемо
\begin{align*}
    &\int_{0}^{a} \frac{\partial^2 v(x,y)}{\partial x^2} cos(\alpha_n x)dx = \frac{\partial v(x,y)}{\partial x} cos(\alpha_n x) |_{x=0}^{x=a} + \alpha_n \int_{0}^{a} \frac{\partial v(x,y)}{\partial x} sin(\alpha_n x) dx = \\
    &=\frac{\partial v(x,y)}{\partial x} cos(\alpha_n x) |_{x=0}^{x=a} + \alpha_n \left(v(x,y) sin(\alpha_n x)|_{x=0}^{x=a} - \alpha_n \int_{0}^{a} v(x,y) cos(\alpha_n x) dx  \right) = \\
    &= -(\frac{\alpha_2}{\beta_2}\chi_4(y) cos(\alpha_n a) - \frac{\alpha_1}{\beta_1}\chi_2(y)) -\alpha_n^2 v_n(y)
\end{align*}

Розглянемо
\begin{align*}
    &\int_{0}^{a} \frac{\partial^2 v(x,y)}{\partial y^2} cos(\alpha_n x)dx = \frac{\partial^2}{\partial y^2} \int_{0}^{a} v(x,y) cos(\alpha_n x)dx = v_n^{''}(y)
\end{align*}

Розглянемо
\begin{align*}
    &\int_{0}^{a} \frac{\partial^2 u(x,y)}{\partial y \partial x} cos(\alpha_n x)dx = \frac{\partial u(x,y)}{\partial y} cos(\alpha_n x) |_{x=0}^{x=a} + \alpha_n \int_{0}^{a} \frac{\partial u(x,y)}{\partial y} sin(\alpha_n x) dx = \\
    &=\frac{\partial u(x,y)}{\partial y} cos(\alpha_n x) |_{x=0}^{x=a} + \alpha_n \frac{\partial}{\partial y} \int_{0}^{a} u(x,y) sin(\alpha_n x) dx = \alpha_n u_n^{'}(y) + \\
    &+(\chi_3^{'}(y) cos(\alpha_n a) -\chi_1^{'}(y))
\end{align*}

Тоді друге рівняння у просторі трансформант прийме вигляд:
\begin{align*}
    &(1 + \mu_0) v_n^{''}(y) + \alpha_n \mu_0 u_n^{'}(y)  - (\alpha_n^2 -  \frac{\omega^2}{c_2^2}) v_n(y) = \\ 
    &= (\frac{\alpha_2}{\beta_2}\chi_4(y) cos(\alpha_n a) - \frac{\alpha_1}{\beta_1}\chi_2(y)) - \mu_0 (\chi_3^{'}(y) cos(\alpha_n a) -\chi_1^{'}(y))
\end{align*}

Отримана система рівнянь задачі у просторі трансформант:
\begin{equation}\label{transf_gen}
    \begin{cases}
        u_n^{''}(y) - \alpha_n \mu_0 v_n^{'}(y) - (\alpha_n^2 + \alpha_n^2 \mu_0 - \frac{\omega^2}{c_1^2}) u_n(y) = \\
        = \alpha_n(1 + \mu_0)(\chi_3(y) cos(\alpha_n a) - \chi_1(y)) \\
        \\
        (1 + \mu_0) v_n^{''}(y) + \alpha_n \mu_0 u_n^{'}(y) - (\alpha_n^2 - \frac{\omega^2}{c_2^2}) v_n(y) = \\
        = (\frac{\alpha_2}{\beta_2}\chi_4(y) cos(\alpha_n a) - \frac{\alpha_1}{\beta_1}\chi_2(y)) - \mu_0 (\chi_3^{'}(y) cos(\alpha_n a) -\chi_1^{'}(y))
    \end{cases}
\end{equation}

Застосовуючи інтегральне перетворення до граничних умов,
отримаємо наступні умови задачі у просторі трансформант:
\begin{equation}\label{transf_bound_gen}
    \begin{cases}
        \left( (2G + \lambda)v_n^{'}(y) + \alpha_n \lambda u_n(y) \right)|_{y=b} = -p_n \\
        \left(u_n^{'}(y) - \alpha_n v_n(y)  \right)|_{y=b} = 0 \\
        v_n(y)|_{y=0} = 0 \\
        \left(u_n^{'}(y) - \alpha_n v_n(y)  \right)|_{y=0} = 0
    \end{cases}
\end{equation}
де $p_n = \int_{0}^{a} p(x) cos(\alpha_n x) dx$

% \subsection{Зведення задачі у просторі трансформант до матрично-векторної форми}
Для того щоб розв'язати задачу у простосторі трансформант, перепишмо її у матрично-векторній формі.
Рівняння рівноваги \eqref{transf_gen} запишемо у наступному вигляді:
\begin{align}\label{transf_mat_gen}
    &L_2\left[ Z_n(y) \right] = A * Z_n^{''}(y) + B * Z_n^{'}(y) + C * Z_n(y) \nonumber \\
    &L_2\left[ Z_n(y) \right] = F_n(y)
\end{align}
Де
\begin{equation*}
    A = \begin{pmatrix}
        1 & 0 \\
        0 & 1 + \mu_0
    \end{pmatrix}, \quad
    B = \begin{pmatrix}
        0 & -\alpha_n \mu_0 \\
        \alpha_n \mu_0 & 0
    \end{pmatrix}
\end{equation*}
\begin{equation*}
    C = \begin{pmatrix}
        -\alpha_n^2 -\alpha_n^2 \mu_0 + \frac{\omega^2}{c_1^2} & 0 \\
        0 & -\alpha_n^2 + \frac{\omega^2}{c_2^2}
    \end{pmatrix}, \quad
    Z_n(y) = \begin{pmatrix}
        u_n(y) \\
        v_n(y)
    \end{pmatrix}
\end{equation*}
\begin{equation*}
    F_n(y) = \begin{pmatrix}
        \alpha_n(1 + \mu_0)(\chi_3(y) cos(\alpha_n a) - \chi_1(y)) \\
        (\frac{\alpha_2}{\beta_2}\chi_4(y) cos(\alpha_n a) - \frac{\alpha_1}{\beta_1}\chi_2(y)) - \mu_0 (\chi_3^{'}(y) cos(\alpha_n a) -\chi_1^{'}(y))
    \end{pmatrix}
\end{equation*}
Граничні умови \eqref{transf_bound_gen} запишемо у наступному вигляді:
\begin{align}\label{transf_bound_mat_gen}
    &U_i\left[ Z_n(y) \right] = E_i * Z_n^{'}(b_i) + F_i * Z_n(b_i) \nonumber \\
    &U_i\left[ Z_n(y) \right] = D_i
\end{align}
де $i = \overline{0, 1}$, $b_0 = b$, $b_1 = 0$,
\begin{equation*}
    E_0 = \begin{pmatrix}
        1 & 0 \\
        0 & 2G + \lambda
    \end{pmatrix}, \quad
    F_0 = \begin{pmatrix}
        0 & -\alpha_n \\
        \alpha_n \lambda & 0
    \end{pmatrix}, \quad
\end{equation*}
\begin{equation*}
    E_1 = \begin{pmatrix}
        1 & 0 \\
        0 & 0
    \end{pmatrix}, \quad
    F_1 = \begin{pmatrix}
        0 & -\alpha_n \\
        0 & 1
    \end{pmatrix}, \quad
\end{equation*}
\begin{equation*}
    D_0 = \begin{pmatrix}
        0 \\
        -p_n
    \end{pmatrix}, \quad
    D_1 = \begin{pmatrix}
        0 \\
        0
    \end{pmatrix}, \quad
\end{equation*}

Для знаходження розв'язку задачі у просторі трансформант, знайдем фундаментальну матрицю рівняння \eqref{transf_mat_gen}.
Шукати її будем у наступному вигляді \cite{gantmaher}:
\begin{equation}
    Y(y) = \frac{1}{2\pi i} \oint_C e^{sy} M^{-1}(s)ds
\end{equation}
де $M(s)$ - характерестична матриця рівняння \eqref{transf_mat_gen}, а $C$ - замкнений контур який містить усі особливі точки $M^{-1}(s)$. $M(s)$ будемо шукати з наступної умовни
\begin{equation}
    L_2\left[ e^{sy}*I \right] = e^{sy} * M(s), \quad I = \begin{pmatrix} 1 & 0 \\ 0 & 1 \end{pmatrix}
\end{equation}
\begin{align*}
    &L_2\left[ e^{sy}*I \right] = e^{sy} \left( s^2A * I + s B*I + C*I \right) = \\
    &=e^{sy} \begin{pmatrix}
        s^2 - \alpha_n^2 - \alpha_n^2\mu_0 + \frac{\omega^2}{c_1^2} & -\alpha_n \mu_0 s \\
        \alpha_n \mu_0 s & s^2 (1 + \mu_0) -\alpha_n^2 + \frac{\omega^2}{c_1^2}
     \end{pmatrix} \Rightarrow
\end{align*}

\begin{equation}
    M(s) = \begin{pmatrix}
        s^2 - \alpha_n^2 - \alpha_n^2\mu_0 + \frac{\omega^2}{c_1^2} & -\alpha_n \mu_0 s \\
        \alpha_n \mu_0 s & s^2 (1 + \mu_0) -\alpha_n^2 + \frac{\omega^2}{c_2^2}
     \end{pmatrix}
\end{equation}

Знайдемо тепер $M^{-1}(s) = \frac{\widetilde{M(s)}}{det[M(s)]}$.
\begin{equation}
    \widetilde{M(s)} = \begin{pmatrix}
        s^2 (1 + \mu_0) -\alpha_n^2 + \frac{\omega^2}{c_2^2} & \alpha_n \mu_0 s \\
        -\alpha_n \mu_0 s & s^2 - \alpha_n^2 - \alpha_n^2\mu_0 + \frac{\omega^2}{c_1^2}
     \end{pmatrix}
\end{equation}
\begin{align}
    &det[M(s)] = \begin{vmatrix}
        s^2 - \alpha_n^2 - \alpha_n^2\mu_0 + \frac{\omega^2}{c_1^2} & -\alpha_n \mu_0 s \\
        \alpha_n \mu_0 s & s^2 (1 + \mu_0) -\alpha_n^2 + \frac{\omega^2}{c_2^2}
     \end{vmatrix} = \nonumber \\
    &=(s - s_1)(s + s_1)(s - s_2)(s + s_2)
\end{align}
де $s_1$, $s_2$, $-s_1$, $-s_2$ корені $det[M(s)]=0$, детальне знаходження яких наведено в (\nameref{ap_B}).

Враховучи це, знайдемо значення фундаментальної матрицю за допомогою теореми про лишки:
\begin{align*}
    &\frac{1}{2\pi i} \oint_C e^{sy} M^{-1}(s)ds = \frac{2 \pi i}{2 \pi i (1 + \mu_0)} \sum_{i=1}^{4} Res\left[ e^{sy} \frac{\widetilde{M(s)}}{det[M(s)]} \right] = \\
    & = \left(Y_0(y) + Y_1(y) + Y_2(y) + Y_3(y) \right)
\end{align*}
Знайдемо $Y_0(y)$:
\begin{align}
    &Y_0(y) =  \left( \frac{e^{sy}}{(s+s_1)(s - s_2)(s + s_2)} \widetilde{M(s)} \right) \Big|_{s=s_1} = \nonumber \\
    &=\frac{e^{s_1 y}}{2s_1 (s_1^2 - s_2^2)} \begin{pmatrix}
        s_1^2 (1 + \mu_0) -\alpha_n^2 + \frac{\omega^2}{c_2^2} & \alpha_n \mu_0 s_1 \\
        -\alpha_n \mu_0 s_1 & s_1^2 - \alpha_n^2 - \alpha_n^2\mu_0 + \frac{\omega^2}{c_1^2}
    \end{pmatrix}
\end{align}
Знайдемо $Y_1(y)$:
\begin{align}
    &Y_1(y) =  \left( \frac{e^{sy}}{(s-s_1)(s - s_2)(s + s_2)} \widetilde{M(s)} \right) \Big|_{s=-s_1} = \nonumber \\
    &=-\frac{e^{-s_1 y}}{2s_1 (s_1^2 - s_2^2)} \begin{pmatrix}
        s_1^2 (1 + \mu_0) -\alpha_n^2 + \frac{\omega^2}{c_2^2} & -\alpha_n \mu_0 s_1 \\
        \alpha_n \mu_0 s_1 & s_1^2 - \alpha_n^2 - \alpha_n^2\mu_0 + \frac{\omega^2}{c_1^2}
    \end{pmatrix}
\end{align}
Знайдемо $Y_2(y)$:
\begin{align}
    &Y_2(y) =  \left( \frac{e^{sy}}{(s+s_2)(s - s_1)(s + s_1)} \widetilde{M(s)} \right) \Big|_{s=s_2} = \nonumber \\
    &=\frac{e^{s_2 y}}{2s_2 (s_2^2 - s_1^2)} \begin{pmatrix}
        s_2^2 (1 + \mu_0) -\alpha_n^2 + \frac{\omega^2}{c_2^2} & \alpha_n \mu_0 s_2 \\
        -\alpha_n \mu_0 s_2 & s_2^2 - \alpha_n^2 - \alpha_n^2\mu_0 + \frac{\omega^2}{c_1^2}
    \end{pmatrix}
\end{align}
Знайдемо $Y_3(y)$:
\begin{align}
    &Y_3(y) =  \left( \frac{e^{sy}}{(s-s_2)(s - s_1)(s + s_1)} \widetilde{M(s)} \right) \Big|_{s=-s_2} = \nonumber \\
    &=-\frac{e^{-s_2 y}}{2s_2 (s_2^2 - s_1^2)} \begin{pmatrix}
        s_2^2 (1 + \mu_0) -\alpha_n^2 + \frac{\omega^2}{c_2^2} & -\alpha_n \mu_0 s_2 \\
        \alpha_n \mu_0 s_2 & s_2^2 - \alpha_n^2 - \alpha_n^2\mu_0 + \frac{\omega^2}{c_1^2}
    \end{pmatrix}
\end{align}

\subsection{Побудова матриці-функції Гріна}
Для побудови матриці-функції Гріна спочатку знайдемо тепер фундамельні бизисні матриці $\Psi_0(y)$, $\Psi_1(y)$, шукати їх будем у наступному вигляді:
\begin{equation}\label{psi_gen}
    \Psi_i(y) = \left( Y_0(y) + Y_1(y) \right) * C_1^i + \left( Y_2(y) + Y_3(y) \right) * C_2^i
\end{equation}

Залишилось знайти невідомі матриці коєфіцієнтів $C_1^0$, $C_2^0$, $C_1^1$, $C_2^1$ використовуючи граничні умови \eqref{transf_bound_mat_gen}.
Покрокове знаходження яких наведено у (\nameref{ap_C}).
Для подальшого введемо наступні позначення для елементів матриць $\Psi_0(y)$, $\Psi_1(y)$:
\begin{equation*}
    \Psi_0(y) = \begin{pmatrix}
        \Psi_1^0(y) &  \Psi_2^0(y) \\
        \Psi_3^0(y) &  \Psi_4^0(y) 
    \end{pmatrix}, \quad 
    \Psi_1(y) = \begin{pmatrix}
        \Psi_1^1(y) &  \Psi_2^1(y) \\
        \Psi_3^1(y) &  \Psi_4^1(y) 
    \end{pmatrix}      
\end{equation*}

Таким чином матрицю Гріна можемо записати у вигляді:
\begin{equation}
    G(y,\xi) = 
    \begin{cases}
        \Psi_0(y) * \Psi_1(\xi), \quad 0 \le y < \xi \\
        \Psi_1(y) * \Psi_0(\xi), \quad \xi < y \le b
    \end{cases}
\end{equation}

Для данної матриці Гріна виконано усі властивості, зокрема виконані однорідні граничні умови \eqref{transf_bound_mat_gen}
та однорідні рівняння рівноваги у просторі трансформант \eqref{transf_mat_gen}:
\begin{equation*}
    L_2\left[  G(y, \xi) \right] = 0
\end{equation*}
\begin{equation*}
    U_0\left[ G(y, \xi) \right] = 0, \quad  U_1\left[ G(y, \xi) \right] = 0,
\end{equation*}

Таким чином ми можемо записати розв'язок крайової задачі у просторі трансформант:
\begin{equation}
    Z_n(y) = \int_0^b G(y,\xi) F_n(\xi) d\xi + \Psi_0(y) * D_0 + \Psi_1(y) * D_1
\end{equation}

Введемо наступні позначення $G(y, \xi) = \begin{pmatrix}
    g_1(y,\xi) & g_2(y,\xi) \\
    g_3(y,\xi) & g_4(y,\xi)
\end{pmatrix}$, $F_n(y) = \begin{pmatrix}
    f_n^1(y) \\
    f_n^2(y)
\end{pmatrix}$. Враховуючи це, шукані функціі перемішень у просторі трансформант можна записати у наступному вигляді
\begin{align}\label{transf_sol_u_gen}
    &u_n(y) = \int_0^b \left[g_1(y, \xi)f_n^1(\xi) + g_2(y, \xi)f_n^2(\xi) \right]d\xi - \psi_0^2(y) p_n
\end{align}
\begin{align}\label{transf_sol_v_gen}
    &v_n(y) = \int_0^b \left[g_3(y, \xi)f_n^1(\xi) + g_4(y, \xi)f_n^2(\xi) \right]d\xi - \psi_0^4(y) p_n
\end{align}

% \subsection{Побудова розв'язоку вихідної задачі}
Викорустовуючи обернене інтегральне перетворення Фур'є до розв'язку задачі у просторі трансформант
(\ref{transf_sol_u_gen}), (\ref{transf_sol_v_gen}), отримаємо фінальний розв'язок задачі
\begin{equation}
    u(x,y) = \frac{2}{a} \sum_{n=1}^{\infty} u_n(y) sin(\alpha_n x), \quad \alpha_n = \frac{\pi n}{a}
\end{equation}
\begin{equation}
    v(x,y) = \frac{v_0(y)}{a} + \frac{2}{a} \sum_{n=1}^{\infty} v_n(y) cos(\alpha_n x), \quad \alpha_n = \frac{\pi n}{a}
\end{equation}

Знайдем тепер $v_0(y)$ розглянувши задачу у просторі трансформант \eqref{transf_gen}, \eqref{transf_bound_gen} при $n=0$, $\alpha_n = 0$.
Детальний розв'язок якої наведено в (\nameref{ap_D}). Тоді остаточний розв'язок $v(x,y)$ буде мати вигляд
\begin{align}
    &v(x,y) = \frac{2}{a} \sum_{n=1}^{\infty} v_n(y) cos(\alpha_n x) - \psi_0(y) \frac{p_0}{a(2G + \lambda)} + \nonumber \\
    &+ \frac{1}{a(1+\mu_0)} \int_{0}^{b}g(y,\xi) [ (\frac{\alpha_2}{\beta_2}\chi_4(\xi) cos(\alpha_n a) - \frac{\alpha_1}{\beta_1}\chi_2(\xi)) - \nonumber \\
    & - \frac{\mu_0}{(1+\mu_0)} (\chi_3^{'}(\xi) cos(\alpha_n a) -\chi_1^{'}(\xi)) ] d\xi
\end{align}

Залишилось знайти невідомі функції $\chi_1(y)$, $\chi_2(y)$, $\chi_3(y)$, $\chi_4(y)$.
В подальшому в данній роботі розглянуто випадок таких граничних умов які призводять лише до однієї невідомої функції $f(y) = \frac{\partial v(x,y)}{\partial x}|_{x=a}$.
Для знаходження якої буде побудовано інтегральне рівняння завдяки граничній умові $\sigma_y(x, y) |_{y=b} = -p(x)$.

\subsection{Загальна схема розв'язку сінгулярного інтегрального рівняння}
Розглянемо випадок граничних умов другої основної задачі теорії пружності, в результаті отримаємо лише одну невідому функцію $f(y) = \frac{\partial v(x,y)}{\partial x}|_{x=a}$.
З цього отримаємо значення $f_n^1(\xi) = 0$, $f_n^2(\xi)= -cos(\alpha_n a) f(\xi)$.
Запишемо тепер фінальний розв'язок для цього випадку:
\begin{equation}
    u(x,y) = -\frac{2}{a} \sum_{n=1}^{\infty} \left( \int_0^b \left[g_2(y, \xi)cos(\alpha_n a) f(\xi) \right]d\xi + \psi_0^2(y) p_n \right) sin(\alpha_n x)
\end{equation}
\begin{align}
    &v(x,y) = -\frac{1}{a(1+\mu_0)} \int_{0}^{b}g(y,\xi) f(\xi) d\xi - \psi_0(y) \frac{p_0}{a(2G + \lambda)} \nonumber \\
    &- \frac{2}{a} \sum_{n=1}^{\infty} \left( \int_0^b \left[g_4(y, \xi) cos(\alpha_n a) f(\xi) \right]d\xi + \psi_0^4(y) p_n  \right) cos(\alpha_n x)
\end{align}

Використиємо граничну умову $\sigma_y(x, y) |_{y=b} = -p(x)$ для того, щоб отримати інтегральне рівняння:
\begin{equation*}
    (2G + \lambda)\frac{\partial v(x,y)}{\partial y}|_{y=b} + \lambda\frac{\partial u(x,y)}{\partial x}|_{y=b} = -p(x) \Leftrightarrow
\end{equation*}
\begin{align*}
    &-\frac{(2G + \lambda)}{a(1+\mu_0)} \int_{0}^{b}\frac{\partial g(y, \xi)}{\partial y}|_{y=b} f(\xi) d\xi - \psi_0^{'}(b) \frac{p_0}{a} - \\
    &- \frac{2(2G + \lambda)}{a} \frac{\partial}{\partial y} \sum_{n=1}^{\infty} \left( \int_0^b \left[g_4(y, \xi) cos(\alpha_n a) f(\xi) \right]d\xi + \psi_0^{4}(y) p_n \right) \\ 
    &cos(\alpha_n x)|_{y=b} - \frac{2\lambda}{a} \frac{\partial}{\partial x} \sum_{n=1}^{\infty} \left( \int_0^b \left[g_2(y, \xi)cos(\alpha_n a) f(\xi) \right]d\xi + \psi_0^2(y) p_n \right) \\ 
    &sin(\alpha_n x)|_{y=b} = -p(x)
\end{align*}
Введемо позначення:
\begin{align}
    &a_1(x) = a p(x) - 2(2G + \lambda) \frac{\partial}{\partial y} \sum_{n=1}^{\infty} \psi_0^{4}(y) p_n cos(\alpha_n x)|_{y=b} - \nonumber \\
    &- 2\lambda \frac{\partial}{\partial x} \sum_{n=1}^{\infty}\psi_0^2(y) p_n sin(\alpha_n x)|_{y=b} - \psi_0^{'}(b) p_0
\end{align}
Враховуючи його отримаємо наступне інтегральне рівняння відносно $f(\xi)$:
\begin{align}\label{int_gen}
    &\frac{(2G + \lambda)}{(1+\mu_0)} \int_{0}^{b}\frac{\partial g(y, \xi)}{\partial y}|_{y=b} f(\xi) d\xi + \nonumber \\ 
    &+ \int_{0}^{b} \sum_{n=1}^{\infty} cos(\alpha_n a) cos(\alpha_n x) \left[(2G + \lambda) \frac{\partial g_4(y, \xi)}{\partial y} + \alpha_n \lambda g_2(y, \xi) \right]|_{y=b} \nonumber \\
    &f(\xi) d\xi = a_1(x)
\end{align}
Розглянемо ряд:
\begin{align*}
    &\sum_{n=1}^{\infty} cos(\alpha_n a) cos(\alpha_n x) \left[(2G + \lambda) \frac{\partial g_4(y, \xi)}{\partial y} + \alpha_n \lambda g_2(y, \xi) \right]|_{y=b} = \\
    &= \sum_{n=1}^{\infty} (-1)^n \alpha_n^{-1} e^{\alpha_n (\xi - b)} cos(\alpha_n x) \left[ \frac{\partial \widetilde{g_4(y, \xi)}}{\partial y} + \lambda \widetilde{g_2(y, \xi)} \right]|_{y=b} = \\
    &= \sum_{n=1}^{N} (-1)^n \alpha_n^{-1} e^{\alpha_n (\xi - b)} cos(\alpha_n x) \left[ \frac{\partial \widetilde{g_4(y, \xi)}}{\partial y} + \lambda \widetilde{g_2(y, \xi)} \right]|_{y=b} + \\
    & + a_2 \sum_{n=N}^{\infty} (-1)^n (2n + 1)^{-1} e^{-(2n + 1) \frac{\pi}{2a} (b - \xi)} cos((2n + 1) \frac{\pi}{2a} x) + \\
    & + a_2 \sum_{n=0}^{N} (-1)^n (2n + 1)^{-1} e^{-(2n + 1) \frac{\pi}{2a} (b - \xi)} cos((2n + 1) \frac{\pi}{2a} x) - \\
    & - a_2 \sum_{n=0}^{N} (-1)^n (2n + 1)^{-1} e^{-(2n + 1) \frac{\pi}{2a} (b - \xi)} cos((2n + 1) \frac{\pi}{2a} x) = \\
    &= a_2 \sum_{n=0}^{\infty} (-1)^n (2n + 1)^{-1} e^{-(2n + 1) \frac{\pi}{2a} (b - \xi)} cos((2n + 1) \frac{\pi}{2a} x) + a_3(\xi, x)
\end{align*}

де:
\begin{equation*}
    a_2 = \frac{2}{\pi} \lim_{n \rightarrow \infty}\left[ \frac{\partial \widetilde{g_4(y, \xi)}}{\partial y} + \lambda \widetilde{g_2(y, \xi)} \right]|_{y=b}, 
\end{equation*}
\begin{align*}
    &a_3(\xi, x) = \sum_{n=1}^{N} cos(\alpha_n a) cos(\alpha_n x) \left[(2G + \lambda) \frac{\partial g_4(y, \xi)}{\partial y} + \alpha_n \lambda g_2(y, \xi) \right]|_{y=b} - \\
    & - a_2 \sum_{n=0}^{N} (-1)^n (2n + 1)^{-1} e^{-(2n + 1) \frac{\pi}{2a} (b - \xi)} cos((2n + 1) \frac{\pi}{2a} x)
\end{align*}

Використовуючи формулу 5.4.12.8 \cite{prudnikov} отримаємо:
\begin{align*}
    &a_2 \sum_{n=0}^{\infty} (-1)^n (2n + 1)^{-1} e^{-(2n + 1) \frac{\pi}{2a} (b - \xi)} cos((2n + 1) \frac{\pi}{2a} x) + a_3(\xi, x) = \\
    &= \frac{a_2}{4} ln\left[ \frac{ch(\frac{\pi}{2a}(b - \xi)) + cos(\frac{\pi}{2a}x)}{ch(\frac{\pi}{2a}(b - \xi)) - cos(\frac{\pi}{2a}x)} \right] + a_3(\xi, x)
\end{align*}

Повернемося до інтегралу
\begin{align}
    &\frac{(2G + \lambda)}{(1+\mu_0)} \int_{0}^{b}\frac{\partial g(y, \xi)}{\partial y}|_{y=b} f(\xi) d\xi + \nonumber \\ 
    &+ \int_{0}^{b} \left( \frac{a_2}{4} ln\left[ \frac{ch(\frac{\pi}{2a}(b - \xi)) + cos(\frac{\pi}{2a}x)}{ch(\frac{\pi}{2a}(b - \xi)) - cos(\frac{\pi}{2a}x)} \right] + a_3(\xi, x) \right) f(\xi) d\xi = \nonumber \\
    &= \int_{0}^{b} ( \frac{a_2}{4} ln\left[ \frac{ch(\frac{\pi}{2a}(b - \xi)) + cos(\frac{\pi}{2a}x)}{ch(\frac{\pi}{2a}(b - \xi)) - cos(\frac{\pi}{2a}x)} \right] + a_3(\xi, x) + \nonumber \\
    &+ \frac{(2G + \lambda)}{(1+\mu_0)} \frac{\partial g(y, \xi)}{\partial y}|_{y=b} ) f(\xi) d\xi = \nonumber \\
    &= \left[
        \begin{matrix}
            t = \frac{ch(\frac{\pi}{2a}(b - \xi)) - 1}{1 - ch(\frac{\pi b}{2a})} \\
            sh(\frac{\pi}{2a}(b - \xi))d\xi = -\frac{2a}{\pi} (ch(\frac{\pi b}{2a}) - 1) dt \\
            \xi = 0, \quad t = 1 \\
            \xi = b, \quad t = 0 \\
            \xi = b - \frac{2a}{\pi} arch((ch(\frac{\pi b}{2a}) - 1)t + 1)
        \end{matrix}
        \right] = \nonumber \\
    &=a_5 \int_{0}^{b} a_4(t) \left( \frac{a_2}{4} ln\left[ \frac{t + cos(\frac{\pi}{2a}x)}{t - cos(\frac{\pi}{2a}x)} \right] + \widetilde{a_3(t, x)} \right) \widetilde{f(t)} dt
\end{align}
де:
\begin{align*}
    &\widetilde{a_3(t, x)} = a_3\left(b - \frac{2a}{\pi} arch((ch(\frac{\pi b}{2a}) - 1)t + 1), x \right) + \\
    &+ \frac{(2G + \lambda)}{(1+\mu_0)} \frac{\partial g(y, b - \frac{2a}{\pi} arch((ch(\frac{\pi b}{2a}) - 1)t + 1))}{\partial y}|_{y=b} \\
    &f(t) = f(b - \frac{2a}{\pi} arch((ch(\frac{\pi b}{2a}) - 1)t + 1)) \\
    &a_4(t) = \frac{1}{ sh\left(arch\left[ (ch(\frac{\pi b}{2a}) - 1)t + 1 \right]\right) } \\
    &a_5 = 2a (ch(\frac{\pi b}{2a}) - 1)
\end{align*}

Таким чином отримаємо наступне інтегральне рівняння:
\begin{equation}
    \frac{a_5}{\pi} \int_{0}^{b} a_4(t) \left( \frac{a_2}{4} ln\left[ \frac{t + cos(\frac{\pi}{2a}x)}{t - cos(\frac{\pi}{2a}x)} \right] + \widetilde{a_3(t, x)} \right) \widetilde{f(t)} dt = a_1(x)
\end{equation}

Розв'язок якого будемо шукати у наступному вигляді:
\begin{equation}
    \widetilde{f(t)} = \frac{1}{a_2 a_4(t)} \frac{1}{\sqrt{1 - t^2}} \sum_{k=0}^{\infty} \varphi_k T_{2k + 1}(t) 
\end{equation}
Де $\varphi_k$ - невідомі коєфіцієнти, $T_{2k + 1}(t)$ - поліном Чебишева першого роду.

Таким чином отримаємо
\begin{align*}
    & \sum_{k=0}^{\infty}  \frac{\varphi_k}{4} \frac{1}{\pi} \int_{0}^{1} ln\left[ \frac{t + cos(\frac{\pi}{2a}x)}{t - cos(\frac{\pi}{2a}x)} \right] \frac{T_{2k + 1}(t)}{\sqrt{1 - t^2}} dt + \\
    & + \sum_{k=0}^{\infty} \varphi_k \frac{1}{\pi} \int_{0}^{1} \frac{\widetilde{a_3(t, x)}}{a_2} \frac{T_{2k + 1}(t)}{\sqrt{1 - t^2}} dt = \frac{a_1(x)}{a_5} \Leftrightarrow
\end{align*}
Використовуючи формулу B.1.9 \cite{ortogonal}
\begin{equation}\label{int_2_gen}
    \sum_{k=0}^{\infty}  \varphi_k \frac{T_{2k + 1}( cos(\frac{\pi}{2a}x) )}{4(2k + 1)} + \sum_{k=0}^{\infty} \varphi_k \frac{1}{\pi} \int_{0}^{1} \frac{\widetilde{a_3(t, x)}}{a_2} \frac{T_{2k + 1}(t)}{\sqrt{1 - t^2}} dt = \frac{a_1(x)}{a_5}
\end{equation}

Введемо позначення
\begin{equation*}
    l = cos(\frac{\pi}{2a}x), \quad \widetilde{a_1(l)} = \frac{a_1(\frac{2a}{\pi} arccos(l))}{a_5}
\end{equation*}
Помножимо обидві частини рівняння \eqref{int_2_gen} скалярно на $\frac{T_{2m + 1}(l)}{\sqrt{1 - l^2}}$ та проінтегруєм по змінній $l$ на інтервалі $(-1; 1)$.
Та використовуючи формулу 2.3.2 \cite{ortogonal} отримаєм наступне бескінечну алгебричну систему відносно невідомих коєфіцієнтів $\varphi_k$, яка в подальшому буде розв'язуватись методом редукції.
\begin{equation}\label{int_system_gen}
    \frac{\phi_m \pi}{8(2m + 1)} + \sum_{k=0}^{\infty} \phi_k g_{k, m} = f_m
\end{equation}
Де $g_{k, m} = \frac{1}{\pi} \int_{-1}^{1} \frac{T_{2m + 1}(l)}{\sqrt{1 - l^2}} \int_{0}^{1} \frac{\widetilde{a_3(t, \frac{2a}{\pi} arccos(l) )}}{a_2} \frac{T_{2k + 1}(t)}{\sqrt{1 - t^2}} dt dl$,
$f_m = \int_{-1}^{1} \frac{T_{2m + 1}(l) \widetilde{a_1(l)}}{\sqrt{1 - l^2}} dl$ інтеграли відомих функцій


\subsection{Висновки до другого розділу}
Безпосередньо застосовані інтегральні перетворення до рівнянь рівноваги Ламе та крайових умов плоскої задачі теорії пружності для прямокутної області.
Це дозволило уникнути використання допоміжних гармонічних або бігармонічних функцій.
Зведено вихідну задачу до одновимірної векторної крайової задачі у просторі трансформант.
Цю задачу було розв'язано за допомогою методів диференціального матричного числення.
Для цього була побудована фундаментальна базисна матрична система розв'язків однорідного матричного рівняння та матриця-функція Гріна для диференціального векторного рівняння другого порядку.
Побудовано та розв'язано сінгульрне інтегральне рівняння відносно невідомої функції шляхом викорстання методу ортагональних поліномів, та зведення рівнняння до бескінечної алгебричної системи,
яка в подальшому була розв'язана методом редукціі.

\newpage

\section{СТАТИЧНА ЗАДАЧА ТЕОРІЇ ПРУЖНОСТІ ДЛЯ ПРЯМОКУТНОЇ ОБЛАСТІ
ЗА УМОВ ІДЕАЛЬНОГО КОНТАКТУ НА БІЧНИХ ГРАНЯХ}
\subsubsection{Постановка задачі}
\begin{figure}[h]
    \begin{center}
        \includegraphics[scale=1]{images/geometry/image_3.jpg}
    \end{center}
    \caption{Геометрія проблеми}\label{geom_static_1}
\end{figure}
Розглядається пружна прямокутна область (Рис: \ref{geom_static_1}), яка займає облась,
що описується у декартовій системі координат співвідношенням $0 \le x \le a$, $0 \le y \le b$.

До прямокутної області на грані $y=b$ додане нормальне навантаження
\begin{equation}
    \sigma_y(x, y) |_{y=b} = -p(x), \quad  \tau_{xy}(x,y) |_{y=b} =0
\end{equation}
де $p(x)$ відома функція.

На бічних гранях виконується умова ідеального контакту
\begin{equation}\label{bound_1_static_1}
    u(x,y) |_{x=0} = 0, \quad \tau_{xy}(x,y) |_{x=0} =0
\end{equation}
\begin{equation}\label{bound_2_static_1}
    u(x,y) |_{x=a} = 0, \quad \tau_{xy}(x,y) |_{x=a} =0
\end{equation}
На нижній грані виконуються наступні умови
\begin{equation}
    v(x,y) |_{y=0} = 0, \quad \tau_{xy}(x,y) |_{y=0} =0
\end{equation}
Розглядаються наступні рівняння рівноваги Ламе:
\begin{equation}\label{lame_static_1}
    \begin{cases}
        \frac{\partial^2 u(x,y)}{\partial x^2} + \frac{\partial^2 u(x,y)}{\partial y^2} + \mu_0 (\frac{\partial^2 u(x,y)}{\partial x^2} + \frac{\partial^2 v(x,y)}{\partial x\partial y}) = 0 \\
        \frac{\partial^2 v(x,y)}{\partial x^2} + \frac{\partial^2 v(x,y)}{\partial y^2} + \mu_0 (\frac{\partial^2 u(x,y)}{\partial x \partial y} + \frac{\partial^2 v(x,y)}{\partial y^2}) = 0 \\
    \end{cases}
\end{equation}

\subsubsection{Побудова точного розв'язку вихідної задачі}
Для того, щоб звести задачу до одновимірної задачі, використаємо інтегральне перетворення Фур'є по змінній $x$ до рівнянь (\ref{lame_static_1}) в наступному вигляді:
\begin{equation}
    \begin{pmatrix}
        u_n(y) \\
        v_n(y)
    \end{pmatrix} = \int_{0}^{a} 
    \begin{pmatrix}
        u(x,y) sin(\alpha_n x) \\
        v(x,y) cos(\alpha_n x)
    \end{pmatrix} dx, \quad \alpha_n = \frac{\pi n}{a}
\end{equation}

Для цього помножимо перше та друге рівняння (\ref{lame_static_1}) на $sin(\alpha_n x)$ та $cos(\alpha_n x)$ відповідно та проінтегруємо по змінній $x$ на інтервалі $0 \le x \le a$.

Розглянемо перше рівняня
\begin{align*}
    &\int_{0}^{a} \frac{\partial^2 u(x,y)}{\partial x^2} sin(\alpha_n x)dx + \int_{0}^{a} \frac{\partial^2 u(x,y)}{\partial y^2} sin(\alpha_n x)dx + \\ 
    & + \mu_0 \left( \int_{0}^{a} \frac{\partial^2 u(x,y)}{\partial x^2} sin(\alpha_n x)dx + \int_{0}^{a} \frac{\partial^2 v(x,y)}{\partial x \partial y} sin(\alpha_n x) dx\right) = 0
\end{align*}

Розглянемо
\begin{align*}
    &\int_{0}^{a} \frac{\partial^2 u(x,y)}{\partial x^2} sin(\alpha_n x)dx = \frac{\partial u(x,y)}{\partial x} sin(\alpha_n x) |_{x=0}^{x=a} - \alpha_n \int_{0}^{a} \frac{\partial u(x,y)}{\partial x} cos(\alpha_n x)dx = \\
    &= \frac{\partial u(x,y)}{\partial x} sin(\alpha_n x) |_{x=0}^{x=a} - \alpha_n \left( u(x,y) cos(\alpha_n x) |_{x=0}^{x=a} + \alpha_n \int_{0}^{a} u(x,y) sin(\alpha_n x) dx \right) = \\
    &= -\alpha_n^2 u_n(y)
\end{align*}

Розглянемо
\begin{align*}
    &\int_{0}^{a} \frac{\partial^2 u(x,y)}{\partial y^2} sin(\alpha_n x)dx = \frac{\partial^2}{\partial y^2} \int_{0}^{a} u(x,y) sin(\alpha_n x)dx = u_n^{''}(y)
\end{align*}

Розглянемо
\begin{align*}
    &\int_{0}^{a} \frac{\partial^2 v(x,y)}{\partial x \partial y} sin(\alpha_n x) dx = \frac{\partial v(x,y)}{\partial y} sin(\alpha_n x) |_{x=0}^{x=a} - \alpha_n \int_{0}^{a} \frac{\partial v(x,y)}{\partial y} cos(\alpha_n x) dx = \\
    &= -\alpha_n \frac{\partial}{\partial y} \int_{0}^{a} v(x,y) cos(\alpha_n x) dx = -\alpha_n v_n^{'}(y)
\end{align*}

Тоді перше рівняння у просторі трансформант прийме вигляд:
\begin{align*}
    &u_n^{''}(y) - \alpha_n \mu_0 v_n^{'}(y) -(\alpha_n^2 + \alpha_n^2 \mu_0) u_n(y) = 0
\end{align*}

Розлянемо друге рівняння
\begin{align*}
    &\int_{0}^{a} \frac{\partial^2 v(x,y)}{\partial x^2} cos(\alpha_n x)dx + \int_{0}^{a} \frac{\partial^2 v(x,y)}{\partial y^2} cos(\alpha_n x)dx + \\ 
    & + \mu_0 \left( \int_{0}^{a} \frac{\partial^2 u(x,y)}{\partial x \partial y} cos(\alpha_n x)dx +  \int_{0}^{a} \frac{\partial^2 v(x,y)}{\partial y^2} cos(\alpha_n x) dx\right) = 0
\end{align*}

Розглянемо
\begin{align*}
    &\int_{0}^{a} \frac{\partial^2 v(x,y)}{\partial x^2} cos(\alpha_n x)dx = \frac{\partial v(x,y)}{\partial x} cos(\alpha_n x) |_{x=0}^{x=a} + \alpha_n \int_{0}^{a} \frac{\partial v(x,y)}{\partial x} sin(\alpha_n x) dx = \\
    &=\frac{\partial v(x,y)}{\partial x} cos(\alpha_n x) |_{x=0}^{x=a} + \alpha_n \left(v(x,y) sin(\alpha_n x)|_{x=0}^{x=a} - \alpha_n \int_{0}^{a} v(x,y) cos(\alpha_n x) dx  \right) = \\
    &= -\alpha_n^2 v_n(y)
\end{align*}

Розглянемо
\begin{align*}
    &\int_{0}^{a} \frac{\partial^2 v(x,y)}{\partial y^2} cos(\alpha_n x)dx = \frac{\partial^2}{\partial y^2} \int_{0}^{a} v(x,y) cos(\alpha_n x)dx = v_n^{''}(y)
\end{align*}

Розглянемо
\begin{align*}
    &\int_{0}^{a} \frac{\partial^2 u(x,y)}{\partial y \partial x} cos(\alpha_n x)dx = \frac{\partial u(x,y)}{\partial y} cos(\alpha_n x) |_{x=0}^{x=a} + \alpha_n \int_{0}^{a} \frac{\partial u(x,y)}{\partial y} sin(\alpha_n x) dx = \\
    &=\frac{\partial u(x,y)}{\partial y} cos(\alpha_n x) |_{x=0}^{x=a} + \alpha_n \frac{\partial}{\partial y} \int_{0}^{a} u(x,y) sin(\alpha_n x) dx = \alpha_n u_n^{'}(y)
\end{align*}

Тоді друге рівняння у просторі трансформант прийме вигляд:
\begin{align*}
    &(1 + \mu_0) v_n^{''}(y) + \alpha_n \mu_0 u_n^{'}(y)  - \alpha_n^2 v_n(y) = 0
\end{align*}

Отримана система рівнянь задачі у просторі трансформант:
\begin{equation}\label{transf_static_1}
    \begin{cases}
        u_n^{''}(y) - \alpha_n \mu_0 v_n^{'}(y) - \alpha_n^2 (1 + \mu_0) u_n(y) = 0 \\
        (1 + \mu_0) v_n^{''}(y) + \alpha_n \mu_0 u_n^{'}(y)  - \alpha_n^2 v_n(y) = 0 \\
    \end{cases}
\end{equation}

Застосовуючи інтегральне перетворення до граничних умов,
отримаємо наступні умови задачі у просторі трансформант
\begin{equation}\label{transf_bound_static_1}
    \begin{cases}
        \left( (2G + \lambda)v_n^{'}(y) + \alpha_n \lambda u_n(y) \right)|_{y=b} = -p_n \\
        \left(u_n^{'}(y) - \alpha_n v_n(y)  \right)|_{y=b} = 0 \\
        v_n(y)|_{y=0} = 0 \\
        \left(u_n^{'}(y) - \alpha_n v_n(y)  \right)|_{y=0} = 0
    \end{cases}
\end{equation}
де $p_n = \int_{0}^{a} p(x) cos(\alpha_n x) dx$

Для того щоб розв'язати задачу у простосторі трансформант, перепишемо її у матрично-векторній формі.
Рівняння рівноваги (\ref{transf_static_1}) запишемо у наступному вигляді:
\begin{align}\label{transf_mat_static_1}
    &L_2\left[ Z_n(y) \right] = A * Z_n^{''}(y) + B * Z_n^{'}(y) + C * Z_n(y) \nonumber \\
    & L_2\left[ Z_n(y) \right] = 0
\end{align}
Де
\begin{equation*}
    A = \begin{pmatrix}
        1 & 0 \\
        0 & 1 + \mu_0
    \end{pmatrix}, \quad
    B = \begin{pmatrix}
        0 & -\alpha_n \mu_0 \\
        \alpha_n \mu_0 & 0
    \end{pmatrix}, \quad
    C = \begin{pmatrix}
        -\alpha_n^2(1 + \mu_0) & 0 \\
        0 & -\alpha_n^2
    \end{pmatrix}
\end{equation*}
\begin{equation*}
    Z_n(y) = \begin{pmatrix}
        u_n(y) \\
        v_n(y)
    \end{pmatrix}
\end{equation*}
Граничні умови (\ref{transf_bound_static_1}) запишемо у наступному вигляді:
\begin{align}\label{transf_bound_mat_static_1}
    &U_i\left[ Z_n(y) \right] = E_i * Z_n^{'}(b_i) + F_i * Z_n(b_i) \nonumber \\
    & U_i\left[ Z_n(y) \right] = D_i
\end{align}
де $i = \overline{0, 1}$, $b_0 = b$, $b_1 = 0$,
\begin{equation*}
    E_0 = \begin{pmatrix}
        1 & 0 \\
        0 & 2G + \lambda
    \end{pmatrix}, \quad
    F_0 = \begin{pmatrix}
        0 & -\alpha_n \\
        \alpha_n \lambda & 0
    \end{pmatrix}, \quad
\end{equation*}
\begin{equation*}
    E_1 = \begin{pmatrix}
        1 & 0 \\
        0 & 0
    \end{pmatrix}, \quad
    F_1 = \begin{pmatrix}
        0 & -\alpha_n \\
        0 & 1
    \end{pmatrix}, \quad
\end{equation*}
\begin{equation*}
    D_0 = \begin{pmatrix}
        0 \\
        -p_n
    \end{pmatrix}, \quad
    D_1 = \begin{pmatrix}
        0 \\
        0
    \end{pmatrix}, \quad
\end{equation*}

Для знаходження розв'язку задачі у просторі трансформант, знайдемо фундаментальну матрицю рівняння (\ref{transf_mat_static_1}).
Шукати її будем у наступному вигляді \cite{gantmaher}:
\begin{equation}
    Y(y) = \frac{1}{2\pi i} \oint_C e^{sy} M^{-1}(s)ds
\end{equation}
Де $M(s)$ - характерестична матриця рівняння (\ref{transf_mat_static_1}), а $C$ - замкнений контур який містить усі особливі точки. Яку будемо шукати з наступної умовни
\begin{equation}
    L_2\left[ e^{sy}*I \right] = e^{sy} * M(s), \quad I = \begin{pmatrix} 1 & 0 \\ 0 & 1 \end{pmatrix}
\end{equation}
\begin{align*}
    &L_2\left[ e^{sy}*I \right] = e^{sy} \left( s^2A * I + s B*I + C*I \right) = \\
    &=e^{sy} \left( \begin{pmatrix}
        s^2 & 0 \\
        0 & s^2 (1 + \mu_0)
    \end{pmatrix} + \begin{pmatrix}
        0 & -\alpha_n \mu_0 s\\
        \alpha_n \mu_0 s & 0
    \end{pmatrix} + \begin{pmatrix}
        -\alpha_n^2(1 + \mu_0) & 0 \\
        0 & -\alpha_n^2
    \end{pmatrix} \right) =  \\
    &=e^{sy} \begin{pmatrix}
        s^2 -\alpha_n^2(1 + \mu_0) & -\alpha_n \mu_0 s \\
        \alpha_n \mu_0 s & s^2 (1 + \mu_0) -\alpha_n^2
     \end{pmatrix} \Rightarrow
\end{align*}

\begin{equation}
    M(s) = \begin{pmatrix}
        s^2 -\alpha_n^2(1 + \mu_0) & -\alpha_n \mu_0 s \\
        \alpha_n \mu_0 s & s^2 (1 + \mu_0) -\alpha_n^2
     \end{pmatrix}
\end{equation}

Знайдемо тепер $M^{-1}(s) = \frac{\widetilde{M(s)}}{det[M(s)]}$.
\begin{equation}
    \widetilde{M(s)} = \begin{pmatrix}
        s^2 (1 + \mu_0) -\alpha_n^2 & \alpha_n \mu_0 s \\
        -\alpha_n \mu_0 s & s^2 -\alpha_n^2(1 + \mu_0)
     \end{pmatrix}
\end{equation}
\begin{align}
    &det[M(s)] = \begin{vmatrix}
        s^2 - \alpha_n^2 - \alpha_n^2\mu_0 & -\alpha_n \mu_0 s \\
        \alpha_n \mu_0 s & s^2 (1 + \mu_0) -\alpha_n^2
     \end{vmatrix} = \nonumber \\
    &=(1+\mu_0)(s - \alpha_n)^2(s + \alpha_n)^2
\end{align}
Де $\alpha_n$, $-\alpha_n$, корені $det[M(s)]=0$, детальне знаходження яких наведено в (\nameref{ap_B}).

Враховучи це, тепер знайдемо значення фундаментальної матрицю за допомогою теореми про лишки:
\begin{align*}
    &\frac{1}{2\pi i} \oint_C e^{sy} M^{-1}(s)ds = \frac{2 \pi i}{2 \pi i (1 + \mu_0)} \sum_{i=1}^{2} Res\left[ e^{sy} \frac{\widetilde{M(s)}}{det[M(s)]} \right] = \\
    & = \frac{1}{(1 + \mu_0)} \left(Y_0(y) + Y_1(y) \right)
\end{align*}
Знайдемо $Y_0(y)$:
\begin{align}
    &Y_0(y) =  \frac{\partial}{\partial s} \left( \frac{e^{sy}}{(s+\alpha_n)^2} \widetilde{M(s)} \right) \Big|_{s=\alpha_n} = \nonumber \\
    &=\frac{e^{\alpha_n y}}{4\alpha_n} \begin{pmatrix}
    \alpha_n \mu_0 y + 2 + \mu_0 & \alpha_n \mu_0 y \\
    -\alpha_n \mu_0 y & -\alpha_n \mu_0 y + 2 + \mu_0
    \end{pmatrix}
\end{align}
Знайдемо $Y_1(y)$:
\begin{align}
    &Y_1(y) = \frac{\partial}{\partial s} \left(\frac{e^{sy}}{(s-\alpha_n)^2} \widetilde{M(s)} \right) \Big|_{s=-\alpha_n} = \nonumber \\
    =&\frac{e^{-\alpha_n y}}{4\alpha_n} \begin{pmatrix}
    \alpha_n \mu_0 y - 2 - \mu_0 & -\alpha_n \mu_0 y \\
    \alpha_n \mu_0 y & -\alpha_n \mu_0 y - 2 - \mu_0
    \end{pmatrix}
\end{align}

Таким чином можна записати розв'язок задачі у просторі трансформант:
\begin{equation}
    Z_n(y) = \frac{1}{1 + \mu_0} \left( Y_0(y) * \begin{pmatrix} c_1 \\ c_2 \end{pmatrix} +  Y_1(y) * \begin{pmatrix} c_3 \\ c_4 \end{pmatrix}  \right)
\end{equation}
Залишилось знайти невідомі коєфіцієнти $c_1$, $c_2$, $c_3$, $c_4$, використовуючи граничні умови (\ref{transf_bound_mat_static_1}).
Покрокове знаходження коєфіцієнтів наведено у (\nameref{ap_E}).
Таким чином можна записати розв'зок у просторі трансформант:
\begin{align}\label{transf_sol_u_static_1}
    &u_n(y) = \frac{e^{\alpha_n y}}{4 \alpha_n (1 + \mu_0)} \left[c_1 (\alpha_n \mu_0 y + 2 + \mu_0) + c_2 (\alpha_n \mu_0 y) \right] + \nonumber \\
    &\quad + \frac{e^{-\alpha_n y}}{4 \alpha_n (1 + \mu_0)} \left[c_3 (\alpha_n \mu_0 y - 2 - \mu_0) + c_4 (-\alpha_n \mu_0 y)\right]
\end{align}
\begin{align}\label{transf_sol_v_static_1}
    &v_n(y) = \frac{e^{\alpha_n y}}{4 \alpha_n (1 + \mu_0)} \left[c_1 (-\alpha_n \mu_0 y) + c_2 (-\alpha_n \mu_0 y + 2 + \mu_0) \right] + \nonumber \\
    &\quad + \frac{e^{-\alpha_n y}}{4 \alpha_n (1 + \mu_0)} \left[c_3 (\alpha_n \mu_0 y) + c_4 (-\alpha_n \mu_0 y - 2 - \mu_0)\right]
\end{align}

Викорустовуючи обернене інтегральне перетворення Фур'є до розв'язку задачі у просторі трансформант
(\ref{transf_sol_u_static_1}), (\ref{transf_sol_v_static_1}), отримаємо фінальний розв'язок задачі
\begin{equation}
    u(x,y) = \frac{2}{a} \sum_{n=1}^{\infty} u_n(y) sin(\alpha_n x), \quad \alpha_n = \frac{\pi n}{a}
\end{equation}
\begin{equation}
    v(x,y) = \frac{v_0(y)}{a} + \frac{2}{a} \sum_{n=1}^{\infty} v_n(y) cos(\alpha_n x), \quad \alpha_n = \frac{\pi n}{a}
\end{equation}

Останній крок це знаходження $v_0(y)$ у випадку коли $n=0$, $\alpha_n =0$.
Для цього повернемся до другого рівняння (\ref{transf_static_1}), та запишем його для цього випадку:
\begin{equation}\label{transf_v_0_static_1}
    (1 + \mu_0) v_n^{''}(y) = 0
\end{equation}
Та граничні умови:
\begin{equation}\label{transf_bound_v_0_static_1}
    \begin{cases}
        (2G + \lambda)v_0^{'}(y)|_{y=b} = -p_0 \\
        v_0(y)|_{y=0} = 0
    \end{cases}
\end{equation}
Де $p_0 = \int_{0}^{a}p(x)dx$

Розв'язок рівняння (\ref{transf_v_0_static_1}):
\begin{equation}
    v_0(y) = c_1 + c_2 y
\end{equation}
Застовоючи граничні умови (\ref{transf_bound_v_0_static_1}) для знаходження коєфіцієнтів $c_1$, $c_2$, отримаємо розв'язок задачі задачі:
\begin{equation}
    v_0(y) = \frac{-p_0}{(2G + \lambda)}y
\end{equation}
Тепер остаточний розв'зок задачі можна записати у вигляді:
\begin{equation}
    \begin{cases}
        u(x,y) = \frac{2}{a} \sum_{n=1}^{\infty} u_n(y) sin(\alpha_n x), \quad \alpha_n = \frac{\pi n}{a} \\
        v(x,y) = \frac{-p_0}{(2G + \lambda)a}y + \frac{2}{a} \sum_{n=1}^{\infty} v_n(y) cos(\alpha_n x), \quad \alpha_n = \frac{\pi n}{a}
    \end{cases}
\end{equation}

\subsubsection{Чисельні розрахунки}
Наведені чисельні експеренти розглядаються для сталі ($E=200$ ГПА, $\mu=0.25$).

Розглянута прямокунта область $0 \le x \le 10$, $0 \le y \le 15$, при функції навантаження $p(x)=(x-2.5)^2$.
На малюнках (Рис: \ref{static_1_u_1}), (Рис: \ref{static_1_v_1}), (Рис: \ref{static_1_sigma_x_1}), (Рис: \ref{static_1_sigma_y_1})
представлені функіі переміщень $u(x,y)$, $v(x,y)$ та напружень $\sigma_x(x,y)$, $\sigma_y(x,y)$ відповідно.
\begin{figure}[h!]
    \begin{center}
        \includegraphics[width=0.49\textwidth, scale=1]{images/results/static_1/function_u_1.png}
        \includegraphics[width=0.49\textwidth, scale=1]{images/results/static_1/function_u_2.png}
        \caption{Функція $u(x, y)$}\label{static_1_u_1}
    \end{center}
\end{figure}
\newpage
\begin{figure}[h!]
    \begin{center}
        \includegraphics[width=0.49\textwidth, scale=1]{images/results/static_1/function_v_1.png}
        \includegraphics[width=0.49\textwidth, scale=1]{images/results/static_1/function_v_2.png}
        \caption{Функція $v(x, y)$}\label{static_1_v_1}
    \end{center}
\end{figure}
\begin{figure}[h!]
    \begin{center}
        \includegraphics[width=0.49\textwidth, scale=1]{images/results/static_1/function_sigma_x_1.png}
        \includegraphics[width=0.49\textwidth, scale=1]{images/results/static_1/function_sigma_x_2.png}
        \caption{Функція $\sigma_x(x, y)$}\label{static_1_sigma_x_1}
    \end{center}
\end{figure}
\begin{figure}[h!]
    \begin{center}
        \includegraphics[width=0.49\textwidth, scale=1]{images/results/static_1/function_sigma_y_1.png}
        \includegraphics[width=0.49\textwidth, scale=1]{images/results/static_1/function_sigma_y_2.png}
        \caption{Функція $\sigma_y(x, y)$}\label{static_1_sigma_y_1}
    \end{center}
\end{figure}

\newpage

\section{ДИНАМІЧНА ЗАДАЧА ТЕОРІЇ ПРУЖНОСТІ ДЛЯ ПРЯМОКУТНОЇ ОБЛАСТІ
ЗА УМОВ ІДЕАЛЬНОГО КОНТАКТУ НА БІЧНИХ ГРАНЯХ}
\subsubsection{Постановка задачі}
Розглядається пружне прямокутне тіло (Рис: \ref{geom_gen}), яке займає облась,
що описується у декартовій системі координат співвідношеннями $0 \le x \le a$, $0 \le y \le b$.

До грані $y=b$ додане нормальне навантаження
\begin{equation}\label{bound_1_dynamic_1}
    \sigma_y(x, y, t) |_{y=b} = -p(x, t), \quad  \tau_{xy}(x,y,t) |_{y=b} =0
\end{equation}
де $p(x,t)$ відома функція.
На бічних та нижній гранях виконуються умови ідеального контакту
\begin{equation}\label{bound_2_dynamic_1}
    u(x,y,t) |_{x=0} = 0, \quad \tau_{xy}(x,y,t) |_{x=0} =0
\end{equation}
\begin{equation}\label{bound_3_dynamic_1}
    u(x,y,t) |_{x=a} = 0, \quad \tau_{xy}(x,y,t) |_{x=a} =0
\end{equation}
\begin{equation}\label{bound_4_dynamic_1}
    v(x,y,t) |_{y=0} = 0, \quad \tau_{xy}(x,y,t) |_{y=0} =0
\end{equation}
Потрібно відшукати розв'язок рівняннь Ламе
\begin{equation}
    \begin{cases}
        \frac{\partial^2 u(x,y,t)}{\partial x^2} + \frac{\partial^2 u(x,y,t)}{\partial y^2} + \mu_0 (\frac{\partial^2 u(x,y,t)}{\partial x^2} + \frac{\partial^2 v(x,y,t)}{\partial x\partial y}) = \frac{1}{c_1^2} \frac{\partial^2 u(x,y,t)}{\partial t^2} \\
        \frac{\partial^2 v(x,y,t)}{\partial x^2} + \frac{\partial^2 v(x,y,t)}{\partial y^2} + \mu_0 (\frac{\partial^2 u(x,y,t)}{\partial x \partial y} + \frac{\partial^2 v(x,y,t)}{\partial y^2}) = \frac{1}{c_2^2} \frac{\partial^2 v(x,y,t)}{\partial t^2} \\
    \end{cases}
\end{equation}
за умови виконання крайових умов \eqref{bound_1_dynamic_1} - \eqref{bound_4_dynamic_1}

Тут і далі розглянуто постановкe задачі у випадку гармонічних коливань
\begin{equation}\label{garm_dynamic_1}
    u(x,y,t) = u(x,y) e^{i \omega t}, \quad v(x,y,t) = v(x,y) e^{i \omega t}, \quad p(x, y, t) = p(x, y) e^{i \omega t}
\end{equation}
З урахуванням подання переміщеннь \eqref{garm_dynamic_1} рівняння Ламе переформульовано:
\begin{equation}\label{lame_dynamic_1}
    \begin{cases}
        \frac{\partial^2 u(x,y)}{\partial x^2} + \frac{\partial^2 u(x,y)}{\partial y^2} + \mu_0 (\frac{\partial^2 u(x,y)}{\partial x^2} + \frac{\partial^2 v(x,y)}{\partial x\partial y}) = -\frac{\omega^2}{c_1^2}  u(x,y) \\
        \frac{\partial^2 v(x,y)}{\partial x^2} + \frac{\partial^2 v(x,y)}{\partial y^2} + \mu_0 (\frac{\partial^2 u(x,y)}{\partial x \partial y} + \frac{\partial^2 v(x,y)}{\partial y^2}) = -\frac{\omega^2}{c_2^2} v(x,y) \\
    \end{cases}
\end{equation}
Граничні умови набувають вигляду:
\begin{equation}\label{bound_dynamic_1}
    \begin{cases}
        \sigma_y(x, y) |_{y=b} = -p(x, t), \quad  \tau_{xy}(x,y) |_{y=b} = 0 \\
        v(x,y) |_{y=0} = 0, \quad \tau_{xy}(x,y) |_{y=0} = 0 \\
        u(x,y) |_{x=0} = 0, \quad \tau_{xy}(x,y) |_{x=0} = 0 \\
        u(x,y) |_{x=a} = 0, \quad \tau_{xy}(x,y) |_{x=a} = 0 
    \end{cases}
\end{equation}

Треба знайти хвильове поле пружного прямокутника,
що задовольняє крайову задачу \eqref{lame_dynamic_1}, \eqref{bound_dynamic_1}.

\subsubsection{Побудова точного розв'язку вихідної задачі}
Для того, щоби звести задачу до одновимірної задачі у просторі трансформант, використано інтегральне перетворення Фур'є за змінною $x$ до рівнянь (\ref{lame_static_1}):
\begin{equation}\label{int_trans_dynamic_1}
    \begin{pmatrix}
        u_n(y) \\
        v_n(y)
    \end{pmatrix} = \int_{0}^{a} 
    \begin{pmatrix}
        u(x,y) sin(\alpha_n x) \\
        v(x,y) cos(\alpha_n x)
    \end{pmatrix} dx, \quad \alpha_n = \frac{\pi n}{a}
\end{equation}

Після інтегрування за частинами обох рівнянь Ламе оримаємо наступні рівняння у просторі трансформант
\begin{equation}\label{transf_dynamic_1}
    \begin{cases}
        u_n^{''}(y) - \alpha_n \mu_0 v_n^{'}(y) + (-\alpha_n^2 -\alpha_n^2 \mu_0 + \frac{\omega^2}{c_1^2}) u_n(y) = 0 \\
        (1 + \mu_0) v_n^{''}(y) + \alpha_n \mu_0 u_n^{'}(y) + (- \alpha_n^2 + \frac{\omega^2}{c_2^2}) v_n(y) = 0 \\
    \end{cases}
\end{equation}
Застосовуючи інтегральне перетворення \eqref{int_trans_dynamic_1} до крайових умов \eqref{bound_dynamic_1},
отримаємо крайові умови задачі у просторі трансформант
\begin{equation}\label{transf_bound_dynamic_1}
    \begin{cases}
        \left( (2G + \lambda)v_n^{'}(y) + \alpha_n \lambda u_n(y) \right)|_{y=b} = -p_n \\
        \left(u_n^{'}(y) - \alpha_n v_n(y)  \right)|_{y=b} = 0 \\
        v_n(y)|_{y=0} = 0 \\
        \left(u_n^{'}(y) - \alpha_n v_n(y)  \right)|_{y=0} = 0
    \end{cases}
\end{equation}
де $p_n = \int_{0}^{a} p(x) cos(\alpha_n x) dx$

Для того щоб розв'язати задачу у простосторі трансформант, її переписано у векторній формі.
Рівняння рівноваги (\ref{transf_dynamic_1}) запишемо у наступному вигляді:
\begin{equation}\label{transf_mat_dynamic_1}
    L_2\left[ Z_n(y) \right] = 0
\end{equation}
\begin{equation}
    L_2\left[ Z_n(y) \right] = A * Z_n^{''}(y) + B * Z_n^{'}(y) + C * Z_n(y)
\end{equation}
де
\begin{equation*}
    A = \begin{pmatrix}
        1 & 0 \\
        0 & 1 + \mu_0
    \end{pmatrix}, \quad
    B = \begin{pmatrix}
        0 & -\alpha_n \mu_0 \\
        \alpha_n \mu_0 & 0
    \end{pmatrix}
\end{equation*}
\begin{equation*}
    C = \begin{pmatrix}
        -\alpha_n^2 -\alpha_n^2 \mu_0 + \frac{\omega^2}{c_1^2} & 0 \\
        0 & -\alpha_n^2 + \frac{\omega^2}{c_2^2}
    \end{pmatrix}, \quad
    Z_n(y) = \begin{pmatrix}
        u_n(y) \\
        v_n(y)
    \end{pmatrix}
\end{equation*}
Граничні умови (\ref{transf_bound_dynamic_1}) запишемо у наступному вигляді:
\begin{equation}\label{transf_bound_mat_dynamic_1}
    U_i\left[ Z_n(y) \right] = D_i
\end{equation}
\begin{equation}
    U_i\left[ Z_n(y) \right] = E_i * Z_n^{'}(b_i) + F_i * Z_n(b_i)
\end{equation}
де $i = \overline{0, 1}$, $b_0 = b$, $b_1 = 0$,
\begin{equation*}
    E_0 = \begin{pmatrix}
        1 & 0 \\
        0 & 2G + \lambda
    \end{pmatrix}, \quad
    F_0 = \begin{pmatrix}
        0 & -\alpha_n \\
        \alpha_n \lambda & 0
    \end{pmatrix}, \quad
\end{equation*}
\begin{equation*}
    E_1 = \begin{pmatrix}
        1 & 0 \\
        0 & 0
    \end{pmatrix}, \quad
    F_1 = \begin{pmatrix}
        0 & -\alpha_n \\
        0 & 1
    \end{pmatrix}, \quad
\end{equation*}
\begin{equation*}
    D_0 = \begin{pmatrix}
        0 \\
        -p_n
    \end{pmatrix}, \quad
    D_1 = \begin{pmatrix}
        0 \\
        0
    \end{pmatrix}, \quad
\end{equation*}

Будемо шукати однорідний розв'язок рівняння,
для цього спочатку побудуємо фундаментальну матрицю рівняння \eqref{transf_mat_static_1}.
Її будемо шукати у наступному поданні \cite{gantmaher}:
\begin{equation}
    Y(y) = \frac{1}{2\pi i} \oint_C e^{sy} M^{-1}(s)ds
\end{equation}
Де $M(s)$ - характерестична матриця рівняння (\ref{transf_mat_dynamic_1}), а $C$ - замкнений контур який містить усі особливі точки.
Матрицю $M(s)$ будемо шукати з наступної умови
\begin{equation}
    L_2\left[ e^{sy}*I \right] = e^{sy} * M(s), \quad I = \begin{pmatrix} 1 & 0 \\ 0 & 1 \end{pmatrix}
\end{equation}
де матриця $M(s)$ має вигляд:
\begin{equation}
    M(s) = \begin{pmatrix}
        s^2 - \alpha_n^2 - \alpha_n^2\mu_0 + \frac{\omega^2}{c_1^2} & -\alpha_n \mu_0 s \\
        \alpha_n \mu_0 s & s^2 (1 + \mu_0) -\alpha_n^2 + \frac{\omega^2}{c_2^2}
     \end{pmatrix}
\end{equation}

Знайдемо тепер $M^{-1}(s)$, яку побудовано у наступній формі $M^{-1}(s) = \frac{\widetilde{M(s)}}{det[M(s)]}$, де $\widetilde{M(s)}$ - транспонована матриця алгебричних доповнень,
$det[M(s)]$ - детермінант матриці
\begin{equation}
    \widetilde{M(s)} = \begin{pmatrix}
        s^2 (1 + \mu_0) -\alpha_n^2 + \frac{\omega^2}{c_2^2} & \alpha_n \mu_0 s \\
        -\alpha_n \mu_0 s & s^2 - \alpha_n^2 - \alpha_n^2\mu_0 + \frac{\omega^2}{c_1^2}
     \end{pmatrix}
\end{equation}
\begin{align}
    &det[M(s)] = \begin{vmatrix}
        s^2 - \alpha_n^2 - \alpha_n^2\mu_0 + \frac{\omega^2}{c_1^2} & -\alpha_n \mu_0 s \\
        \alpha_n \mu_0 s & s^2 (1 + \mu_0) -\alpha_n^2 + \frac{\omega^2}{c_2^2}
     \end{vmatrix} = \nonumber \\
    &=(s - s_1)(s + s_1)(s - s_2)(s + s_2),
\end{align}
де $s_1$, $s_2$, $-s_1$, $-s_2$ корені $det[M(s)]=0$, детальне знаходження яких наведено у Додатку B.

Враховучи це, тепер знайдемо значення фундаментальної матриці за допомогою теореми про лишки:
\begin{align*}
    &\frac{1}{2\pi i} \oint_C e^{sy} M^{-1}(s)ds = \frac{2 \pi i}{2 \pi i (1 + \mu_0)} \sum_{i=1}^{4} Res\left[ e^{sy} \frac{\widetilde{M(s)}}{det[M(s)]} \right] = \\
    & = \left(Y_0(y) + Y_1(y) + Y_2(y) + Y_3(y) \right)
\end{align*}
де
\begin{align}
    &Y_0(y) =  \left( \frac{e^{sy}}{(s+s_1)(s - s_2)(s + s_2)} \widetilde{M(s)} \right) \Big|_{s=s_1} = \nonumber \\
    &=\frac{e^{s_1 y}}{2s_1 (s_1^2 - s_2^2)} \begin{pmatrix}
        s_1^2 (1 + \mu_0) -\alpha_n^2 + \frac{\omega^2}{c_2^2} & \alpha_n \mu_0 s_1 \\
        -\alpha_n \mu_0 s_1 & s_1^2 - \alpha_n^2 - \alpha_n^2\mu_0 + \frac{\omega^2}{c_1^2}
    \end{pmatrix}
\end{align}
\begin{align}
    &Y_1(y) =  \left( \frac{e^{sy}}{(s-s_1)(s - s_2)(s + s_2)} \widetilde{M(s)} \right) \Big|_{s=-s_1} = \nonumber \\
    &=-\frac{e^{-s_1 y}}{2s_1 (s_1^2 - s_2^2)} \begin{pmatrix}
        s_1^2 (1 + \mu_0) -\alpha_n^2 + \frac{\omega^2}{c_2^2} & -\alpha_n \mu_0 s_1 \\
        \alpha_n \mu_0 s_1 & s_1^2 - \alpha_n^2 - \alpha_n^2\mu_0 + \frac{\omega^2}{c_1^2}
    \end{pmatrix}
\end{align}
\begin{align}
    &Y_2(y) =  \left( \frac{e^{sy}}{(s+s_2)(s - s_1)(s + s_1)} \widetilde{M(s)} \right) \Big|_{s=s_2} = \nonumber \\
    &=\frac{e^{s_2 y}}{2s_2 (s_2^2 - s_1^2)} \begin{pmatrix}
        s_2^2 (1 + \mu_0) -\alpha_n^2 + \frac{\omega^2}{c_2^2} & \alpha_n \mu_0 s_2 \\
        -\alpha_n \mu_0 s_2 & s_2^2 - \alpha_n^2 - \alpha_n^2\mu_0 + \frac{\omega^2}{c_1^2}
    \end{pmatrix}
\end{align}
\begin{align}
    &Y_3(y) =  \left( \frac{e^{sy}}{(s-s_2)(s - s_1)(s + s_1)} \widetilde{M(s)} \right) \Big|_{s=-s_2} = \nonumber \\
    &=-\frac{e^{-s_2 y}}{2s_2 (s_2^2 - s_1^2)} \begin{pmatrix}
        s_2^2 (1 + \mu_0) -\alpha_n^2 + \frac{\omega^2}{c_2^2} & -\alpha_n \mu_0 s_2 \\
        \alpha_n \mu_0 s_2 & s_2^2 - \alpha_n^2 - \alpha_n^2\mu_0 + \frac{\omega^2}{c_1^2}
    \end{pmatrix}
\end{align}

Отримаємо розв'зок однорідного рівняння у просторі трансформант, який подано у формі
\begin{equation}
    Z_n(y) =\left( Y_0(y) +  Y_1(y)  \right) * \begin{pmatrix} c_1 \\ c_2 \end{pmatrix} + \left( Y_2(y) +  Y_3(y) \right) * \begin{pmatrix} c_3 \\ c_4 \end{pmatrix}
\end{equation}
Для того, щоб відшукати невідомі коєфіцієнти $c_i$, $i=\overline{1, 4}$ потрібно задовольнити граничні умови \eqref{transf_bound_mat_dynamic_1} (\nameref{ap_E}).
Тепер можна записати фінальний розв'язок векторної крайової задачі у просторі трансформант:
\begin{align}\label{transf_sol_u_dynamic_1}
    &u_n(y) = \frac{( s_1^2 (1 + \mu_0) -\alpha_n^2 + \frac{\omega^2}{c_2^2})(e^{s_1y} - e^{-s_1y})}{2s_1(s_1^2 - s_2^2)(1 + \mu_0)}c_1 + \nonumber \\
    & + \frac{( s_2^2 (1 + \mu_0) -\alpha_n^2 + \frac{\omega^2}{c_2^2})(e^{s_2y} - e^{-s_2y})}{2s_2(s_2^2 - s_1^2)(1 + \mu_0)}c_3 + \nonumber \\
    & + \frac{( s_1 \alpha_n y)(e^{s_1y} + e^{-s_1y})}{2s_1(s_1^2 - s_2^2)(1 + \mu_0)}c_2 + \frac{(s_2 \alpha_n y)(e^{s_2y} + e^{-s_2y})}{2s_2(s_2^2 - s_1^2)(1 + \mu_0)}c_4
\end{align}
\begin{align}\label{transf_sol_v_dynamic_1}
    &v_n(y) = \frac{(s_1^2 - \alpha_n^2 - \alpha_n^2\mu_0 + \frac{\omega^2}{c_1^2})(e^{s_1y} - e^{-s_1y})}{2s_1(s_1^2 - s_2^2)(1 + \mu_0)}c_2 + \nonumber \\
    & +\frac{(s_2^2 - \alpha_n^2 - \alpha_n^2\mu_0 + \frac{\omega^2}{c_1^2})(e^{s_2y} - e^{-s_2y})}{2s_2(s_2^2 - s_1^2)(1 + \mu_0)}c_4 - \nonumber \\
    & - \frac{(s_1 \alpha_n y)(e^{s_1y} + e^{-s_1y})}{2s_1(s_1^2 - s_2^2)(1 + \mu_0)}c_1 - \frac{(s_2 \alpha_n y)(e^{s_2y} + e^{-s_2y})}{2s_2(s_2^2 - s_1^2)(1 + \mu_0)}c_3
\end{align}

Використовуючи оберненне інтегральне перетворення Фур'є для трансформант переміщень \eqref{transf_sol_u_dynamic_1}, \eqref{transf_sol_v_dynamic_1} завершує побудову розв'язку вихідної задачі
\begin{equation}
    u(x,y) = \frac{2}{a} \sum_{n=1}^{\infty} u_n(y) sin(\alpha_n x), \quad \alpha_n = \frac{\pi n}{a}
\end{equation}
\begin{equation}
    v(x,y) = \frac{v_0(y)}{a} + \frac{2}{a} \sum_{n=1}^{\infty} v_n(y) cos(\alpha_n x), \quad \alpha_n = \frac{\pi n}{a}
\end{equation}

Останній крок - це знаходження $v_0(y)$ у випадку коли $n=0$, $\alpha_0 =0$.
Для цього повернемся до другого рівняння (\ref{transf_dynamic_1}), та запишем його для цього випадку:
\begin{equation}\label{transf_v_0_dynamic_1}
    (1 + \mu_0) v_0^{''}(y) + \frac{\omega^2}{c_2^2}v_0(y) = 0
\end{equation}
Та запишимо відповідні граничні умови:
\begin{equation}\label{transf_bound_v_0_dynamic_1}
    \begin{cases}
        (2G + \lambda)v_0^{'}(y)|_{y=b} = -p_0 \\
        v_0(y)|_{y=0} = 0
    \end{cases}
\end{equation}
де $p_0 = \int_{0}^{a}p(x)dx$

Розв'язок рівняння (\ref{transf_v_0_dynamic_1}) буде мати вигляд:
\begin{equation}
    v_0(y) = c_1 cos \left(y \sqrt{\frac{\omega^2}{c_2^2(1 + \mu_0)}} \right) + c_2 sin \left( y \sqrt{\frac{\omega^2}{c_2^2(1 + \mu_0)}} \right)
\end{equation}
Отже
\begin{equation}
    v_0(y) = \frac{-p_0}{(2G + \lambda) \sqrt{\frac{\omega^2}{c_2^2(1 + \mu_0)}} sin \left(b \sqrt{\frac{\omega^2}{c_2^2(1 + \mu_0)}} \right) } sin \left(y \sqrt{\frac{\omega^2}{c_2^2(1 + \mu_0)}} \right)
\end{equation}

\subsubsection{Числові розрахунки}
Наведені чисельні експеренти розглядаються для сталі ($E=200$ ГПА, $\mu=0.25$).

Розглянута прямокунта область $0 \le x \le 10$, $0 \le y \le 15$, при функції навантаження $p(x)=(x-2.5)^2$ та частоті коливань $\omega=0.75$.
На малюнках (Рис: \ref{static_1_u_1}), (Рис: \ref{static_1_v_1}), (Рис: \ref{static_1_sigma_x_1}), (Рис: \ref{static_1_sigma_y_1})
представлені функіі переміщень $u(x,y)$, $v(x,y)$ та напружень $\sigma_x(x,y)$, $\sigma_y(x,y)$ відповідно.

\newpage

\section{СТАТИЧНА ЗАДАЧА ТЕОРІЇ ПРУЖНОСТІ ДЛЯ ПРЯМОКУТНОЇ ОБЛАСТІ
ЗА УМОВ ДРУГОЇ ОСНОВНОЇ ЗАДАЧІ ТЕОРІЇ ПРУЖНОСТІ}
У даному розділі досліджено плоска статична задача теорії пружності для прямокутної області,
за умов другої основної задачі теорії пружності на бічних гранях.

Вихідна задача зведена до одновимірної задачі у просторі трансформант за допомогою інтегрального перетворення Фур'є.
Отримана крайова задача розв'язана точно за допомогою методу матрично диференціального числення,
фундаментальний розв'язок представлений як інтеграл по замкненому контору, який в свою чергу, був знайденний за допомогою теоремі про лишки.
Остаточний вигляд для функцій переміщеннь та напружень отриман шляхом оберненого перетворення Фур'є.
Побудовано та розв'язано сінгульрне інтегральне рівняння відносно невідомої функції шляхом викорстання методу ортагональних поліномів, та зведення рівнняння до бескінечної алгебричної системи,
яка в подальшому була розв'язана методом редукціі.

Проведено чисельний аналіз отриманих функцій переміщень та напружень для різних розмірів прямокутної області та різних видів навантаження.

\subsection{Постановка задачі}
\begin{figure}[ht!]
    \begin{center}
        \includegraphics[scale=1]{images/geometry/image_1.jpg}
    \end{center}
    \caption{Геометрія проблеми}\label{geom_static_2}
\end{figure}
Розглядається пружна прямокутна область (Рис: \ref{geom_static_2}), яка займає облась,
що описується у декартовій системі координат співвідношенням $-a \le x \le a$, $0 \le y \le b$.

До прямокутної області на грані $y=b$ додане нормальне навантаження
\begin{equation}
    \sigma_y(x, y) |_{y=b} = -p(x), \quad  \tau_{xy}(x,y) |_{y=b} =0
\end{equation}
де $p(x)$ відома функція.

На бічних гранях виконується умова другої основної задачі теорії пружності
\begin{equation}
    u(x,y) |_{x=\pm a} = 0, \quad v(x,y) |_{x=\pm a} =0
\end{equation}
На нижній грані виконуються наступні умови
\begin{equation}
    v(x,y) |_{y=0} = 0, \quad \tau_{xy}(x,y) |_{y=0} =0
\end{equation}
Розглядаються наступні рівняння рівноваги Ламе:
\begin{equation}\label{lame_static_2}
    \begin{cases}
        \frac{\partial^2 u(x,y)}{\partial x^2} + \frac{\partial^2 u(x,y)}{\partial y^2} + \mu_0 (\frac{\partial^2 u(x,y)}{\partial x^2} + \frac{\partial^2 v(x,y)}{\partial x\partial y}) = 0 \\
        \frac{\partial^2 v(x,y)}{\partial x^2} + \frac{\partial^2 v(x,y)}{\partial y^2} + \mu_0 (\frac{\partial^2 u(x,y)}{\partial x \partial y} + \frac{\partial^2 v(x,y)}{\partial y^2}) = 0 \\
    \end{cases}
\end{equation}
Щоб розв'язати поставлену задачу буде розглянута тільки половона прямокутної області $0 \le x \le a$, $0 \le y \le b$
та використовуючи властивості симметрії граничні умови на бічних гранях будуть мати вигляд:
\begin{equation}\label{bound_1_static_2}
    u(x,y) |_{x=0} = 0, \quad \tau_{xy}(x,y) |_{x=0} =0
\end{equation}
\begin{equation}\label{bound_2_static_2}
    u(x,y) |_{x=a} = 0, \quad v(x,y) |_{x=a} = 0
\end{equation}

\subsection{Зведеня задачі до одновимірної у просторі трансформант}
Для того, щоб звести задачу до одновимірної задачі, використаєм інтегральне перетворення Фур'є по змінній $x$ у до рівнянь (\ref{lame_static_2}) наступному вигляді:
\begin{equation}
    \begin{pmatrix}
        u_n(y) \\
        v_n(y)
    \end{pmatrix} = \int_{0}^{a} 
    \begin{pmatrix}
        u(x,y) sin(\alpha_n x) \\
        v(x,y) cos(\alpha_n x)
    \end{pmatrix} dx, \quad \alpha_n = \frac{\pi n}{a}
\end{equation}

Для цього помножим перше та друге рівняння (\ref{lame_static_2}) на $sin(\alpha_n x)$ та $cos(\alpha_n x)$ відповідно та проінтегруєм по змінній $x$ на інтервалі $0 \le x \le a$.
Покрокове інтегрування рівняння (\ref{lame_static_2}) наведено у (\nameref{ap_A}).
Отримана система рівнянь задачі у просторі трансформант:
\begin{equation}\label{transf_static_2}
    \begin{cases}
        u_n^{''}(y) - \alpha_n \mu_0 v_n^{'}(y) - \alpha_n^2 (1 + \mu_0) u_n(y) = 0 \\
        (1 + \mu_0) v_n^{''}(y) + \alpha_n \mu_0 u_n^{'}(y)  - \alpha_n^2 v_n(y) = -cos(\alpha_n) f(y) \\
    \end{cases}
\end{equation}
Де $f(y) = \frac{\partial v(x,y)}{\partial x}|_{x=a}$ - невідома функція

Застосовуючи інтегральне перетворення до граничних умов,
отримаємо наступні умови задачі у просторі трансформант
\begin{equation}\label{transf_bound_static_2}
    \begin{cases}
        \left( (2G + \lambda)v_n^{'}(y) + \alpha_n \lambda u_n(y) \right)|_{y=b} = -p_n \\
        \left(u_n^{'}(y) - \alpha_n v_n(y)  \right)|_{y=b} = 0 \\
        v_n(y)|_{y=0} = 0 \\
        \left(u_n^{'}(y) - \alpha_n v_n(y)  \right)|_{y=0} = 0
    \end{cases}
\end{equation}
Де $p_n = \int_{0}^{a} p(x) cos(\alpha_n x) dx$

\subsection{Зведення задачі у просторі трансформант до матрично-векторної форми}
Для того щоб розв'язати задачу у простосторі трансформант, перепишмо її у матрично-векторній формі.
Рівняння рівноваги (\ref{transf_static_2}) запишемо у наступному вигляді:
\begin{align}\label{transf_mat_static_2}
    &L_2\left[ Z_n(y) \right] = A * Z_n^{''}(y) + B * Z_n^{'}(y) + C * Z_n(y) \nonumber \\
    &L_2\left[ Z_n(y) \right] = F_n(y)
\end{align}
Де
\begin{equation*}
    A = \begin{pmatrix}
        1 & 0 \\
        0 & 1 + \mu_0
    \end{pmatrix}, \quad
    B = \begin{pmatrix}
        0 & -\alpha_n \mu_0 \\
        \alpha_n \mu_0 & 0
    \end{pmatrix}, \quad
    C = \begin{pmatrix}
        -\alpha_n^2(1 + \mu_0) & 0 \\
        0 & -\alpha_n^2
    \end{pmatrix}
\end{equation*}
\begin{equation*}
    Z_n(y) = \begin{pmatrix}
        u_n(y) \\
        v_n(y)
    \end{pmatrix}, \quad 
    F_n(y) = \begin{pmatrix}
        0 \\
        - cos(\alpha_n a) f(y)
    \end{pmatrix}
\end{equation*}
Граничні умови (\ref{transf_bound_static_2}) запишемо у наступному вигляді:
\begin{align}\label{transf_bound_mat_static_2}
    &U_i\left[ Z_n(y) \right] = E_i * Z_n^{'}(b_i) + F_i * Z_n(b_i) \nonumber \\
    & U_i\left[ Z_n(y) \right] = D_i
\end{align}
Де $i = \overline{0, 1}$, $b_0 = b$, $b_1 = 0$,
\begin{equation*}
    E_0 = \begin{pmatrix}
        1 & 0 \\
        0 & 2G + \lambda
    \end{pmatrix}, \quad
    F_0 = \begin{pmatrix}
        0 & -\alpha_n \\
        \alpha_n \lambda & 0
    \end{pmatrix}, \quad
\end{equation*}
\begin{equation*}
    E_1 = \begin{pmatrix}
        1 & 0 \\
        0 & 0
    \end{pmatrix}, \quad
    F_1 = \begin{pmatrix}
        0 & -\alpha_n \\
        0 & 1
    \end{pmatrix}, \quad
\end{equation*}
\begin{equation*}
    D_0 = \begin{pmatrix}
        0 \\
        -p_n
    \end{pmatrix}, \quad
    D_1 = \begin{pmatrix}
        0 \\
        0
    \end{pmatrix}, \quad
\end{equation*}

Для знаходження розв'язку задачі у просторі трансформант, знайдем фундаментальну матрицю рівняння (\ref{transf_mat_static_2}).
Шукати її будем у наступному вигляді:
\begin{equation}
    Y(y) = \frac{1}{2\pi i} \oint_C e^{sy} M^{-1}(s)ds
\end{equation}
Де $M(s)$ - характерестична матриця рівняння (\ref{transf_mat_static_2}), а $C$ - замкнений контур який містить усі особливі точки. Яку будемо шукати з наступної умовни
\begin{equation}
    L_2\left[ e^{sy}*I \right] = e^{sy} * M(s), \quad I = \begin{pmatrix} 1 & 0 \\ 0 & 1 \end{pmatrix}
\end{equation}
\begin{align*}
    &L_2\left[ e^{sy}*I \right] = e^{sy} \left( s^2A * I + s B*I + C*I \right) = \\
    &=e^{sy} \left( \begin{pmatrix}
        s^2 & 0 \\
        0 & s^2 (1 + \mu_0)
    \end{pmatrix} + \begin{pmatrix}
        0 & -\alpha_n \mu_0 s\\
        \alpha_n \mu_0 s & 0
    \end{pmatrix} + \begin{pmatrix}
        -\alpha_n^2(1 + \mu_0) & 0 \\
        0 & -\alpha_n^2
    \end{pmatrix} \right) =  \\
    &=e^{sy} \begin{pmatrix}
        s^2 -\alpha_n^2(1 + \mu_0) & -\alpha_n \mu_0 s \\
        \alpha_n \mu_0 s & s^2 (1 + \mu_0) -\alpha_n^2
     \end{pmatrix} =>
\end{align*}

\begin{equation}
    M(s) = \begin{pmatrix}
        s^2 -\alpha_n^2(1 + \mu_0) & -\alpha_n \mu_0 s \\
        \alpha_n \mu_0 s & s^2 (1 + \mu_0) -\alpha_n^2
     \end{pmatrix}
\end{equation}

Знайдемо тепер $M^{-1}(s) = \frac{\widetilde{M(s)}}{det[M(s)]}$.
\begin{equation}
    \widetilde{M(s)} = \begin{pmatrix}
        s^2 (1 + \mu_0) -\alpha_n^2 & \alpha_n \mu_0 s \\
        -\alpha_n \mu_0 s & s^2 -\alpha_n^2(1 + \mu_0)
     \end{pmatrix}
\end{equation}
\begin{align}
    &det[M(s)] = \begin{vmatrix}
        s^2 - \alpha_n^2 - \alpha_n^2\mu_0 & -\alpha_n \mu_0 s \\
        \alpha_n \mu_0 s & s^2 (1 + \mu_0) -\alpha_n^2
     \end{vmatrix} = \nonumber \\
    &=(1+\mu_0)(s - \alpha_n)^2(s + \alpha_n)^2
\end{align}
Де $\alpha_n$, $-\alpha_n$, корені $det[M(s)]=0$, детальне знаходження яких наведено в (\nameref{ap_B}).

Враховучи це, тепер знайдемо значення фундаментальної матрицю за допомогою теореми про лишки:
\begin{align*}
    &\frac{1}{2\pi i} \oint_C e^{sy} M^{-1}(s)ds = \frac{2 \pi i}{2 \pi i (1 + \mu_0)} \sum_{i=1}^{2} Res\left[ e^{sy} \frac{\widetilde{M(s)}}{det[M(s)]} \right] = \\
    & = \frac{1}{(1 + \mu_0)} \left(Y_0(y) + Y_1(y) \right)
\end{align*}
Знайдем $Y_0(y)$:
\begin{align}
    &Y_0(y) =  \frac{\partial}{\partial s} \left( \frac{e^{sy}}{(s+\alpha_n)^2} \widetilde{M(s)} \right) \Big|_{s=\alpha_n} = \nonumber \\
    &=\frac{e^{\alpha_n y}}{4\alpha_n} \begin{pmatrix}
    \alpha_n \mu_0 y + 2 + \mu_0 & \alpha_n \mu_0 y \\
    -\alpha_n \mu_0 y & -\alpha_n \mu_0 y + 2 + \mu_0
    \end{pmatrix}
\end{align}
Знайдем $Y_1(y)$:
\begin{align}
    &Y_1(y) = \frac{\partial}{\partial s} \left(\frac{e^{sy}}{(s-\alpha_n)^2} \widetilde{M(s)} \right) \Big|_{s=-\alpha_n} = \nonumber \\
    =&\frac{e^{-\alpha_n y}}{4\alpha_n} \begin{pmatrix}
    \alpha_n \mu_0 y - 2 - \mu_0 & -\alpha_n \mu_0 y \\
    \alpha_n \mu_0 y & -\alpha_n \mu_0 y - 2 - \mu_0
    \end{pmatrix}
\end{align}
\newpage

\section{ДИНАМІЧНА ЗАДАЧА ТЕОРІЇ ПРУЖНОСТІ ДЛЯ ПРЯМОКУТНОЇ ОБЛАСТІ
ЗА УМОВ ДРУГОЇ ОСНОВНОЇ ЗАДАЧІ ТЕОРІЇ ПРУЖНОСТІ}
У даному розділі досліджено плоска динамічна задача теорії пружності для прямокутної області,
за другої основної задачі теорії пружності на бічних гранях.

Вихідна задача зведена до одновимірної задачі у просторі трансформант за допомогою інтегрального перетворення Фур'є.
Отримана крайова задача розв'язана точно за допомогою методу матрично диференціального числення,
фундаментальний розв'язок представлений як інтеграл по замкненому контору, який в свою чергу, був знайденний за допомогою теоремі про лишки.
Побудована матриця-функція Гріна як комбінація фундаментальних базисних розв'язків задачі у просторі трансформант.
Остаточний вигляд для функцій переміщеннь та напружень отриман шляхом оберненого перетворення Фур'є.
Побудовано та розв'язано сінгульрне інтегральне рівняння відносно невідомої функції шляхом викорстання методу ортагональних многочленів, та зведення рівнняння до бескінечної алгебричної системи,
яка в подальшому була розв'язана методом редукціі \cite{popov_3}.

Проведено чисельний аналіз отриманих функцій переміщень та напружень для різних розмірів прямокутної області та різних видів навантаження.

\subsection{Постановка задачі}
\begin{figure}[h]
    \begin{center}
        \includegraphics[scale=1]{images/geometry/image_2.jpg}
    \end{center}
    \caption{Геометрія проблеми}\label{geom_dynamic_2}
\end{figure}
Розглядається пружна сама прямокутна область (Рис: \ref{geom_dynamic_1}), яка займає облась,
що описується у декартовій системі координат співвідношенням $0 \le x \le a$, $0 \le y \le b$.

До прямокутної області на грані $y=b$ додане нормальне навантаження
\begin{equation}
    \sigma_y(x, y, t) |_{y=b} = -p(x, t), \quad  \tau_{xy}(x,y,t) |_{y=b} =0
\end{equation}
де $p(x,t)$ відома функція.
На бічних гранях виконується умова другої основної задачі теорії пружності
\begin{equation}
    u(x,y,t) |_{x=\pm a} = 0, \quad v(x,y,t) |_{x=\pm a} =0
\end{equation}
На нижній грані виконуються наступні умови
\begin{equation}
    v(x,y,t) |_{y=0} = 0, \quad \tau_{xy}(x,y,t) |_{y=0} =0
\end{equation}
Розглядаються наступні рівняння рівноваги Ламе:
\begin{equation}
    \begin{cases}
        \frac{\partial^2 u(x,y,t)}{\partial x^2} + \frac{\partial^2 u(x,y,t)}{\partial y^2} + \mu_0 (\frac{\partial^2 u(x,y,t)}{\partial x^2} + \frac{\partial^2 v(x,y,t)}{\partial x\partial y}) = \frac{1}{c_1^2} \frac{\partial^2 u(x,y,t)}{\partial t^2} \\
        \frac{\partial^2 v(x,y,t)}{\partial x^2} + \frac{\partial^2 v(x,y,t)}{\partial y^2} + \mu_0 (\frac{\partial^2 u(x,y,t)}{\partial x \partial y} + \frac{\partial^2 v(x,y,t)}{\partial y^2}) = \frac{1}{c_2^2} \frac{\partial^2 v(x,y,t)}{\partial t^2} \\
    \end{cases}
\end{equation}

Будемо розглядати випадок гармонічних коливань, тому можемо предствавити функції у наступному вигляді:
\begin{equation}
    u(x,y,t) = u(x,y) e^{i \omega t}, \quad v(x,y,t) = v(x,y) e^{i \omega t}, \quad p(x,t) = p(x) e^{i \omega t}
\end{equation}
Таким чином отримаємо наступні рівняння рівноваги:
\begin{equation}\label{lame_dynamic_2}
    \begin{cases}
        \frac{\partial^2 u(x,y)}{\partial x^2} + \frac{\partial^2 u(x,y)}{\partial y^2} + \mu_0 (\frac{\partial^2 u(x,y)}{\partial x^2} + \frac{\partial^2 v(x,y)}{\partial x\partial y}) = -\frac{\omega^2}{c_1^2}  u(x,y) \\
        \frac{\partial^2 v(x,y)}{\partial x^2} + \frac{\partial^2 v(x,y)}{\partial y^2} + \mu_0 (\frac{\partial^2 u(x,y)}{\partial x \partial y} + \frac{\partial^2 v(x,y)}{\partial y^2}) = -\frac{\omega^2}{c_2^2} v(x,y) \\
    \end{cases}
\end{equation}
Також, щоб розв'язати поставлену задачу буде розглянута тільки половона прямокутної області $0 \le x \le a$, $0 \le y \le b$
та використовуючи властивості симметрії граничні умови в результаті будуть мати вигляд:
\begin{equation}\label{bound_dynamic_2}
    \begin{cases}
        \sigma_y(x, y) |_{y=b} = -p(x, t), \quad  \tau_{xy}(x,y) |_{y=b} =0 \\
        v(x,y) |_{y=0} = 0, \quad \tau_{xy}(x,y) |_{y=0} = 0 \\
        u(x,y) |_{x=0} = 0, \quad \tau_{xy}(x,y) |_{x=0} = 0 \\
        u(x,y) |_{x=a} = 0, \quad v(x,y) |_{x=a} = 0
    \end{cases}
\end{equation}

\subsection{Зведеня задачі до одновимірної у просторі трансформант}
Для того, щоб звести задачу до одновимірної задачі, використаєм інтегральне перетворення Фур'є по змінній $x$ у до рівнянь (\ref{lame_dynamic_2}) наступному вигляді:
\begin{equation}
    \begin{pmatrix}
        u_n(y) \\
        v_n(y)
    \end{pmatrix} = \int_{0}^{a} 
    \begin{pmatrix}
        u(x,y) sin(\alpha_n x) \\
        v(x,y) cos(\alpha_n x)
    \end{pmatrix} dx, \quad \alpha_n = \frac{\pi n}{a}
\end{equation}

Для цього помножим перше та друге рівняння (\ref{lame_dynamic_2}) на $sin(\alpha_n x)$ та $cos(\alpha_n x)$ відповідно та проінтегруєм по змінній $x$ на інтервалі $0 \le x \le a$.
Покрокове інтегрування рівняння (\ref{lame_dynamic_2}) наведено у (\nameref{ap_A}).
Отримана система рівнянь задачі у просторі трансформант:
\begin{equation}\label{transf_dynamic_2}
    \begin{cases}
        u_n^{''}(y) - \alpha_n \mu_0 v_n^{'}(y) + (-\alpha_n^2 -\alpha_n^2 \mu_0 + \frac{\omega^2}{c_1^2}) u_n(y) = 0 \\
        (1 + \mu_0) v_n^{''}(y) + \alpha_n \mu_0 u_n^{'}(y) + (- \alpha_n^2 + \frac{\omega^2}{c_2^2}) v_n(y) = -cos(\alpha_n) f(y)\\
    \end{cases}
\end{equation}
Де $f(y) = \frac{\partial v(x,y)}{\partial x}|_{x=a}$ - невідома функція

Застосовуючи інтегральне перетворення до граничних умов,
отримаємо наступні умови задачі у просторі трансформант
\begin{equation}\label{transf_bound_dynamic_2}
    \begin{cases}
        \left( (2G + \lambda)v_n^{'}(y) + \alpha_n \lambda u_n(y) \right)|_{y=b} = -p_n \\
        \left(u_n^{'}(y) - \alpha_n v_n(y)  \right)|_{y=b} = 0 \\
        v_n(y)|_{y=0} = 0 \\
        \left(u_n^{'}(y) - \alpha_n v_n(y)  \right)|_{y=0} = 0
    \end{cases}
\end{equation}
Де $p_n = \int_{0}^{a} p(x) cos(\alpha_n x) dx$

\subsection{Зведення задачі у просторі трансформант до матрично-векторної форми}
Для того щоб розв'язати задачу у простосторі трансформант, перепишмо її у матрично-векторній формі.
Рівняння рівноваги (\ref{transf_dynamic_2}) запишемо у наступному вигляді:
\begin{align}\label{transf_mat_dynamic_2}
    &L_2\left[ Z_n(y) \right] = A * Z_n^{''}(y) + B * Z_n^{'}(y) + C * Z_n(y) \nonumber \\
    &L_2\left[ Z_n(y) \right] = F_n(y)
\end{align}
Де
\begin{equation*}
    A = \begin{pmatrix}
        1 & 0 \\
        0 & 1 + \mu_0
    \end{pmatrix}, \quad
    B = \begin{pmatrix}
        0 & -\alpha_n \mu_0 \\
        \alpha_n \mu_0 & 0
    \end{pmatrix},
\end{equation*}
\begin{equation*}
    \centering
    C = \begin{pmatrix}
        -\alpha_n^2 -\alpha_n^2 \mu_0 + \frac{\omega^2}{c_1^2} & 0 \\
        0 & -\alpha_n^2 + \frac{\omega^2}{c_2^2}
    \end{pmatrix},
\end{equation*}
\begin{equation*}
    Z_n(y) = \begin{pmatrix}
        u_n(y) \\
        v_n(y)
    \end{pmatrix}, \quad 
    F_n(y) = \begin{pmatrix}
        0 \\
        - cos(\alpha_n a) f(y)
    \end{pmatrix}
\end{equation*}
Граничні умови (\ref{transf_bound_dynamic_2}) запишемо у наступному вигляді:
\begin{align}\label{transf_bound_mat_dynamic_2}
    &U_i\left[ Z_n(y) \right] = E_i * Z_n^{'}(b_i) + F_i * Z_n(b_i) \nonumber \\
    & U_i\left[ Z_n(y) \right] = D_i
\end{align}
Де $i = \overline{0, 1}$, $b_0 = b$, $b_1 = 0$,
\begin{equation*}
    E_0 = \begin{pmatrix}
        1 & 0 \\
        0 & 2G + \lambda
    \end{pmatrix}, \quad
    F_0 = \begin{pmatrix}
        0 & -\alpha_n \\
        \alpha_n \lambda & 0
    \end{pmatrix}, \quad
\end{equation*}
\begin{equation*}
    E_1 = \begin{pmatrix}
        1 & 0 \\
        0 & 0
    \end{pmatrix}, \quad
    F_1 = \begin{pmatrix}
        0 & -\alpha_n \\
        0 & 1
    \end{pmatrix}, \quad
\end{equation*}
\begin{equation*}
    D_0 = \begin{pmatrix}
        0 \\
        -p_n
    \end{pmatrix}, \quad
    D_1 = \begin{pmatrix}
        0 \\
        0
    \end{pmatrix}, \quad
\end{equation*}

Для знаходження розв'язку задачі у просторі трансформант, знайдем фундаментальну матрицю рівняння (\ref{transf_mat_dynamic_2}).
Шукати її будем у наступному вигляді \cite{gantmaher}:
\begin{equation}
    Y(y) = \frac{1}{2\pi i} \oint_C e^{sy} M^{-1}(s)ds
\end{equation}
Де $M(s)$ - характерестична матриця рівняння (\ref{transf_mat_dynamic_2}), а $C$ - замкнений контур який містить усі особливі точки. Яку будемо шукати з наступної умовни
\begin{equation}
    L_2\left[ e^{sy}*I \right] = e^{sy} * M(s), \quad I = \begin{pmatrix} 1 & 0 \\ 0 & 1 \end{pmatrix}
\end{equation}
\begin{align*}
    &L_2\left[ e^{sy}*I \right] = e^{sy} \left( s^2A * I + s B*I + C*I \right) = \\
    &=e^{sy} \begin{pmatrix}
        s^2 - \alpha_n^2 - \alpha_n^2\mu_0 + \frac{\omega^2}{c_1^2} & -\alpha_n \mu_0 s \\
        \alpha_n \mu_0 s & s^2 (1 + \mu_0) -\alpha_n^2 + \frac{\omega^2}{c_1^2}
     \end{pmatrix} \Rightarrow
\end{align*}

\begin{equation}
    M(s) = \begin{pmatrix}
        s^2 - \alpha_n^2 - \alpha_n^2\mu_0 + \frac{\omega^2}{c_1^2} & -\alpha_n \mu_0 s \\
        \alpha_n \mu_0 s & s^2 (1 + \mu_0) -\alpha_n^2 + \frac{\omega^2}{c_2^2}
     \end{pmatrix}
\end{equation}

Знайдемо тепер $M^{-1}(s) = \frac{\widetilde{M(s)}}{det[M(s)]}$.
\begin{equation}
    \widetilde{M(s)} = \begin{pmatrix}
        s^2 (1 + \mu_0) -\alpha_n^2 + \frac{\omega^2}{c_2^2} & \alpha_n \mu_0 s \\
        -\alpha_n \mu_0 s & s^2 - \alpha_n^2 - \alpha_n^2\mu_0 + \frac{\omega^2}{c_1^2}
     \end{pmatrix}
\end{equation}
\begin{align}
    &det[M(s)] = \begin{vmatrix}
        s^2 - \alpha_n^2 - \alpha_n^2\mu_0 + \frac{\omega^2}{c_1^2} & -\alpha_n \mu_0 s \\
        \alpha_n \mu_0 s & s^2 (1 + \mu_0) -\alpha_n^2 + \frac{\omega^2}{c_2^2}
     \end{vmatrix} = \nonumber \\
    &=(s - s_1)(s + s_1)(s - s_2)(s + s_2)
\end{align}
Де $s_1$, $s_2$, $-s_1$, $-s_2$ корені $det[M(s)]=0$, детальне знаходження яких наведено в (\nameref{ap_B}).

Враховучи це, тепер знайдемо значення фундаментальної матрицю за допомогою теореми про лишки:
\begin{align*}
    &\frac{1}{2\pi i} \oint_C e^{sy} M^{-1}(s)ds = \frac{2 \pi i}{2 \pi i (1 + \mu_0)} \sum_{i=1}^{4} Res\left[ e^{sy} \frac{\widetilde{M(s)}}{det[M(s)]} \right] = \\
    & = \left(Y_0(y) + Y_1(y) + Y_2(y) + Y_3(y) \right)
\end{align*}
Знайдем $Y_0(y)$:
\begin{align}
    &Y_0(y) =  \left( \frac{e^{sy}}{(s+s_1)(s - s_2)(s + s_2)} \widetilde{M(s)} \right) \Big|_{s=s_1} = \nonumber \\
    &=\frac{e^{s_1 y}}{2s_1 (s_1^2 - s_2^2)} \begin{pmatrix}
        s_1^2 (1 + \mu_0) -\alpha_n^2 + \frac{\omega^2}{c_2^2} & \alpha_n \mu_0 s_1 \\
        -\alpha_n \mu_0 s_1 & s_1^2 - \alpha_n^2 - \alpha_n^2\mu_0 + \frac{\omega^2}{c_1^2}
    \end{pmatrix}
\end{align}
Знайдем $Y_1(y)$:
\begin{align}
    &Y_1(y) =  \left( \frac{e^{sy}}{(s-s_1)(s - s_2)(s + s_2)} \widetilde{M(s)} \right) \Big|_{s=-s_1} = \nonumber \\
    &=-\frac{e^{-s_1 y}}{2s_1 (s_1^2 - s_2^2)} \begin{pmatrix}
        s_1^2 (1 + \mu_0) -\alpha_n^2 + \frac{\omega^2}{c_2^2} & -\alpha_n \mu_0 s_1 \\
        \alpha_n \mu_0 s_1 & s_1^2 - \alpha_n^2 - \alpha_n^2\mu_0 + \frac{\omega^2}{c_1^2}
    \end{pmatrix}
\end{align}
Знайдем $Y_2(y)$:
\begin{align}
    &Y_2(y) =  \left( \frac{e^{sy}}{(s+s_2)(s - s_1)(s + s_1)} \widetilde{M(s)} \right) \Big|_{s=s_2} = \nonumber \\
    &=\frac{e^{s_2 y}}{2s_2 (s_2^2 - s_1^2)} \begin{pmatrix}
        s_2^2 (1 + \mu_0) -\alpha_n^2 + \frac{\omega^2}{c_2^2} & \alpha_n \mu_0 s_2 \\
        -\alpha_n \mu_0 s_2 & s_2^2 - \alpha_n^2 - \alpha_n^2\mu_0 + \frac{\omega^2}{c_1^2}
    \end{pmatrix}
\end{align}
Знайдем $Y_3(y)$:
\begin{align}
    &Y_3(y) =  \left( \frac{e^{sy}}{(s-s_2)(s - s_1)(s + s_1)} \widetilde{M(s)} \right) \Big|_{s=-s_2} = \nonumber \\
    &=-\frac{e^{-s_2 y}}{2s_2 (s_2^2 - s_1^2)} \begin{pmatrix}
        s_2^2 (1 + \mu_0) -\alpha_n^2 + \frac{\omega^2}{c_2^2} & -\alpha_n \mu_0 s_2 \\
        \alpha_n \mu_0 s_2 & s_2^2 - \alpha_n^2 - \alpha_n^2\mu_0 + \frac{\omega^2}{c_1^2}
    \end{pmatrix}
\end{align}

\subsection{Побудова матриці-функції Гріна}
Для побудови матриці-функції Гріна спочатку знайдем тепер фундамельні бизисні матриці $\Psi_0(y)$, $\Psi_1(y)$, шукати їх будем у наступному вигляді:
\begin{equation}\label{psi_dynamic_2}
    \Psi_i(y) = \left( Y_0(y) + Y_1(y) \right) * C_1^i + \left( Y_2(y) + Y_3(y) \right) * C_2^i
\end{equation}

Залишилось знайти невідомі матриці коєфіцієнтів $C_1^0$, $C_2^0$, $C_1^1$, $C_2^1$ використовуючи граничні умови \eqref{transf_bound_mat_dynamic_2}.
Покрокове знаходження яких наведено у (\nameref{ap_C}).
Для подальшого введемо наступні позначення для елементів матриць $\Psi_0(y)$, $\Psi_1(y)$:
\begin{equation*}
    \Psi_0(y) = \begin{pmatrix}
        \Psi_1^0(y) &  \Psi_2^0(y) \\
        \Psi_3^0(y) &  \Psi_4^0(y) 
    \end{pmatrix}, \quad 
    \Psi_1(y) = \begin{pmatrix}
        \Psi_1^1(y) &  \Psi_2^1(y) \\
        \Psi_3^1(y) &  \Psi_4^1(y) 
    \end{pmatrix}      
\end{equation*}

Таким чином матрицю Гріна можемо записати у вигляді:
\begin{equation}
    G(y,\xi) = 
    \begin{cases}
        \Psi_0(y) * \Psi_1(\xi), \quad 0 \le y < \xi \\
        \Psi_1(y) * \Psi_0(\xi), \quad \xi < y \le b
    \end{cases}
\end{equation}

Для данної матриці Гріна виконано усі властивості, зокрема виконані однорідні граничні умови \eqref{transf_bound_mat_dynamic_2}
та однорідні рівняння рівноваги у просторі трансформант \eqref{transf_mat_dynamic_2}:
\begin{equation*}
    L_2\left[  G(y, \xi) \right] = 0
\end{equation*}
\begin{equation*}
    U_0\left[ G(y, \xi) \right] = 0, \quad  U_1\left[ G(y, \xi) \right],
\end{equation*}

Введемо наступні позначення $G(y, \xi) = \begin{pmatrix}
    g_1(y,\xi) & g_2(y,\xi) \\
    g_3(y,\xi) & g_4(y,\xi)
\end{pmatrix}$. Враховуючи це, шукані функціі перемішень у просторі трансформант можна записати у наступному вигляді
\begin{align}\label{transf_sol_u_dynamic_2}
    &u_n(y) = -\int_0^b g_2(y, \xi)f(\xi) cos(\alpha_n a) d\xi - \psi_0^2(y) p_n
\end{align}
\begin{align}\label{transf_sol_v_dynamic_2}
    &v_n(y) = -\int_0^b g_4(y, \xi)f(\xi) cos(\alpha_n a) d\xi - \psi_0^4(y) p_n
\end{align}

\subsection{Побудова розв'язоку вихідної задачі}
Викорустовуючи обернене інтегральне перетворення Фур'є до розв'язку задачі у просторі трансформант
(\ref{transf_sol_u_dynamic_2}), (\ref{transf_sol_u_dynamic_2}), отримаємо фінальний розв'язок задачі
\begin{equation}
    u(x,y) = \frac{2}{a} \sum_{n=1}^{\infty} u_n(y) sin(\alpha_n x), \quad \alpha_n = \frac{\pi n}{a}
\end{equation}
\begin{equation}
    v(x,y) = \frac{v_0(y)}{a} + \frac{2}{a} \sum_{n=1}^{\infty} v_n(y) cos(\alpha_n x), \quad \alpha_n = \frac{\pi n}{a}
\end{equation}

Знайдем тепер $v_0(y)$ розглянувши задачу у просторі трансформант \eqref{transf_dynamic_2}, \eqref{transf_bound_dynamic_2} при $n=0$, $\alpha_n = 0$.
Детальний розв'язок якої наведено в (\nameref{ap_D}). Тоді остаточний розв'язок $v(x,y)$ буде мати вигляд
\begin{align}
    &v(x,y) = -\frac{1}{a(1+\mu_0)} \int_{0}^{b}g(y,\xi) f(\xi) d\xi - \psi_0(y) \frac{p_0}{a(2G + \lambda)} \nonumber \\
    &- \frac{2}{a} \sum_{n=1}^{\infty} \left( \int_0^b \left[g_4(y, \xi) cos(\alpha_n a) f(\xi) \right]d\xi + \psi_0^4(y) p_n  \right) cos(\alpha_n x)
\end{align}
\newpage

\begin{thebibliography}{1}
    \bibitem{popov_1}
    Попов Г. Я. Концентрация упругих напряжений возле штампов разрезов тонких включений и подкреплений. М.: Наука. Главная редакция физико-математической литературы, 1982. 344 с.
    \bibitem{popov_2}
    Попов Г.Я. Точные решения некоторых краевых задач механики деформируемого твѐрдого тела. Одесса: Астропринт, 2013. 424 с.
    \bibitem{popov_3}
    Popov G. On the method of orthogonal polynomials in contact problems of the theory of elasticity. Journal of Applied Mathematics and Mechanics (1969). Volume 33, Issue 3, pp. 503-517
    \bibitem{gantmaher}
    Gantmakher F. R. (1998) The theory of matrices. AMS Chelsea Publishing, Providence, Rohde Island.
    \bibitem{ortogonal}
    Попов Г. Я., Реут В. В., Моісеєв М. Г., Вайсфельд Н. Д. Рівняння математичної фізики. Метод ортогональних многочленів. Одесса: Астропринт, 2010. 120 с.
    \bibitem{prudnikov}
    Прудников А.П.,Брычков Ю.А., Маричев О.И. Интегралы и ряды специальные функции. В 3 т. Т 1. Элементарные функции. 2-е издание, исправленное. М.: ФИЗМАТЛИТ, 2002. 632 с.
    \bibitem{pozhylenkov_1}
    D. Nerukh, O. Pozhylenkov, N. Vaysfeld (2019) Mixed plain boundary value problem of elasticity for a rectangular domain. 25-th International Conference Engineering Mechanics. 2019, May 13-16, Svratka, Czech Republic. p. 255
    \bibitem{pozhylenkov_2}
    O. V. Pozhylenkov (2019) The stress state of a rectangular elastic domain. Researches in Mathematics and Mechanics, Volume 24, Issue 2(34), pp. 88-96
    \bibitem{pozhylenkov_3}
    Пожиленков О. В. Вайсфельд Н. Д. (2019) Мішана крайова задача теорії пружності для прямокутної області. Математичні проблеми механіки неоднорідних структур, випуск 5, Львів, ст. 30-32
    \bibitem{conf_1}
    D. Nerukh, O. Pozhylenkov, N. Vaysfeld 25-th international conference «Engineering Mechanics 2019» // Czech Republic, Svratka, 2019
    \bibitem{conf_2}
    Пожиленков О. В., Вайсфельд Н. Д. Х Мiжнародна наукова конференцiя «Математичнi проблеми механiки неоднорiдних структур» // Львiв, 2019
    \bibitem{pozhylenkov_4}
    O. Pozhyenkov, N. Vaysfeld (2020) Stress state of a rectangular domain with the mixed boundary conditions. Procedia Structural Integrity, Volume 28, pp. 458-463
    \bibitem{conf_3}
    O. Pozhyenkov, N. Vaysfeld «1st Virtual European Conference on Fracture» // Italy, 2020
    \bibitem{pozhylenkov_5}
    O. Pozhyenkov, N. Vaysfeld (2021) Stress state of an elastic rectangular domain under steady load. Procedia Structural Integrity, Volume 33, pp. 385-390
    \bibitem{conf_4}
    O. Pozhyenkov, N. Vaysfeld «26th International Conference on Fracture and Structural Integrity» // Italy, Turin, 2021
    \bibitem{pozhylenkov_6}
    O. Pozhylenkov, N. Vaysfeld (2022) Dynamic mixed problem of elasticity for a rectangular domain. Recent trends in Wave Mechanics and Vibrations, pp. 211-218
    \bibitem{conf_5}
    O. Pozhyenkov, N. Vaysfeld «10th International Conference on Wave Mechanics and Vibrations» // Portugal, Lisbon, 2022
\end{thebibliography}

\addcontentsline{toc}{section}{\protect\numberline{}Додаток А ПОКРОКОВЕ ІНТЕГРУВАННЯ РІВНЯНЬ ЛАМЕ ЗА ЗМІННОЮ $x$}
\section*{\centering Додаток А}\label{ap_A}
\section*{\centering ПОКРОКОВЕ ІНТЕГРУВАННЯ РІВНЯНЬ ЛАМЕ ЗА ЗМІННОЮ $x$}
Граничные условия
\begin{align*}
    &u(x,y) |_{x=0} = 0, \quad \tau_{xy}(x,y) |_{x=0} =0, \quad 0 \le y \le b, \\
    &u(x,y) |_{x=a} = 0, \quad v(x,y) |_{x=a} = 0, \quad 0 \le y \le b.
\end{align*}

Рассмотрим
\begin{align*}
    &\int_{0}^{a} \frac{\partial^2 u(x,y)}{\partial x^2} sin(\alpha_n x)dx = \frac{\partial u(x,y)}{\partial x} sin(\alpha_n x) |_{x=0}^{x=a} - \alpha_n \int_{0}^{a} \frac{\partial u(x,y)}{\partial x} cos(\alpha_n x)dx = \\
    &= \frac{\partial u(x,y)}{\partial x} sin(\alpha_n x) |_{x=0}^{x=a} - \alpha_n \left( u(x,y) cos(\alpha_n x) |_{x=0}^{x=a} + \alpha_n \int_{0}^{a} u(x,y) sin(\alpha_n x) dx \right) = \\
    &=\frac{\partial u(x,y)}{\partial x}|_{x=a} sin(\alpha_n a) -\alpha_n^2 u_n(y)
\end{align*}
использованные граниычные условия
\begin{equation*}
    u(x,y)|_{x=0} = 0
\end{equation*}

Розглянемо
\begin{align*}
    &\int_{0}^{a} \frac{\partial^2 u(x,y)}{\partial y^2} sin(\alpha_n x)dx = \frac{\partial^2}{\partial y^2} \int_{0}^{a} u(x,y) sin(\alpha_n x)dx = u_n^{''}(y)
\end{align*}

Рассмотрим
\begin{align*}
    &\int_{0}^{a} \frac{\partial^2 v(x,y)}{\partial x \partial y} sin(\alpha_n x) dx = \frac{\partial v(x,y)}{\partial y} sin(\alpha_n x) |_{x=0}^{x=a} - \alpha_n \int_{0}^{a} \frac{\partial v(x,y)}{\partial y} cos(\alpha_n x) dx = \\
    &= -\alpha_n \frac{\partial}{\partial y} \int_{0}^{a} v(x,y) cos(\alpha_n x) dx = -\alpha_n v_n^{'}(y)
\end{align*}
использованные граниычные условия
\begin{equation*}
    \frac{\partial v(x,y)}{\partial y}|_{x=a} = 0
\end{equation*}

Рассмотрим
\begin{align*}
    &\int_{0}^{a} \frac{\partial^2 v(x,y)}{\partial x^2} cos(\alpha_n x)dx = \frac{\partial v(x,y)}{\partial x} cos(\alpha_n x) |_{x=0}^{x=a} + \alpha_n \int_{0}^{a} \frac{\partial v(x,y)}{\partial x} sin(\alpha_n x) dx = \\
    &=\frac{\partial v(x,y)}{\partial x} cos(\alpha_n x) |_{x=0}^{x=a} + \alpha_n \left(v(x,y) sin(\alpha_n x)|_{x=0}^{x=a} - \alpha_n \int_{0}^{a} v(x,y) cos(\alpha_n x) dx  \right) = \\
    &= -\alpha_n^2 v_n(y)
\end{align*}
использованные граниычные условия
\begin{equation*}
    \frac{\partial v(x,y)}{\partial x}|_{x=0} = 0, \quad v(x, y)|_{x=a} = 0
\end{equation*}

Рассмотрим
\begin{align*}
    &\int_{0}^{a} \frac{\partial^2 v(x,y)}{\partial y^2} cos(\alpha_n x)dx = \frac{\partial^2}{\partial y^2} \int_{0}^{a} v(x,y) cos(\alpha_n x)dx = v_n^{''}(y)
\end{align*}

Рассмотрим
\begin{align*}
    &\int_{0}^{a} \frac{\partial^2 u(x,y)}{\partial y \partial x} cos(\alpha_n x)dx = \frac{\partial u(x,y)}{\partial y} cos(\alpha_n x) |_{x=0}^{x=a} + \alpha_n \int_{0}^{a} \frac{\partial u(x,y)}{\partial y} sin(\alpha_n x) dx = \\
    &=\frac{\partial u(x,y)}{\partial y} cos(\alpha_n x) |_{x=0}^{x=a} + \alpha_n \frac{\partial}{\partial y} \int_{0}^{a} u(x,y) sin(\alpha_n x) dx = \alpha_n u_n^{'}(y)
\end{align*}
использованные граниычные условия
\begin{equation*}
    \frac{\partial u(x,y)}{\partial y}|_{x=0} = 0
\end{equation*}

В итоге были использованы след граничные условия:
\begin{align*}
    &u(x,y)|_{x=0} = 0, \\
    &\frac{\partial u(x,y)}{\partial y}|_{x=0} = 0, \\
    &\frac{\partial v(x,y)}{\partial x}|_{x=0} = 0, \\
    &\tau_{xy}(x,y) |_{x=0} =0, \\
    &v(x, y)|_{x=a} = 0, \\
    &\frac{\partial v(x,y)}{\partial y}|_{x=a} = 0
\end{align*}

Не использованные
\begin{align*}
    &u(x,y)|_{x=a} = 0, \quad boundary \quad condition \\
    &\frac{\partial u(x,y)}{\partial x}|_{x=a} = f(y), \\
    &v(x,y)|_{x=0} = 0, \\
    &\frac{\partial v(x,y)}{\partial y}|_{x=0} = 0, \\
\end{align*}


\addcontentsline{toc}{section}{\protect\numberline{}Додаток B ЗНАХОДЖЕННЯ КОРЕНІВ РІВНЯННЯ $det[M(s)]=0$}
\section*{\centering Додаток B}\label{ap_B}
\section*{\centering ЗНАХОДЖЕННЯ КОРЕНІВ РІВНЯННЯ $det[M(s)]=0$}
Знайдемо корені $det[M(s)]=0$
\begin{align*}
    &det[M(s)] = 
    \begin{vmatrix}
        s^2 - \alpha_n^2 - \alpha_n^2\mu_0 + \frac{\omega^2}{c_1^2} & -\alpha_n \mu_0 s \\
        \alpha_n \mu_0 s & s^2 (1 + \mu_0) -\alpha_n^2 + \frac{\omega^2}{c_2^2}
     \end{vmatrix} = \\
    &= (s^2 (1 + \mu_0) -\alpha_n^2 + \frac{\omega^2}{c_2^2})(s^2 - \alpha_n^2 - \alpha_n^2\mu_0 + \frac{\omega^2}{c_1^2}) + (\alpha_n \mu_0 s)^2 = \\
    &= s^4 + s^4 \mu_0 - s^2 \alpha_n^2 + s^2 \frac{\omega^2}{c_2^2} - s^2 \alpha_n^2 - s^2 \alpha_n^2 \mu_0 + \alpha_n^4 - \alpha_n^2 \frac{\omega^2}{c_2^2} - s^2 \alpha_n^2 \mu_0 - \\
    &- s^2 \alpha_n^2 \mu_0 + \alpha_n^4 \mu_0 - \alpha_n^2 \frac{\omega^2}{c_2^2} + s^2 \frac{\omega^2}{c_1^2} + s^2 \mu_0 \frac{\omega^2}{c_1^2} - \alpha_n^2 \frac{\omega^2}{c_1^2} + \frac{\omega^4}{c_1^2 c_2^2} + s^2 \alpha_n^2 \mu_0^2 = \\
    &=(1 + \mu_0) s^4 + (-2 \alpha_n^2 - 2 \alpha_n^2 \mu_0 + \frac{\omega^2}{c_2^2} + \frac{\omega^2}{c_1^2} + \mu_0 \frac{\omega^2}{c_1^2}) s^2 + (\alpha_n^4 - \alpha_n^2 \frac{\omega^2}{c_2^2} + \\ 
    &+ \alpha_n^4 \mu_0 - \alpha_n^2 \mu_0 \frac{\omega^2}{c_2^2} - \alpha_n^2 \frac{\omega^2}{c_1^2} + \frac{\omega^4}{c_1^2 c_2^2})
\end{align*}
Введемо наступні позначення:
\begin{align*}
    &a_1 = -2 \alpha_n^2 - 2 \alpha_n^2 \mu_0 + \frac{\omega^2}{c_2^2} + \frac{\omega^2}{c_1^2} + \mu_0 \frac{\omega^2}{c_1^2} \\
    &a_2 = \alpha_n^4 - \alpha_n^2 \frac{\omega^2}{c_2^2} + \alpha_n^4 \mu_0 - \alpha_n^2 \mu_0 \frac{\omega^2}{c_2^2} - \alpha_n^2 \frac{\omega^2}{c_1^2} + \frac{\omega^4}{c_1^2 c_2^2}
\end{align*}
Враховучи введені позначення отримаємо наступне рівняння:
\begin{equation*}
    (1 + \mu_0) s^4 + a_1 s^2 + a_2 = 0
\end{equation*}
Таким чином отримаємо наступні корені рівняння:
\begin{align*}
    &s_1 = \sqrt{\frac{ -a_1 + \sqrt{a_1^2 - 4(1 + \mu_0)a_2}}{2 (1 + \mu_0)}} \\
    &s_2 = -\sqrt{\frac{ -a_1 + \sqrt{a_1^2 - 4(1 + \mu_0)a_2}}{2 (1 + \mu_0)}} \\
    &s_3 = \sqrt{\frac{ -a_1 - \sqrt{a_1^2 - 4(1 + \mu_0)a_2}}{2 (1 + \mu_0)}} \\
    &s_4 = -\sqrt{\frac{ -a_1 - \sqrt{a_1^2 - 4(1 + \mu_0)a_2}}{2 (1 + \mu_0)}}
\end{align*}
У випадку статичної задачі коли $\omega = 0$ отримаємо наступні рівняння:
\begin{equation*}
    (1 + \mu_0) (s^4 - 2 \alpha_n^2 s^2 + \alpha_n^4) = 0
\end{equation*}
Таким чином отримаємо наступні корені
\begin{align*}
    &s_{1,2} = \alpha_n \\
    &s_{3, 4} = -\alpha_n
\end{align*}


\addcontentsline{toc}{section}{\protect\numberline{}Додаток C ЗНАХОДЖЕННЯ ФУНДАМЕНТАЛЬНИХ БАЗИСНИХ МАТРИЦЬ $\Psi_i(y)$, $i=\overline{0,1}$}
\section*{\centering Додаток C}\label{ap_C}
\section*{\centering ЗНАХОДЖЕННЯ ФУНДАМЕНТАЛЬНИХ БАЗИСНИХ МАТРИЦЬ $\Psi_i(y)$, $i=\overline{0,1}$}
Для знаходження матриць коєфіцієтів $C_k^i$ для фундаентальних базісних матриць $\Psi_i(y)$, $i=\overline{0,1}$, $k=\overline{1,2}$.
Використовуючи граничні умови \eqref{transf_bound_mat_gen} шукати їх будем з наступних умов:
\begin{align*}
    &U_0\left[ \Psi_0(y) \right] = I, \quad U_1\left[ \Psi_0(y) \right] = 0 \\
    &U_0\left[ \Psi_1(y) \right] = 0, \quad U_1\left[ \Psi_1(y) \right] = I, \quad I = \begin{pmatrix}
        1 && 0 \\
        0 && 1
    \end{pmatrix} \\
    &U_0\left[ \Psi_i(y) \right] =  \begin{pmatrix}
        1 & 0 \\
        0 & 2G + \lambda
    \end{pmatrix} * \Psi_i^{'}(b) + \begin{pmatrix}
        0 & -\alpha_n \\
        \alpha_n \lambda & 0
    \end{pmatrix} * \Psi_i(b) \\
    &U_1\left[ \Psi_i(y) \right] =  \begin{pmatrix}
        1 & 0 \\
        0 & 0
    \end{pmatrix} * \Psi_i^{'}(0) + \begin{pmatrix}
        0 & -\alpha_n \\
        0 & 1
    \end{pmatrix} * \Psi_i(0) \\
\end{align*}

Введемо наступні позначення:
\begin{equation*}
    C_1^i = \begin{pmatrix}
        d_1^i && d_2^i \\
        d_3^i && d_4^i
    \end{pmatrix}, \quad
    C_2^i = \begin{pmatrix}
        f_1^i && f_2^i \\
        f_3^i && f_4^i
    \end{pmatrix},
\end{equation*}
\begin{align*}
    &x_1 =  s_1^2 (1 + \mu_0) -\alpha_n^2 + \frac{\omega^2}{c_2^2}, \quad x_2 = s_1^2 - \alpha_n^2 - \alpha_n^2\mu_0 + \frac{\omega^2}{c_1^2} \\
    &x_3 =  s_2^2 (1 + \mu_0) -\alpha_n^2 + \frac{\omega^2}{c_2^2}, \quad x_4 = s_2^2 - \alpha_n^2 - \alpha_n^2\mu_0 + \frac{\omega^2}{c_1^2} \\
    &x_5 =  s_1 \alpha_n \mu_0, \quad x_6 = s_2 \alpha_n \mu_0 \\
    &y_1 = 2s_1 (s_1^2 - s_2^2), \quad y_2 = 2s_2 (s_2^2 - s_1^2)
\end{align*}

Враховуючи їх представлення \eqref{psi_gen} випишем вигляд $\Psi_i(y)$:
\begin{align*}
    &\Psi_i(y) = \frac{1}{y_1} \begin{pmatrix}
        x_1(e^{y s_1} - e^{-y s_1})d_1^i + x_5(e^{y s_1} + e^{-y s_1})d_3^i && x_1(e^{y s_1} - e^{-y s_1})d_2^i + x_5(e^{y s_1} + e^{-y s_1})d_4^i \\
        -x_5(e^{y s_1} + e^{-y s_1})d_1^i + x_2(e^{y s_1} - e^{-y s_1})d_3^i && -x_5(e^{y s_1} + e^{-y s_1})d_2^i + x_2(e^{y s_1} - e^{-y s_1})d_4^i
    \end{pmatrix} + \\
    &+ \frac{1}{y_2} \begin{pmatrix}
        x_3(e^{y s_2} - e^{-y s_2})f_1^i + x_6(e^{y s_2} + e^{-y s_2})f_3^i && x_3(e^{y s_2} - e^{-y s_2})f_2^i + x_6(e^{y s_2} + e^{-y s_2})f_4^i \\
        -x_6(e^{y s_2} + e^{-y s_2})f_1^i + x_4(e^{y s_2} - e^{-y s_2})f_3^i && -x_6(e^{y s_2} + e^{-y s_2})f_2^i + x_4(e^{y s_2} - e^{-y s_2})f_4^i
    \end{pmatrix}
\end{align*}
\begin{align*}
    &\Psi_i^{'}(y) = \frac{s_1}{y_1} \begin{pmatrix}
        x_1(e^{y s_1} - e^{-y s_1})d_1^i + x_5(e^{y s_1} + e^{-y s_1})d_3^i && x_1(e^{y s_1} - e^{-y s_1})d_2^i + x_5(e^{y s_1} + e^{-y s_1})d_4^i \\
        -x_5(e^{y s_1} + e^{-y s_1})d_1^i + x_2(e^{y s_1} - e^{-y s_1})d_3^i && -x_5(e^{y s_1} + e^{-y s_1})d_2^i + x_2(e^{y s_1} - e^{-y s_1})d_4^i
    \end{pmatrix} + \\
    &+ \frac{s_2}{y_2} \begin{pmatrix}
        x_3(e^{y s_2} - e^{-y s_2})f_1^i + x_6(e^{y s_2} + e^{-y s_2})f_3^i && x_3(e^{y s_2} - e^{-y s_2})f_2^i + x_6(e^{y s_2} + e^{-y s_2})f_4^i \\
        -x_6(e^{y s_2} + e^{-y s_2})f_1^i + x_4(e^{y s_2} - e^{-y s_2})f_3^i && -x_6(e^{y s_2} + e^{-y s_2})f_2^i + x_4(e^{y s_2} - e^{-y s_2})f_4^i
    \end{pmatrix}
\end{align*}

Розглянемо $U_0\left[ \Psi_i(y) \right]$:
\begin{align*}
    &U_0\left[ \Psi_i(y) \right]_{1,1} = \frac{s_1}{y_1} \left(x_1(e^{b s_1} - e^{-b s_1})d_1^i + x_5(e^{b s_1} + e^{-b s_1})d_3^i \right) + \\
    &+ \frac{s_2}{y_2} \left( x_3(e^{b s_2} - e^{-b s_2})f_1^i + x_6(e^{b s_2} + e^{-b s_2})f_3^i \right) 
    + \frac{\alpha_n}{y_1} \left( x_5(e^{b s_1} + e^{-b s_1})d_1^i - x_2(e^{b s_1} - e^{-b s_1})d_3^i \right) + \\
    & + \frac{\alpha_n}{y_2} \left( x_6(e^{b s_2} + e^{-b s_2})f_1^i + x_4(e^{b s_2} - e^{-b s_2})f_3^i \right)
\end{align*}
\begin{align*}
    &U_0\left[ \Psi_i(y) \right]_{1,2} = \frac{s_1}{y_1} \left(x_1(e^{b s_1} - e^{-b s_1})d_2^i + x_5(e^{b s_1} + e^{-b s_1})d_4^i \right) + \\
    &+ \frac{s_2}{y_2} \left( x_3(e^{b s_2} - e^{-b s_2})f_2^i + x_6(e^{b s_2} + e^{-b s_2})f_4^i \right) 
    + \frac{\alpha_n}{y_1} \left( x_5(e^{b s_1} + e^{-b s_1})d_2^i - x_2(e^{b s_1} - e^{-b s_1})d_4^i \right) + \\
    & + \frac{\alpha_n}{y_2} \left( x_6(e^{b s_2} + e^{-b s_2})f_2^i + x_4(e^{b s_2} - e^{-b s_2})f_4^i \right)
\end{align*}
\begin{align*}
    &U_0\left[ \Psi_i(y) \right]_{2,1} = \frac{s_1 (2G + \lambda)}{y_1} \left( -x_5(e^{b s_1} + e^{-b s_1})d_1^i + x_2(e^{b s_1} - e^{-b s_1})d_3^i \right) + \\
    &+ \frac{s_2 (2G + \lambda)}{y_2} \left( -x_6(e^{b s_2} + e^{-b s_2})f_1^i + x_4(e^{b s_2} - e^{-b s_2})f_3^i \right) 
    + \frac{\alpha_n\lambda}{y_1} \left( x_1(e^{b s_1} - e^{-b s_1})d_1^i + x_5(e^{b s_1} + e^{-b s_1})d_3^i \right) + \\
    & + \frac{\alpha_n\lambda}{y_2} \left( x_3(e^{b s_2} - e^{-b s_2})f_1^i + x_6(e^{b s_2} + e^{-b s_2})f_3^i \right)
\end{align*}
\begin{align*}
    &U_0\left[ \Psi_i(y) \right]_{2,2} = \frac{s_1 (2G + \lambda)}{y_1} \left( -x_5(e^{b s_1} + e^{-b s_1})d_2^i + x_2(e^{b s_1} - e^{-b s_1})d_4^i \right) + \\
    &+ \frac{s_2 (2G + \lambda)}{y_2} \left( -x_6(e^{b s_2} + e^{-b s_2})f_2^i + x_4(e^{b s_2} - e^{-b s_2})f_4^i \right) 
    + \frac{\alpha_n\lambda}{y_1} \left( x_1(e^{b s_1} - e^{-b s_1})d_2^i + x_5(e^{b s_1} + e^{-b s_1})d_4^i \right) + \\
    & + \frac{\alpha_n\lambda}{y_2} \left( x_3(e^{b s_2} - e^{-b s_2})f_2^i + x_6(e^{b s_2} + e^{-b s_2})f_4^i \right)
\end{align*}

Розглянемо $U_1\left[ \Psi_i(y) \right]$:
\begin{align*}
    &U_1\left[ \Psi_i(y) \right]_{1,1} = \frac{s_1}{y_1} \left( 2 x_1 d_1^i \right) + \frac{s_2}{y_2} \left( 2 x_3 f_1^i \right) 
    + \frac{\alpha_n}{y_1} \left(2 x_5 d_1^i \right) + \frac{\alpha_n}{y_2} \left(2 x_6 f_1^i \right) \\
    &U_1\left[ \Psi_i(y) \right]_{1,2} = \frac{s_1}{y_1} \left( 2 x_1 d_2^i \right) + \frac{s_2}{y_2} \left( 2 x_3 f_2^i \right) 
    + \frac{\alpha_n}{y_1} \left(2 x_5 d_2^i \right) + \frac{\alpha_n}{y_2} \left(2 x_6 f_2^i \right) \\
    &U_1\left[ \Psi_i(y) \right]_{2,1} = \frac{1}{y_1} \left( -2 x_5 d_1^i \right) + \frac{1}{y_2} \left( -2 x_6 f_1^i \right) \\
    &U_1\left[ \Psi_i(y) \right]_{2,2} = \frac{1}{y_1} \left( -2 x_5 d_2^i \right) + \frac{1}{y_2} \left( -2 x_6 f_2^i \right)
\end{align*}

Введемо наступні позначення:
\begin{align*}
    &z_1 = \frac{1}{y_1}\left( e^{b s_1} + e^{-b s_1} \right) \left( s_1 x_1 + \alpha_n x_5 \right), \quad
    z_2 = \frac{1}{y_1}\left( e^{b s_1} - e^{-b s_1} \right) \left( s_1 x_5 - \alpha_n x_2 \right), \\
    &z_3 = \frac{1}{y_2}\left( e^{b s_2} + e^{-b s_2} \right) \left( s_2 x_3 + \alpha_n x_6 \right), \quad
    z_4 = \frac{1}{y_2}\left( e^{b s_2} - e^{-b s_2} \right) \left( s_2 x_6 - \alpha_n x_4 \right), \\
    &z_5 = \frac{1}{y_1}\left( e^{b s_1} - e^{-b s_1} \right) \left( -s_1 (2G + \lambda) x_5 + \alpha_n \lambda x_3 \right), \quad
    z_6 = \frac{1}{y_1}\left( e^{b s_1} + e^{-b s_1} \right) \left( s_1 (2G + \lambda) x_2 + \alpha_n \lambda x_5 \right), \\
    &z_7 = \frac{1}{y_2}\left( e^{b s_2} - e^{-b s_2} \right) \left( -s_2 (2G + \lambda) x_6 + \alpha_n \lambda x_3 \right), \quad
    z_8 = \frac{1}{y_2}\left( e^{b s_2} + e^{-b s_2} \right) \left( s_2 (2G + \lambda) x_4 + \alpha_n \lambda x_6 \right), \\
    &z_9 = \frac{1}{y_1} \left( 2 s_1 x_1 + 2 \alpha_n x_5 \right), \quad
    z_{10} = \frac{1}{y_2} \left( 2 s_2 x_3 + 2 \alpha_n x_6 \right), \\
    &z_{11} = -\frac{2}{y_1} x_5, \quad z_{12} = -\frac{2}{y_2} x_6
\end{align*}

Враховучи останнє випишем системи відностно невідомих коєфіцієнтів $d_k^i$, $f_k^i$, $i=\overline{0,1}$, $k=\overline{1,4}$
\begin{equation*}
    \begin{cases}
        z_1 d_1^0 + z_2 d_3^0 + z_3 f_1^0 + z_4 f_3^0 = 1 \\
        z_5 d_1^0 + z_6 d_3^0 + z_7 f_1^0 + z_8 f_3^0 = 0 \\
        z_9 d_1^0 + z_{10} f_1^0 = 0 \\
        z_{11} d_1^0 + z_{12} f_1^0 = 0
    \end{cases}, \quad
    \begin{cases}
        z_1 d_2^0 + z_2 d_4^0 + z_3 f_2^0 + z_4 f_4^0 = 0 \\
        z_5 d_2^0 + z_6 d_4^0 + z_7 f_2^0 + z_8 f_4^0 = 1 \\
        z_9 d_2^0 + z_{10} f_2^0 = 0 \\
        z_{11} d_2^0 + z_{12} f_2^0 = 0
    \end{cases}
\end{equation*}
\begin{equation*}
    \begin{cases}
        z_1 d_1^1 + z_2 d_3^1 + z_3 f_1^1 + z_4 f_3^1 = 0 \\
        z_5 d_1^1 + z_6 d_3^1 + z_7 f_1^1 + z_8 f_3^1 = 0 \\
        z_9 d_1^1 + z_{10} f_1^1 = 1 \\
        z_{11} d_1^1 + z_{12} f_1^1 = 0
    \end{cases}, \quad
    \begin{cases}
        z_1 d_2^1 + z_2 d_4^1 + z_3 f_2^1 + z_4 f_4^1 = 0 \\
        z_5 d_2^1 + z_6 d_4^1 + z_7 f_2^1 + z_8 f_4^1 = 0 \\
        z_9 d_2^1 + z_{10} f_2^1 = 0 \\
        z_{11} d_2^1 + z_{12} f_2^1 = 1
    \end{cases}
\end{equation*}

\addcontentsline{toc}{section}{\protect\numberline{}Додаток D ЗНАХОДЖЕННЯ ФУНКЦІЇ $v_0(y)$ НЕОДНОРІДНОЇ ЗАДАЧІ}
\section*{\centering Додаток D}\label{ap_D}
\section*{\centering ЗНАХОДЖЕННЯ ФУНКЦІЇ $v_0(y)$ НЕОДНОРІДНОЇ ЗАДАЧІ}
Знайдем $v_0(y)$ розглянувши задачу у просторі трансформант \eqref{transf_gen}, \eqref{transf_bound_gen} при $n=0$, $\alpha_n = 0$.
Отримаємо наступну задачу відносно $v_0(y)$:
\begin{align*}
    &v_0^{''}(y) + \frac{\omega^2}{c_2^2(1+\mu_0) }v_0(y) = \frac{f(y)}{1+\mu_0}
\end{align*}
Де $f(y)=(\frac{\alpha_2}{\beta_2}\chi_4(y) cos(\alpha_n a) - \frac{\alpha_1}{\beta_1}\chi_2(y)) - \frac{\mu_0}{(1+\mu_0)} (\chi_3^{'}(y) cos(\alpha_n a) -\chi_1^{'}(y))$.

Та граничні умови:
\begin{equation*}
    (2G + \lambda)v_0^{'}(b) = -p_0, \quad v_0(0) = 0, \quad p_0 = \int_{0}^{a}p(x)dx
\end{equation*}
Спочатку знайдем фундаментальну базисну систему розв'язків задачі $\psi_0(y)$, $\psi_1(y)$:
\begin{equation*}
    \psi_i^{''}(y) + \frac{\omega^2}{c_2^2(1+\mu_0) }\psi_i(y) = 0, i=\overline{0,1} \\
\end{equation*}
\begin{equation*}
    \begin{cases}
        \psi_0(0) = 1 \\
        \psi_0^{'}(b) = 0
    \end{cases}, \quad
    \begin{cases}
        \psi_1(0) = 0 \\
        \psi_1^{'}(b) = 1
    \end{cases}
\end{equation*}
Розв'язок однорідної задачі відносно $\psi_i(y)$ має вигляд:
\begin{equation}
    \psi_i(y) = c_1^i cos\left( \frac{\omega}{c_2 \sqrt{1 + \mu_0}} y \right) + c_2^i sin\left( \frac{\omega}{c_2 \sqrt{1 + \mu_0}} y \right)
\end{equation}
Враховучи граничні умови отримаємо остаточний вигляд $\psi_0(y)$, $\psi_1(y)$:
\begin{align*}
    \begin{cases}
        \psi_0(y) = cos\left( \frac{\omega}{c_2 \sqrt{1 + \mu_0}} y \right) +  tg\left( \frac{\omega}{c_2 \sqrt{1 + \mu_0}} b \right) sin\left( \frac{\omega}{c_2 \sqrt{1 + \mu_0}} y \right) \\
        \psi_1(y) = \frac{c_2 (1 + \mu_0)}{\omega cos\left( \frac{\omega}{c_2 \sqrt{1 + \mu_0}} b \right)} sin\left( \frac{\omega}{c_2 \sqrt{1 + \mu_0}} y \right)
    \end{cases}
\end{align*}
Побудуємо тепер функцію Гріна задачі:
\begin{equation*}
    g(y, \xi) = \begin{cases}
        -a_1(\xi) \psi_1(y), 0 \le y < \xi \\
        a_0(\xi) \psi_0(y), \xi < y \le b
    \end{cases}
\end{equation*}
Де $a_0(\xi)$, $a_1(\xi)$ будуть знайдені з наступної системи
\begin{equation*}
    \begin{cases}
        a_0(\xi) \psi_0^{'}(\xi) + a_1(\xi) \psi_1^{'}(\xi) = 1 \\
        a_0(\xi) \psi_0^(\xi) + a_1(\xi) \psi_1(\xi) = 0
    \end{cases}
\end{equation*}
Таким чином остаточний розв'язок задачі відносно $v_0(y)$ буде мати наступний вигляд:
\begin{equation*}
    v_0(y) = \frac{1}{(1+\mu_0)} \int_{0}^{b}g(y,\xi) f(\xi) d\xi - \psi_0(y) \frac{p_0}{2G + \lambda}
\end{equation*}

У випадку статичної задачі коли $\omega = 0$ отримаємо наступну задачу відносно $v_0(y)$:
\begin{equation*}
    v_0^{''}(y) = \frac{f(y)}{1+\mu_0}
\end{equation*}
\begin{equation*}
    (2G + \lambda)v_0^{'}(b) = -p_0, \quad v_0(0) = 0, \quad p_0 = \int_{0}^{a}p(x)dx
\end{equation*}

\addcontentsline{toc}{section}{\protect\numberline{}Додаток E ЗНАХОДЖЕННЯ КОЄФІЦІЄНТІВ $c_i$, $i=\overline{1,4}$}
\section*{\centering Додаток E}\label{ap_E}
\section*{\centering ЗНАХОДЖЕННЯ КОЄФІЦІЄНТІВ $c_i$, $i=\overline{1,4}$}
Для знаходження коєфіцієтів $c_1$, $c_2$, $c_3$, $c_4$ спочатку
знайдем $Y_0(y) * \begin{pmatrix}c_1 \\ c_2\end{pmatrix}$ та $Y_1(y) * \begin{pmatrix}c_3 \\ c_4\end{pmatrix}$.
\begin{align*}
    &Y_0(y) * \begin{pmatrix}c_1 \\ c_2\end{pmatrix} = \frac{e^{\alpha_n y}}{4\alpha_n} \begin{pmatrix}
        \alpha_n \mu_0 y + 2 + \mu_0 & \alpha_n \mu_0 y \\
        -\alpha_n \mu_0 y & -\alpha_n \mu_0 y + 2 + \mu_0
        \end{pmatrix} * \begin{pmatrix}c_1 \\ c_2\end{pmatrix} = \\
    &=\frac{e^{\alpha_n y}}{4\alpha_n} \begin{pmatrix}
        c_1(\alpha_n \mu_0 y + 2 + \mu_0) + c_2(\alpha_n \mu_0 y) \\
        c_1(-\alpha_n \mu_0 y) + c_2(-\alpha_n \mu_0 y + 2 + \mu_0)
        \end{pmatrix}
\end{align*}
\begin{align*}
    &Y_1(y) * \begin{pmatrix}c_3 \\ c_4\end{pmatrix} = \frac{e^{-\alpha_n y}}{4\alpha_n} \begin{pmatrix}
        \alpha_n \mu_0 y - 2 - \mu_0 & -\alpha_n \mu_0 y \\
        \alpha_n \mu_0 y & -\alpha_n \mu_0 y - 2 - \mu_0
        \end{pmatrix} * \begin{pmatrix}c_3 \\ c_4\end{pmatrix} = \\
    &=\frac{e^{-\alpha_n y}}{4\alpha_n} \begin{pmatrix}
        c_3(\alpha_n \mu_0 y - 2 - \mu_0) + c_4(-\alpha_n \mu_0 y) \\
        c_3(\alpha_n \mu_0 y) + c_4(-\alpha_n \mu_0 y - 2 - \mu_0)
        \end{pmatrix}
\end{align*}
Введемо позначення $c = \frac{1}{4\alpha_n (1 + \mu_0)}$. \newline
Запишем тепер $Z_n(y)$:
\begin{align*}
    &Z_n(y) = c
    \begin{pmatrix}
        c_1 e^{\alpha_n y} (\alpha_n \mu_0 y + 2 + \mu_0) + c_2 e^{\alpha_n y} (\alpha_n \mu_0 y) + 
        \\ + c_3 e^{-\alpha_n y} (\alpha_n \mu_0 y - 2 - \mu_0) + c_4 e^{-\alpha_n y} (-\alpha_n \mu_0 y) \\
        \\
        c_1 e^{\alpha_n y} (-\alpha_n \mu_0 y) + c_2 e^{\alpha_n y} (-\alpha_n \mu_0 y + 2 + \mu_0) + 
        \\ + c_3 e^{-\alpha_n y} (\alpha_n \mu_0 y) + c_4 e^{-\alpha_n y} (-\alpha_n \mu_0 y - 2 - \mu_0)
    \end{pmatrix}
\end{align*}
Тепер $Z_n^{'}(y)$:
\begin{align*}
    &Z_n^{'}(y) = c
    \begin{pmatrix}
        c_1 e^{\alpha_n y} (\alpha_n^2 \mu_0 y + 2 \alpha_n + 2 \alpha_n \mu_0) + c_2 e^{\alpha_n y} (\alpha_n^2 \mu_0 y + \alpha_n \mu_0) + 
        \\ + c_3 e^{-\alpha_n y} (-\alpha_n^2 \mu_0 y + 2 \alpha_n + 2 \alpha_n \mu_0) + c_4 e^{-\alpha_n y} (\alpha_n^2 \mu_0 y - \alpha_n \mu_0) \\
        \\
        c_1 e^{\alpha_n y} (-\alpha_n \mu_0 y) + c_2 e^{\alpha_n y} (-\alpha_n \mu_0 y + 2 + \mu_0) + 
        \\ + c_3 e^{-\alpha_n y} (\alpha_n \mu_0 y) + c_4 e^{-\alpha_n y} (-\alpha_n \mu_0 y - 2 - \mu_0)
    \end{pmatrix}
\end{align*}
Тепер використаєм граничні умови (\ref{transf_bound_mat_1}) та побудуєм алгебричну систему відносно коєфіцієнтів.

Використаєм $U_0\left[ Z_n(y) \right]$:
\begin{eqnarray*}
    E_0 * Z_n^{'}(b) + F_0 * Z_n(b) = D_0 \Leftrightarrow
\end{eqnarray*}
\begin{equation*}
    \begin{pmatrix}
        1 & 0 \\
        0 & 2G + \lambda
    \end{pmatrix} * Z_n^{'}(b) + \begin{pmatrix}
        0 & -\alpha_n \\
        \alpha_n \lambda & 0
    \end{pmatrix} * Z_n(b) = \begin{pmatrix}
        0 \\
        -p_n
    \end{pmatrix}
\end{equation*}
Отримаємо перші 2 рівняння системи:
\begin{equation*}
    \begin{cases}
        c_1 e^{\alpha_n b} (\alpha_n^2 \mu_0 b + \alpha_n \mu_0 + \alpha_n) + c_2 e^{\alpha_n b} (\alpha_n^2 \mu_0 b - \alpha_n) + \\
        + c_3 e^{-\alpha_n b} (-\alpha_n^2 \mu_0 b + \alpha_n + \alpha_n \mu_0) + c_4 e^{-\alpha_n b} (\alpha_n^2 \mu_0 b + \alpha_n) = 0 \\
        \\
        c_1 e^{\alpha_n b} (-2 G \alpha_n^2 \mu_0 b - 2 G \alpha_n \mu_0 + 2 \lambda \alpha_n) + c_2 e^{\alpha_n b} (-2G \alpha_n^2 \mu_0 b + \\
        + (2G + \lambda) 2 \alpha_n) + c_3  e^{-\alpha_n b} (-2 G \alpha_n^2 \mu_0 b + 2G \alpha_n \mu_0 - 2\lambda \alpha_n) + \\ 
        + c_4 e^{-\alpha_n b} (2G \alpha_n^2 \mu_0 b + (2G + \lambda) 2 \alpha_n) = -c p_n
    \end{cases}
\end{equation*}

Використаєм $U_1\left[ Z_n(y) \right]$:
\begin{eqnarray*}
    E_1 * Z_n^{'}(0) + F_1 * Z_n(0) = D_1 \Leftrightarrow
\end{eqnarray*}
\begin{equation*}
    \begin{pmatrix}
        1 & 0 \\
        0 & 0
    \end{pmatrix} * Z_n^{'}(0) + \begin{pmatrix}
        0 & -\alpha_n \\
        0 & 1
    \end{pmatrix} * Z_n(0) = \begin{pmatrix}
        0 \\
        0
    \end{pmatrix}
\end{equation*}
Отримаємо другі 2 рівняння системи:
\begin{equation*}
    \begin{cases}
        c_1 (\alpha_n + \alpha_n \mu_0) + c_2 (-\alpha_n) + c_3 (\alpha_n + \alpha_n \mu_0) + c_4 (\alpha_n) = 0 \\
        \\
        c_2 (2 + \mu_0) + c_4 (-2 - \mu_0) = 0
    \end{cases}
\end{equation*}
Звідси видно, що $c_3 = -c_1$, $c_4 = c_2$.
Введемо наступні позначення:
\begin{align*}
    &a_1 = e^{\alpha_n b} (\alpha_n^2 \mu_0 b + \alpha_n \mu_0 + \alpha_n) - e^{-\alpha_n b} (-\alpha_n^2 \mu_0 b + \alpha_n + \alpha_n \mu_0), \\
    &a_2 = e^{\alpha_n b} (\alpha_n^2 \mu_0 b - \alpha_n) + e^{-\alpha_n b} (\alpha_n^2 \mu_0 b + \alpha_n), \\
    &a_3 = e^{\alpha_n b} (-2 G \alpha_n^2 \mu_0 b - 2 G \alpha_n \mu_0 + 2 \lambda \alpha_n) - \\
    &\quad - e^{-\alpha_n b} (-2 G \alpha_n^2 \mu_0 b + 2G \alpha_n \mu_0 - 2\lambda \alpha_n) \\
    &a_4 = e^{\alpha_n b} (-2G \alpha_n^2 \mu_0 b + (2G + \lambda) 2 \alpha_n) + \\
    &\quad + e^{-\alpha_n b} (2G \alpha_n^2 \mu_0 b + (2G + \lambda) 2 \alpha_n)
\end{align*}
Враховуючи останнє отримаємо:
\begin{equation*}
    \begin{cases}
        c_3 = -c_1 \\
        c_4 = c_2 \\
        c_1 a_1 + c_2 a_2 = 0 \\
        c_1 a_3 + c_2 a_4 = -c p_n
    \end{cases} \Leftrightarrow, \quad 
    \begin{cases}
        c_3 = -c_1 \\
        c_4 = c_2 \\
        c_1 = - c_2 \frac{a_2}{a_1} \\
        c_2(a_4 a_1 - a_2 a_3) = -c p_n a_1
    \end{cases} \Leftrightarrow
\end{equation*}
\begin{equation*}
    \begin{cases}
        c_1 = c p_n \frac{a_2}{(a_4 a_1 - a_2 a_3)} \\
        c_2 = - c p_n \frac{a_1}{(a_4 a_1 - a_2 a_3)} \\
        c_3 = -c p_n \frac{a_2}{(a_4 a_1 - a_2 a_3)} \\
        c_4 = - c p_n \frac{a_1}{(a_4 a_1 - a_2 a_3)}
    \end{cases}
\end{equation*}

\end{document}