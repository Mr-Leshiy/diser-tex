\textit{Пожиленков О. В.}
Плоскі мішані задачі теорії пружності для прямокутної області. - Кваліфікаційна наукова праця на правах рукопису.

Дисертація на здобуття наукового доктора філософії за спеціальністю 113 «Прикладна математика» (11 – Математика та статистика). – Одеський національний університет імені І. І. Мечникова, Одеса, 2023.

Була розглянута плоска мішана задача теорії пружності для скінченної прямокутної області, яка піддається впливу різних навантажень.
Шляхом застосування інтегрального півнескінченного sin- та cos-перетворення Фур'є вихідна задача була зведена до одновимірної векторної крайової задачі.
Задачу у просторі трансформантів було переформульовано як векторну крайову задачу.
Розв'язок цієї задачі було побудовано як суперпозицію загального розв'язку однорідного векторного рівняння та часткового розв'язку неоднорідного рівняння.
Розв'язок однорідного векторного рівняння було отримано за допомогою матричного диференціального числення і представлено за допомогою фундаментальної матричної системи розв'язків відповідного однорідного матричного рівняння.
Для отримання часткового розв'язку неоднорідного векторного рівняння була знайдена матриця-функція Гріна за допомогою фундаментальної базисної матричної системи розв'язків.
Після застосування оберненого перетворення Фур'є до явного розв'язку одновимірної крайової задачі у просторі трансформантів та підсумування слабко-збіжних інтегралів у формулах для переміщень залишається лише одна невідома функція переміщень по короткому торцю прямокунтої області.
Для знаходження цієї функції було отримано сингулярне інтегральне рівняння.
Сингулярне інтегральне рівняння було розв'язано за допомогою методу ортогональних поліномів.
Було проведено дослідження напруженого стану півсмуги при різних типів навантаження та розмірів прямокутної області.

\textit{Ключові слова:}
прямокутна область, динамічна задача, перетворення Фур’є, матриця-функція Гріна, сингулярне інтегральне рівняння, метод ортогональних поліномів.
