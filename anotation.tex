\textit{Пожиленков О. В.}
Плоскі мішані задачі теорії пружності для прямокутної області. - Кваліфікаційна наукова праця на правах рукопису.

Дисертація на здобуття наукового доктора філософії за спеціальністю 113 «Прикладна математика» (11 – Математика та статистика). – Одеський національний університет імені І. І. Мечникова, Одеса, 2023.

Розв'язано плоскі мішані задачі теорії пружності для пружної прямокутної області яка піддається впливу статичних та динамічних навантажень.
Шляхом застосування інтегрального скінченного sin- та cos-перетворення Фур'є вихідну задачу зведено до одновимірної крайової задачі.
Яку у просторі трансформант переформульовано у вигляді векторної крайової задачі.
Розв'язок цієї задачі побудовано як суперпозицію загального розв'язку однорідного векторного рівняння та частинного розв'язку неоднорідного рівняння.
Розв'язок однорідного векторного рівняння отримано за допомогою матричного диференціального числення і зображено за допомогою фундаментальної матричної системи розв'язків відповідного однорідного матричного рівняння.
Частковий розв'язок неоднорідного векторного рівняння знайдено за допомогою зображення матриці-функція Гріна.
Застосування оберненного перетворення Фур'є та реалізація відокремлення слабко-збіжних частин інтегралу подає поле переміщень та напружень через невідому функцію - граничне значення переміщень по торцю прямокутною області.
Для її знаходження за умови виконання крайової умови отримано сингулярне інтегральне рівняння яке розв'язано за допомогою метода ортогональних поліномів.
Було проведено дослідження напруженого стану середовища за різних типів навантаження та різних геометричних розмірів прямокутної області.

\textit{Ключові слова:}
прямокутна область, динамічна задача, перетворення Фур’є, матриця-функція Гріна, сингулярне інтегральне рівняння, метод ортогональних поліномів.
