\subsection{Постановка задачі}
Розглядається пружна прямокутна область, яка займає облась,
що описується у декартовій системі координат співвідношенням $0 \le x \le a$, $0 \le y \le b$.

До прямокутної області на грані $y=b$ додане нормальне навантаження
\begin{equation}
    \sigma_y(x, y) |_{y=b} = -p(x), \quad  \tau_{xy}(x,y) |_{y=b} =0
\end{equation}
де $p(x)$ відома функція.
На бічних гранях виконується умова ідеального контакту
\begin{equation}
    u(x,y) |_{x=0}, \quad \tau_{xy}(x,y) |_{x=0} =0
\end{equation}
\begin{equation}
    u(x,y) |_{x=a}, \quad \tau_{xy}(x,y) |_{x=a} =0
\end{equation}
На нижній грані виконуються наступні умови
\begin{equation}
    v(x,y) |_{y=0}, \quad \tau_{xy}(x,y) |_{y=0} =0
\end{equation}
Розглядаються наступні рівняння рівноваги Ламе:
\begin{equation}\label{lame_1}
    \begin{cases}
        \frac{\partial^2 u(x,y)}{\partial x^2} + \frac{\partial^2 u(x,y)}{\partial y^2} + \mu_0 (\frac{\partial^2 u(x,y)}{\partial x^2} + \frac{\partial^2 v(x,y)}{\partial x\partial y}) = 0 \\
        \frac{\partial^2 v(x,y)}{\partial x^2} + \frac{\partial^2 v(x,y)}{\partial y^2} + \mu_0 (\frac{\partial^2 u(x,y)}{\partial x \partial y} + \frac{\partial^2 v(x,y)}{\partial y^2}) = 0 \\
    \end{cases}
\end{equation}

\subsection{Зведеня задачі до одновимірної у просторі трансформант}
Для того, щоб звести задачу до одновимірної задачі, використаєм інтегральне перетворення Фур'є по змінній $x$ у до рівнянь (\ref{lame_1}) наступному вигляді:
\begin{equation}
    \begin{pmatrix}
        u_n(y) \\
        v_n(y)
    \end{pmatrix} = \int_{0}^{a} 
    \begin{pmatrix}
        u(x,y) sin(\alpha_n x) \\
        v(x,y) cos(\alpha_n x)
    \end{pmatrix} dx, \quad \alpha_n = \frac{\pi n}{a}, n=\overline{1, \infty}
\end{equation}

Для цього помножим перше та друге рівняння (\ref{lame_1}) на $sin(\alpha_n x)$ та $cos(\alpha_n x)$ відповідно та проінтегруєм по змінній $x$ на інтервалі $0 \le x \le a$.
Покрокове інтегрування рівняння (\ref{lame_1}) наведено у (\nameref{ap_A_1}).
Отримана система рівнянь задачі у просторі трансформант:
\begin{equation}\label{transf_1}
    \begin{cases}
        u_n^{''}(y) - \alpha_n \mu_0 v_n^{'}(y) - \alpha_n^2 (1 + \mu_0) u_n(y) = 0 \\
        (1 + \mu_0) v_n^{''}(y) + \alpha_n \mu_0 u_n^{'}(y)  - \alpha_n^2 v_n(y) = 0 \\
    \end{cases}
\end{equation}

Застосовуючи інтегральне перетворення до граничних умов,
отримаємо наступні умови задачі у просторі трансформант
\begin{equation}\label{transf_bound_1}
    \begin{cases}
        \left( (2G + \lambda)v_n^{'}(y) + \alpha_n \lambda u_n(y) \right)|_{y=b} = -p_n \\
        \left(u_n^{'}(y) - \alpha_n v_n(y)  \right)|_{y=b} = 0 \\
        v_n(y)|_{y=0} = 0 \\
        \left(u_n^{'}(y) - \alpha_n v_n(y)  \right)|_{y=0} = 0
    \end{cases}
\end{equation}
Де $p_n = \int_{0}^{a} p(x) cos(\alpha_n x) dx$

\subsection{Зведення задачі у просторі трансформант до матрично-векторної форми}
Для того щоб розв'язати задачу у простосторі трансформант, перепишмо її у матрично-векторній формі.
Рівняння рівноваги (\ref{transf_1}) запишемо у наступному вигляді:
\begin{align}\label{transf_mat_1}
    &L_2\left[ Z_n(y) \right] = A * Z_n^{''}(y) + B * Z_n^{'}(y) + C * Z_n(y) \nonumber \\
    & L_2\left[ Z_n(y) \right] = 0
\end{align}
Де
\begin{equation*}
    A = \begin{pmatrix}
        1 & 0 \\
        0 & 1 + \mu_0
    \end{pmatrix}, \quad
    B = \begin{pmatrix}
        0 & -\alpha_n \mu_0 \\
        \alpha_n \mu_0 & 0
    \end{pmatrix}, \quad
    C = \begin{pmatrix}
        -\alpha_n^2(1 + \mu_0) & 0 \\
        0 & -\alpha_n^2
    \end{pmatrix}
\end{equation*}
\begin{equation*}
    Z_n(y) = \begin{pmatrix}
        u_n(y) \\
        v_n(y)
    \end{pmatrix}
\end{equation*}
Граничні умови (\ref{transf_bound_1}) запишемо у наступному вигляді:
\begin{align}\label{transf_bound_mat_1}
    &U_i\left[ Z_n(y) \right] = E_i * Z_n^{'}(b_i) + F_i * Z_n(b_i) \nonumber \\
    & U_i\left[ Z_n(y) \right] = D_i
\end{align}
Де $i = \overline{0, 1}$, $b_0 = b$, $b_1 = 0$,
\begin{equation*}
    E_0 = \begin{pmatrix}
        1 & 0 \\
        0 & 2G + \lambda
    \end{pmatrix}, \quad
    F_0 = \begin{pmatrix}
        0 & -\alpha_n \\
        \alpha_n \lambda & 0
    \end{pmatrix}, \quad
\end{equation*}
\begin{equation*}
    E_1 = \begin{pmatrix}
        1 & 0 \\
        0 & 0
    \end{pmatrix}, \quad
    F_1 = \begin{pmatrix}
        0 & -\alpha_n \\
        0 & 1
    \end{pmatrix}, \quad
\end{equation*}
\begin{equation*}
    D_0 = \begin{pmatrix}
        0 \\
        -p_n
    \end{pmatrix}, \quad
    D_1 = \begin{pmatrix}
        0 \\
        0
    \end{pmatrix}, \quad
\end{equation*}

Для знаходження розв'язку задачі у просторі трансформант, знайдем фундаментальну матрицю рівняння (\ref{transf_mat_1}).
Шукати її будем у наступному вигляді:
\begin{equation}
    Y(y) = \frac{1}{2\pi i} \oint_C e^{sy} M^{-1}(s)ds
\end{equation}
Де $M(s)$ - характерестична матриця рівняння (\ref{transf_mat_1}), а $C$ - замкнений контур який містить усі особливі точки. Яку будемо шукати з наступної умовни
\begin{equation}
    L_2\left[ e^{sy}*I \right] = e^{sy} * M(s), \quad I = \begin{pmatrix} 1 & 0 \\ 0 & 1 \end{pmatrix}
\end{equation}
\begin{align*}
    &L_2\left[ e^{sy}*I \right] = e^{sy} \left( s^2A * I + s B*I + C*I \right) = \\
    &=e^{sy} \left( \begin{pmatrix}
        s^2 & 0 \\
        0 & s^2 (1 + \mu_0)
    \end{pmatrix} + \begin{pmatrix}
        0 & -\alpha_n \mu_0 s\\
        \alpha_n \mu_0 s & 0
    \end{pmatrix} + \begin{pmatrix}
        -\alpha_n^2(1 + \mu_0) & 0 \\
        0 & -\alpha_n^2
    \end{pmatrix} \right) =  \\
    &=e^{sy} \begin{pmatrix}
        s^2 -\alpha_n^2(1 + \mu_0) & -\alpha_n \mu_0 s \\
        \alpha_n \mu_0 s & s^2 (1 + \mu_0) -\alpha_n^2
     \end{pmatrix} =>
\end{align*}

\begin{equation}
    M(s) = \begin{pmatrix}
        s^2 -\alpha_n^2(1 + \mu_0) & -\alpha_n \mu_0 s \\
        \alpha_n \mu_0 s & s^2 (1 + \mu_0) -\alpha_n^2
     \end{pmatrix}
\end{equation}

Знайдемо тепер $M^{-1}(s) = \frac{\widetilde{M(s)}}{det[M(s)]}$.
\begin{equation}
    \widetilde{M(s)} = \begin{pmatrix}
        s^2 (1 + \mu_0) -\alpha_n^2 & \alpha_n \mu_0 s \\
        -\alpha_n \mu_0 s & s^2 -\alpha_n^2(1 + \mu_0)
     \end{pmatrix}
\end{equation}
\begin{align}
    &det[M(s)] = (s^2 -\alpha_n^2(1 + \mu_0))(s^2 (1 + \mu_0) -\alpha_n^2) + (\alpha_n \mu_0 s)^2 = \nonumber \\
    &=(1+\mu_0)(s - \alpha_n)^2(s + \alpha_n)^2
\end{align}

В раховучи це, тепер знайдемо значення фундаментальної матрицю за допомогою теореми про лишки:
\begin{align*}
    &\frac{1}{2\pi i} \oint_C e^{sy} M^{-1}(s)ds = \frac{2 \pi i}{2 \pi i (1 + \mu_0)} \sum_{i=1}^{2} Res\left[ e^{sy} \frac{\widetilde{M(s)}}{det[M(s)]} \right] = \\
    & = \frac{1}{(1 + \mu_0)} \left(Y_0(y) + Y_1(y) \right)
\end{align*}
Знайдем $Y_0(y)$
\begin{align}
    &Y_0(y) =  \frac{\partial}{\partial s} \left( \frac{e^{sy}}{(s+\alpha_n)^2} \widetilde{M(s)} \right) \Big|_{s=\alpha_n} = \nonumber \\
    &=\frac{e^{\alpha_n y}}{4\alpha_n} \begin{pmatrix}
    \alpha_n \mu_0 y + 2 + \mu_0 & \alpha_n \mu_0 y \\
    -\alpha_n \mu_0 y & -\alpha_n \mu_0 y + 2 + \mu_0
    \end{pmatrix}
\end{align}
Знайдем $Y_1(y)$
\begin{align}
    &Y_1(y) = \frac{\partial}{\partial s} \left(\frac{e^{sy}}{(s-\alpha_n)^2} \widetilde{M(s)} \right) \Big|_{s=-\alpha_n} = \nonumber \\
    =&\frac{e^{-\alpha_n y}}{4\alpha_n} \begin{pmatrix}
    \alpha_n \mu_0 y - 2 - \mu_0 & -\alpha_n \mu_0 y \\
    \alpha_n \mu_0 y & -\alpha_n \mu_0 y - 2 - \mu_0
    \end{pmatrix}
\end{align}

Таким чином ми можемо записати розв'язок задачі у просторі трансформант:
\begin{equation}
    Z_n(y) = \frac{1}{1 + \mu_0} \left( Y_0(y) * \begin{pmatrix} c_1 \\ c_2 \end{pmatrix} +  Y_0(y) * \begin{pmatrix} c_3 \\ c_4 \end{pmatrix}  \right)
\end{equation}
Залишилось знайти невідомі коєфіцієнти $c_1$, $c_2$, $c_3$, $c_4$, використовуючи граничні умови (\ref{transf_bound_mat_1}).
Покрокове знаходження коєфіцієнтів наведено у (\nameref{ap_B_1}).
Таким чином ми можемо записати розв'зок у просторі трансформант:
\begin{align}\label{transf_sol_u}
    &u_n(y) = \frac{e^{\alpha_n y}}{4 \alpha_n (1 + \mu_0)} \left[c_1 (\alpha_n \mu_0 y + 2 + \mu_0) + c_2 (\alpha_n \mu_0 y) \right] + \nonumber \\
    &\quad + \frac{e^{-\alpha_n y}}{4 \alpha_n (1 + \mu_0)} \left[c_3 (\alpha_n \mu_0 y - 2 - \mu_0) + c_4 (-\alpha_n \mu_0 y)\right]
\end{align}
\begin{align}\label{transf_sol_v}
    &v_n(y) = \frac{e^{\alpha_n y}}{4 \alpha_n (1 + \mu_0)} \left[c_1 (-\alpha_n \mu_0 y) + c_2 (-\alpha_n \mu_0 y + 2 + \mu_0) \right] + \nonumber \\
    &\quad + \frac{e^{-\alpha_n y}}{4 \alpha_n (1 + \mu_0)} \left[c_3 (\alpha_n \mu_0 y) + c_4 (-\alpha_n \mu_0 y - 2 - \mu_0)\right]
\end{align}

\subsection{Фінальний розв'язок задачі}
Викорустовуючи обернене інтегральне перетворення Фур'є до розв'язку задачі у просторі трансформант
(\ref{transf_sol_u}), (\ref{transf_sol_v}), отримаємо фінальний розв'язок задачі
\begin{equation}
    u(x,y) = \frac{2}{a} \sum_{n=1}^{\infty} u_n(y) sin(\alpha_n x), \quad \alpha_n = \frac{\pi n}{a}
\end{equation}
\begin{equation}
    v(x,y) = \frac{v_0(y)}{a} + \frac{2}{a} \sum_{n=1}^{\infty} v_n(y) cos(\alpha_n x), \quad \alpha_n = \frac{\pi n}{a}
\end{equation}

Останній крок це знаходження $v_0(y)$ у випадку коли $n=0$, $\alpha_n =0$.
Для цього повернемся до другого рівняння (\ref{transf_1}), та запишем його для цього випадку:
\begin{equation}\label{transf_1_v_0}
    (1 + \mu_0) v_n^{''}(y) = 0
\end{equation}
Та граничні умови:
\begin{equation}\label{transf_bound_1_v_0}
    \begin{cases}
        (2G + \lambda)v_0^{'}(y)|_{y=b} = -p_0 \\
        v_0(y)|_{y=0} = 0
    \end{cases}
\end{equation}
Де $p_0 = \int_{0}^{a}p(x)dx$

Розв'язок рівняння (\ref{transf_1_v_0}):
\begin{equation}
    v_0(y) = c_1 + c_2 y
\end{equation}
Застовоючи граничні умови (\ref{transf_bound_1_v_0}) для знаходження коєфіцієнтів $c_1$, $c_2$, отримаємо розв'язок задачі задачі:
\begin{equation}
    v_0(y) = \frac{-p_0}{(2G + \lambda)}y
\end{equation}
Тепер остаточний розв'зок задачі можна записати у вигляді:
\begin{equation}
    \begin{cases}
        u(x,y) = \frac{2}{a} \sum_{n=1}^{\infty} u_n(y) sin(\alpha_n x), \quad \alpha_n = \frac{\pi n}{a} \\
        v(x,y) = \frac{-p_0}{(2G + \lambda)a}y + \frac{2}{a} \sum_{n=1}^{\infty} v_n(y) cos(\alpha_n x), \quad \alpha_n = \frac{\pi n}{a}
    \end{cases}
\end{equation}

\subsection{Чисельні розрахунки}
