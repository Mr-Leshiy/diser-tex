Знайдем $v_0(y)$ розглянувши задачу у просторі трансформант \eqref{transf_gen}, \eqref{transf_bound_gen} при $n=0$, $\alpha_n = 0$.
Отримаємо наступну задачу відносно $v_0(y)$:
\begin{align*}
    &v_0^{''}(y) + \frac{\omega^2}{c_2^2(1+\mu_0) }v_0(y) = \frac{f(y)}{1+\mu_0}
\end{align*}
Де $f(y)=(\frac{\alpha_2}{\beta_2}\chi_4(y) cos(\alpha_n a) - \frac{\alpha_1}{\beta_1}\chi_2(y)) - \frac{\mu_0}{(1+\mu_0)} (\chi_3^{'}(y) cos(\alpha_n a) -\chi_1^{'}(y))$.

Та граничні умови:
\begin{equation*}
    (2G + \lambda)v_0^{'}(b) = -p_0, \quad v_0(0) = 0, \quad p_0 = \int_{0}^{a}p(x)dx
\end{equation*}
Спочатку знайдем фундаментальну базисну систему розв'язків задачі $\psi_0(y)$, $\psi_1(y)$:
\begin{equation*}
    \psi_i^{''}(y) + \frac{\omega^2}{c_2^2(1+\mu_0) }\psi_i(y) = 0, i=\overline{0,1} \\
\end{equation*}
\begin{equation*}
    \begin{cases}
        \psi_0(0) = 1 \\
        \psi_0^{'}(b) = 0
    \end{cases}, \quad
    \begin{cases}
        \psi_1(0) = 0 \\
        \psi_1^{'}(b) = 1
    \end{cases}
\end{equation*}
Розв'язок однорідної задачі відносно $\psi_i(y)$ має вигляд:
\begin{equation}
    \psi_i(y) = c_1^i cos\left( \frac{\omega}{c_2 \sqrt{1 + \mu_0}} y \right) + c_2^i sin\left( \frac{\omega}{c_2 \sqrt{1 + \mu_0}} y \right)
\end{equation}
Враховучи граничні умови отримаємо остаточний вигляд $\psi_0(y)$, $\psi_1(y)$:
\begin{align*}
    \begin{cases}
        \psi_0(y) = cos\left( \frac{\omega}{c_2 \sqrt{1 + \mu_0}} y \right) +  tg\left( \frac{\omega}{c_2 \sqrt{1 + \mu_0}} b \right) sin\left( \frac{\omega}{c_2 \sqrt{1 + \mu_0}} y \right) \\
        \psi_1(y) = \frac{c_2 (1 + \mu_0)}{\omega cos\left( \frac{\omega}{c_2 \sqrt{1 + \mu_0}} b \right)} sin\left( \frac{\omega}{c_2 \sqrt{1 + \mu_0}} y \right)
    \end{cases}
\end{align*}
Побудуємо тепер функцію Гріна задачі:
\begin{equation*}
    g(y, \xi) = \begin{cases}
        -a_1(\xi) \psi_1(y), 0 \le y < \xi \\
        a_0(\xi) \psi_0(y), \xi < y \le b
    \end{cases}
\end{equation*}
Де $a_0(\xi)$, $a_1(\xi)$ будуть знайдені з наступної системи
\begin{equation*}
    \begin{cases}
        a_0(\xi) \psi_0^{'}(\xi) + a_1(\xi) \psi_1^{'}(\xi) = 1 \\
        a_0(\xi) \psi_0^(\xi) + a_1(\xi) \psi_1(\xi) = 0
    \end{cases}
\end{equation*}
Таким чином остаточний розв'язок задачі відносно $v_0(y)$ буде мати наступний вигляд:
\begin{equation*}
    v_0(y) = \frac{1}{(1+\mu_0)} \int_{0}^{b}g(y,\xi) f(\xi) d\xi - \psi_0(y) \frac{p_0}{2G + \lambda}
\end{equation*}

У випадку статичної задачі коли $\omega = 0$ отримаємо наступну задачу відносно $v_0(y)$:
\begin{equation*}
    v_0^{''}(y) = \frac{f(y)}{1+\mu_0}
\end{equation*}
\begin{equation*}
    (2G + \lambda)v_0^{'}(b) = -p_0, \quad v_0(0) = 0, \quad p_0 = \int_{0}^{a}p(x)dx
\end{equation*}