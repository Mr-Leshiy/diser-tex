\subsection*{Випадок статичної задачі}
Для знаходження коєфіцієтів $c_1$, $c_2$, $c_3$, $c_4$ випадку статичної задачі \eqref{transf_mat_static_1} спочатку
знайдем $Y_0(y) * \begin{pmatrix}c_1 \\ c_2\end{pmatrix}$ та $Y_1(y) * \begin{pmatrix}c_3 \\ c_4\end{pmatrix}$.
\begin{align*}
    &Y_0(y) * \begin{pmatrix}c_1 \\ c_2\end{pmatrix} = \frac{e^{\alpha_n y}}{4\alpha_n} \begin{pmatrix}
        \alpha_n \mu_0 y + 2 + \mu_0 & \alpha_n \mu_0 y \\
        -\alpha_n \mu_0 y & -\alpha_n \mu_0 y + 2 + \mu_0
        \end{pmatrix} * \begin{pmatrix}c_1 \\ c_2\end{pmatrix} = \\
    &=\frac{e^{\alpha_n y}}{4\alpha_n} \begin{pmatrix}
        c_1(\alpha_n \mu_0 y + 2 + \mu_0) + c_2(\alpha_n \mu_0 y) \\
        c_1(-\alpha_n \mu_0 y) + c_2(-\alpha_n \mu_0 y + 2 + \mu_0)
        \end{pmatrix}
\end{align*}
\begin{align*}
    &Y_1(y) * \begin{pmatrix}c_3 \\ c_4\end{pmatrix} = \frac{e^{-\alpha_n y}}{4\alpha_n} \begin{pmatrix}
        \alpha_n \mu_0 y - 2 - \mu_0 & -\alpha_n \mu_0 y \\
        \alpha_n \mu_0 y & -\alpha_n \mu_0 y - 2 - \mu_0
        \end{pmatrix} * \begin{pmatrix}c_3 \\ c_4\end{pmatrix} = \\
    &=\frac{e^{-\alpha_n y}}{4\alpha_n} \begin{pmatrix}
        c_3(\alpha_n \mu_0 y - 2 - \mu_0) + c_4(-\alpha_n \mu_0 y) \\
        c_3(\alpha_n \mu_0 y) + c_4(-\alpha_n \mu_0 y - 2 - \mu_0)
        \end{pmatrix}
\end{align*}
Введемо позначення $c = \frac{1}{4\alpha_n (1 + \mu_0)}$. \newline
Запишем тепер $Z_n(y)$:
\begin{align*}
    &Z_n(y) = c
    \begin{pmatrix}
        c_1 e^{\alpha_n y} (\alpha_n \mu_0 y + 2 + \mu_0) + c_2 e^{\alpha_n y} (\alpha_n \mu_0 y) + 
        \\ + c_3 e^{-\alpha_n y} (\alpha_n \mu_0 y - 2 - \mu_0) + c_4 e^{-\alpha_n y} (-\alpha_n \mu_0 y) \\
        \\
        c_1 e^{\alpha_n y} (-\alpha_n \mu_0 y) + c_2 e^{\alpha_n y} (-\alpha_n \mu_0 y + 2 + \mu_0) + 
        \\ + c_3 e^{-\alpha_n y} (\alpha_n \mu_0 y) + c_4 e^{-\alpha_n y} (-\alpha_n \mu_0 y - 2 - \mu_0)
    \end{pmatrix}
\end{align*}
Тепер $Z_n^{'}(y)$:
\begin{align*}
    &Z_n^{'}(y) = c
    \begin{pmatrix}
        c_1 e^{\alpha_n y} (\alpha_n^2 \mu_0 y + 2 \alpha_n + 2 \alpha_n \mu_0) + c_2 e^{\alpha_n y} (\alpha_n^2 \mu_0 y + \alpha_n \mu_0) + 
        \\ + c_3 e^{-\alpha_n y} (-\alpha_n^2 \mu_0 y + 2 \alpha_n + 2 \alpha_n \mu_0) + c_4 e^{-\alpha_n y} (\alpha_n^2 \mu_0 y - \alpha_n \mu_0) \\
        \\
        c_1 e^{\alpha_n y} (-\alpha_n \mu_0 y) + c_2 e^{\alpha_n y} (-\alpha_n \mu_0 y + 2 + \mu_0) + 
        \\ + c_3 e^{-\alpha_n y} (\alpha_n \mu_0 y) + c_4 e^{-\alpha_n y} (-\alpha_n \mu_0 y - 2 - \mu_0)
    \end{pmatrix}
\end{align*}
Тепер використаєм граничні умови (\ref{transf_bound_mat_static_1}) та побудуєм алгебричну систему відносно коєфіцієнтів.

Використаєм $U_0\left[ Z_n(y) \right]$:
\begin{eqnarray*}
    E_0 * Z_n^{'}(b) + F_0 * Z_n(b) = D_0 \Leftrightarrow
\end{eqnarray*}
\begin{equation*}
    \begin{pmatrix}
        1 & 0 \\
        0 & 2G + \lambda
    \end{pmatrix} * Z_n^{'}(b) + \begin{pmatrix}
        0 & -\alpha_n \\
        \alpha_n \lambda & 0
    \end{pmatrix} * Z_n(b) = \begin{pmatrix}
        0 \\
        -p_n
    \end{pmatrix}
\end{equation*}
Отримаємо перші 2 рівняння системи:
\begin{equation*}
    \begin{cases}
        c_1 e^{\alpha_n b} (\alpha_n^2 \mu_0 b + \alpha_n \mu_0 + \alpha_n) + c_2 e^{\alpha_n b} (\alpha_n^2 \mu_0 b - \alpha_n) + \\
        + c_3 e^{-\alpha_n b} (-\alpha_n^2 \mu_0 b + \alpha_n + \alpha_n \mu_0) + c_4 e^{-\alpha_n b} (\alpha_n^2 \mu_0 b + \alpha_n) = 0 \\
        \\
        c_1 e^{\alpha_n b} (-2 G \alpha_n^2 \mu_0 b - 2 G \alpha_n \mu_0 + 2 \lambda \alpha_n) + c_2 e^{\alpha_n b} (-2G \alpha_n^2 \mu_0 b + \\
        + (2G + \lambda) 2 \alpha_n) + c_3  e^{-\alpha_n b} (-2 G \alpha_n^2 \mu_0 b + 2G \alpha_n \mu_0 - 2\lambda \alpha_n) + \\ 
        + c_4 e^{-\alpha_n b} (2G \alpha_n^2 \mu_0 b + (2G + \lambda) 2 \alpha_n) = -c p_n
    \end{cases}
\end{equation*}

Використаєм $U_1\left[ Z_n(y) \right]$:
\begin{eqnarray*}
    E_1 * Z_n^{'}(0) + F_1 * Z_n(0) = D_1 \Leftrightarrow
\end{eqnarray*}
\begin{equation*}
    \begin{pmatrix}
        1 & 0 \\
        0 & 0
    \end{pmatrix} * Z_n^{'}(0) + \begin{pmatrix}
        0 & -\alpha_n \\
        0 & 1
    \end{pmatrix} * Z_n(0) = \begin{pmatrix}
        0 \\
        0
    \end{pmatrix}
\end{equation*}
Отримаємо другі 2 рівняння системи:
\begin{equation*}
    \begin{cases}
        c_1 (\alpha_n + \alpha_n \mu_0) + c_2 (-\alpha_n) + c_3 (\alpha_n + \alpha_n \mu_0) + c_4 (\alpha_n) = 0 \\
        \\
        c_2 (2 + \mu_0) + c_4 (-2 - \mu_0) = 0
    \end{cases}
\end{equation*}
Звідси видно, що $c_3 = -c_1$, $c_4 = c_2$.
Введемо наступні позначення:
\begin{align*}
    &a_1 = e^{\alpha_n b} (\alpha_n^2 \mu_0 b + \alpha_n \mu_0 + \alpha_n) - e^{-\alpha_n b} (-\alpha_n^2 \mu_0 b + \alpha_n + \alpha_n \mu_0), \\
    &a_2 = e^{\alpha_n b} (\alpha_n^2 \mu_0 b - \alpha_n) + e^{-\alpha_n b} (\alpha_n^2 \mu_0 b + \alpha_n), \\
    &a_3 = e^{\alpha_n b} (-2 G \alpha_n^2 \mu_0 b - 2 G \alpha_n \mu_0 + 2 \lambda \alpha_n) - \\
    &\quad - e^{-\alpha_n b} (-2 G \alpha_n^2 \mu_0 b + 2G \alpha_n \mu_0 - 2\lambda \alpha_n) \\
    &a_4 = e^{\alpha_n b} (-2G \alpha_n^2 \mu_0 b + (2G + \lambda) 2 \alpha_n) + \\
    &\quad + e^{-\alpha_n b} (2G \alpha_n^2 \mu_0 b + (2G + \lambda) 2 \alpha_n)
\end{align*}
Враховуючи останнє отримаємо:
\begin{equation*}
    \begin{cases}
        c_3 = -c_1 \\
        c_4 = c_2 \\
        c_1 a_1 + c_2 a_2 = 0 \\
        c_1 a_3 + c_2 a_4 = -c p_n
    \end{cases} \Leftrightarrow, \quad 
    \begin{cases}
        c_3 = -c_1 \\
        c_4 = c_2 \\
        c_1 = - c_2 \frac{a_2}{a_1} \\
        c_2(a_4 a_1 - a_2 a_3) = -c p_n a_1
    \end{cases} \Leftrightarrow
\end{equation*}
\begin{equation*}
    \begin{cases}
        c_1 = c p_n \frac{a_2}{(a_4 a_1 - a_2 a_3)} \\
        c_2 = - c p_n \frac{a_1}{(a_4 a_1 - a_2 a_3)} \\
        c_3 = -c p_n \frac{a_2}{(a_4 a_1 - a_2 a_3)} \\
        c_4 = - c p_n \frac{a_1}{(a_4 a_1 - a_2 a_3)}
    \end{cases}
\end{equation*}

\subsection*{Випадок динамічної задачі}
Розглянемо випадок динамічної задачі. Введемо наступні позначення
\begin{align*}
    &x_1 = \alpha_n \mu_0 s_1, \quad x_2 = \alpha_n \mu_0 s_2 \\
    &x_3 = s_1^2(1 + \mu_0) - \alpha_n^2 + \frac{\omega^2}{c_2^2}, \quad x_4 = s_2^2(1 + \mu_0) - \alpha_n^2 + \frac{\omega^2}{c_2^2} \\
    &x_5 = s_1^2 - \alpha_n^2(1 + \mu_0) + \frac{\omega^2}{c_1^2}, \quad x_6 = s_2^2 - \alpha_n^2(1 + \mu_0) + \frac{\omega^2}{c_1^2} \\
    &y_1 = 2s_1 (s_1^2 - s_2^2), \quad y_2 = 2 s_2 (s_2^2 - s_1^2) \\
    &z_1 = \frac{(e^{b s_1} + e^{-b s_1}) (s_1 x_3 + \alpha_n x_1)}{y_1}, \quad z_2 = \frac{(e^{b s_1} - e^{-b s_1}) (s_1 x_1 - \alpha_n x_5)}{y_1} \\
    &z_3 = \frac{(e^{b s_2} + e^{-b s_2}) (s_2 x_4 + \alpha_n x_2)}{y_2}, \quad z_4 = \frac{(e^{b s_2} - e^{-b s_2}) (s_2 x_4 - \alpha_n x_6)}{y_2} \\
    &z_5 = \frac{(e^{b s_1} - e^{-b s_1}) (s_1 x_3 - s_1 x_1 (2G + \lambda))}{y_1}, \quad z_6 = \frac{(e^{b s_1} + e^{-b s_1}) (s_1 x_5 (2G + \lambda) + \alpha_n \lambda x_1)}{y_1} \\
    &z_7 = \frac{(e^{b s_2} - e^{-b s_2}) (\alpha_n \lambda x_4 - s_2 x_2 (2G + \lambda))}{y_2}, \quad z_8 = \frac{(e^{b s_2} + e^{-b s_2}) (s_2 x_6 (2G + \lambda) + \alpha_n \lambda x_2)}{y_2} \\
    &z_9 = \frac{s_1 x_3 + \alpha_n x_1}{y_1}, \quad z_{10} = \frac{s_2 x_4 + \alpha_n x_2}{y_2} \\
    &z_{11} = \frac{x_1}{y_1}, \quad z_{12} = \frac{x_2}{y_2}
\end{align*}
Таким чином отримаємо наступну систему:
\begin{equation*}
    \begin{cases}
        z_1 c_1 + z_2 c_2 + z_3 c_3 + z_4 c_4 = 0 \\
        z_5 c_1 + z_6 c_2 + z_7 c_3 + z_8 c_4 = -p_n \\
        z_9 c_1 + z_{10} c_3 = 0 \\
        z_{11} c_1 + z_{12} c_3 = 0
    \end{cases}, \Leftrightarrow 
    \begin{cases}
        c_1 = 0 \\
        c_3 = 0 \\
        z_2 c_2 + z_4 c_4 = 0 \\
        z_6 c_2 + z_8 c_4 = -p_n
    \end{cases}, \Leftrightarrow 
\end{equation*}
\begin{equation*}
    \begin{cases}
        c_1 = 0 \\
        c_3 = 0 \\
        c_2 = - \frac{z_4}{z_2} c_4 \\
        z_6 c_2 + z_8 c_4 = -p_n
    \end{cases}, \Leftrightarrow 
    \begin{cases}
        c_1 = 0 \\
        c_3 = 0 \\
        c_2 = p_n \frac{z_4}{z_8 z_2 - z_4 z_6} \\
        c_4 = -p_n \frac{z_2}{z_8 z_2 - z_4 z_6}
    \end{cases}
\end{equation*}