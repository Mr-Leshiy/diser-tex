\subsubsection{Постановка задачі}
\begin{figure}[h]
    \begin{center}
        \includegraphics[scale=1]{images/geometry/image_3.jpg}
    \end{center}
    \caption{Геометрія проблеми}\label{geom_static_1}
\end{figure}
Розглядається пружна прямокутна область (Рис: \ref{geom_static_1}), яка займає облась,
що описується у декартовій системі координат співвідношенням $0 \le x \le a$, $0 \le y \le b$.

До прямокутної області на грані $y=b$ додане нормальне навантаження
\begin{equation}
    \sigma_y(x, y) |_{y=b} = -p(x), \quad  \tau_{xy}(x,y) |_{y=b} =0
\end{equation}
де $p(x)$ відома функція.

На бічних гранях виконується умова ідеального контакту
\begin{equation}\label{bound_1_static_1}
    u(x,y) |_{x=0} = 0, \quad \tau_{xy}(x,y) |_{x=0} =0
\end{equation}
\begin{equation}\label{bound_2_static_1}
    u(x,y) |_{x=a} = 0, \quad \tau_{xy}(x,y) |_{x=a} =0
\end{equation}
На нижній грані виконуються наступні умови
\begin{equation}
    v(x,y) |_{y=0} = 0, \quad \tau_{xy}(x,y) |_{y=0} =0
\end{equation}
Розглядаються наступні рівняння рівноваги Ламе:
\begin{equation}\label{lame_static_1}
    \begin{cases}
        \frac{\partial^2 u(x,y)}{\partial x^2} + \frac{\partial^2 u(x,y)}{\partial y^2} + \mu_0 (\frac{\partial^2 u(x,y)}{\partial x^2} + \frac{\partial^2 v(x,y)}{\partial x\partial y}) = 0 \\
        \frac{\partial^2 v(x,y)}{\partial x^2} + \frac{\partial^2 v(x,y)}{\partial y^2} + \mu_0 (\frac{\partial^2 u(x,y)}{\partial x \partial y} + \frac{\partial^2 v(x,y)}{\partial y^2}) = 0 \\
    \end{cases}
\end{equation}

\subsubsection{Побудова точного розв'язку вихідної задачі}
Для того, щоб звести задачу до одновимірної задачі, використаємо інтегральне перетворення Фур'є по змінній $x$ до рівнянь (\ref{lame_static_1}) в наступному вигляді:
\begin{equation}
    \begin{pmatrix}
        u_n(y) \\
        v_n(y)
    \end{pmatrix} = \int_{0}^{a} 
    \begin{pmatrix}
        u(x,y) sin(\alpha_n x) \\
        v(x,y) cos(\alpha_n x)
    \end{pmatrix} dx, \quad \alpha_n = \frac{\pi n}{a}
\end{equation}

Для цього помножимо перше та друге рівняння (\ref{lame_static_1}) на $sin(\alpha_n x)$ та $cos(\alpha_n x)$ відповідно та проінтегруємо по змінній $x$ на інтервалі $0 \le x \le a$.

Розглянемо перше рівняня
\begin{align*}
    &\int_{0}^{a} \frac{\partial^2 u(x,y)}{\partial x^2} sin(\alpha_n x)dx + \int_{0}^{a} \frac{\partial^2 u(x,y)}{\partial y^2} sin(\alpha_n x)dx + \\ 
    & + \mu_0 \left( \int_{0}^{a} \frac{\partial^2 u(x,y)}{\partial x^2} sin(\alpha_n x)dx + \int_{0}^{a} \frac{\partial^2 v(x,y)}{\partial x \partial y} sin(\alpha_n x) dx\right) = 0
\end{align*}

Розглянемо
\begin{align*}
    &\int_{0}^{a} \frac{\partial^2 u(x,y)}{\partial x^2} sin(\alpha_n x)dx = \frac{\partial u(x,y)}{\partial x} sin(\alpha_n x) |_{x=0}^{x=a} - \alpha_n \int_{0}^{a} \frac{\partial u(x,y)}{\partial x} cos(\alpha_n x)dx = \\
    &= \frac{\partial u(x,y)}{\partial x} sin(\alpha_n x) |_{x=0}^{x=a} - \alpha_n \left( u(x,y) cos(\alpha_n x) |_{x=0}^{x=a} + \alpha_n \int_{0}^{a} u(x,y) sin(\alpha_n x) dx \right) = \\
    &= -\alpha_n^2 u_n(y)
\end{align*}

Розглянемо
\begin{align*}
    &\int_{0}^{a} \frac{\partial^2 u(x,y)}{\partial y^2} sin(\alpha_n x)dx = \frac{\partial^2}{\partial y^2} \int_{0}^{a} u(x,y) sin(\alpha_n x)dx = u_n^{''}(y)
\end{align*}

Розглянемо
\begin{align*}
    &\int_{0}^{a} \frac{\partial^2 v(x,y)}{\partial x \partial y} sin(\alpha_n x) dx = \frac{\partial v(x,y)}{\partial y} sin(\alpha_n x) |_{x=0}^{x=a} - \alpha_n \int_{0}^{a} \frac{\partial v(x,y)}{\partial y} cos(\alpha_n x) dx = \\
    &= -\alpha_n \frac{\partial}{\partial y} \int_{0}^{a} v(x,y) cos(\alpha_n x) dx = -\alpha_n v_n^{'}(y)
\end{align*}

Тоді перше рівняння у просторі трансформант прийме вигляд:
\begin{align*}
    &u_n^{''}(y) - \alpha_n \mu_0 v_n^{'}(y) -(\alpha_n^2 + \alpha_n^2 \mu_0) u_n(y) = 0
\end{align*}

Розлянемо друге рівняння
\begin{align*}
    &\int_{0}^{a} \frac{\partial^2 v(x,y)}{\partial x^2} cos(\alpha_n x)dx + \int_{0}^{a} \frac{\partial^2 v(x,y)}{\partial y^2} cos(\alpha_n x)dx + \\ 
    & + \mu_0 \left( \int_{0}^{a} \frac{\partial^2 u(x,y)}{\partial x \partial y} cos(\alpha_n x)dx +  \int_{0}^{a} \frac{\partial^2 v(x,y)}{\partial y^2} cos(\alpha_n x) dx\right) = 0
\end{align*}

Розглянемо
\begin{align*}
    &\int_{0}^{a} \frac{\partial^2 v(x,y)}{\partial x^2} cos(\alpha_n x)dx = \frac{\partial v(x,y)}{\partial x} cos(\alpha_n x) |_{x=0}^{x=a} + \alpha_n \int_{0}^{a} \frac{\partial v(x,y)}{\partial x} sin(\alpha_n x) dx = \\
    &=\frac{\partial v(x,y)}{\partial x} cos(\alpha_n x) |_{x=0}^{x=a} + \alpha_n \left(v(x,y) sin(\alpha_n x)|_{x=0}^{x=a} - \alpha_n \int_{0}^{a} v(x,y) cos(\alpha_n x) dx  \right) = \\
    &= -\alpha_n^2 v_n(y)
\end{align*}

Розглянемо
\begin{align*}
    &\int_{0}^{a} \frac{\partial^2 v(x,y)}{\partial y^2} cos(\alpha_n x)dx = \frac{\partial^2}{\partial y^2} \int_{0}^{a} v(x,y) cos(\alpha_n x)dx = v_n^{''}(y)
\end{align*}

Розглянемо
\begin{align*}
    &\int_{0}^{a} \frac{\partial^2 u(x,y)}{\partial y \partial x} cos(\alpha_n x)dx = \frac{\partial u(x,y)}{\partial y} cos(\alpha_n x) |_{x=0}^{x=a} + \alpha_n \int_{0}^{a} \frac{\partial u(x,y)}{\partial y} sin(\alpha_n x) dx = \\
    &=\frac{\partial u(x,y)}{\partial y} cos(\alpha_n x) |_{x=0}^{x=a} + \alpha_n \frac{\partial}{\partial y} \int_{0}^{a} u(x,y) sin(\alpha_n x) dx = \alpha_n u_n^{'}(y)
\end{align*}

Тоді друге рівняння у просторі трансформант прийме вигляд:
\begin{align*}
    &(1 + \mu_0) v_n^{''}(y) + \alpha_n \mu_0 u_n^{'}(y)  - \alpha_n^2 v_n(y) = 0
\end{align*}

Отримана система рівнянь задачі у просторі трансформант:
\begin{equation}\label{transf_static_1}
    \begin{cases}
        u_n^{''}(y) - \alpha_n \mu_0 v_n^{'}(y) - \alpha_n^2 (1 + \mu_0) u_n(y) = 0 \\
        (1 + \mu_0) v_n^{''}(y) + \alpha_n \mu_0 u_n^{'}(y)  - \alpha_n^2 v_n(y) = 0 \\
    \end{cases}
\end{equation}

Застосовуючи інтегральне перетворення до граничних умов,
отримаємо наступні умови задачі у просторі трансформант
\begin{equation}\label{transf_bound_static_1}
    \begin{cases}
        \left( (2G + \lambda)v_n^{'}(y) + \alpha_n \lambda u_n(y) \right)|_{y=b} = -p_n \\
        \left(u_n^{'}(y) - \alpha_n v_n(y)  \right)|_{y=b} = 0 \\
        v_n(y)|_{y=0} = 0 \\
        \left(u_n^{'}(y) - \alpha_n v_n(y)  \right)|_{y=0} = 0
    \end{cases}
\end{equation}
де $p_n = \int_{0}^{a} p(x) cos(\alpha_n x) dx$

Для того щоб розв'язати задачу у простосторі трансформант, перепишемо її у матрично-векторній формі.
Рівняння рівноваги (\ref{transf_static_1}) запишемо у наступному вигляді:
\begin{align}\label{transf_mat_static_1}
    &L_2\left[ Z_n(y) \right] = A * Z_n^{''}(y) + B * Z_n^{'}(y) + C * Z_n(y) \nonumber \\
    & L_2\left[ Z_n(y) \right] = 0
\end{align}
Де
\begin{equation*}
    A = \begin{pmatrix}
        1 & 0 \\
        0 & 1 + \mu_0
    \end{pmatrix}, \quad
    B = \begin{pmatrix}
        0 & -\alpha_n \mu_0 \\
        \alpha_n \mu_0 & 0
    \end{pmatrix}, \quad
    C = \begin{pmatrix}
        -\alpha_n^2(1 + \mu_0) & 0 \\
        0 & -\alpha_n^2
    \end{pmatrix}
\end{equation*}
\begin{equation*}
    Z_n(y) = \begin{pmatrix}
        u_n(y) \\
        v_n(y)
    \end{pmatrix}
\end{equation*}
Граничні умови (\ref{transf_bound_static_1}) запишемо у наступному вигляді:
\begin{align}\label{transf_bound_mat_static_1}
    &U_i\left[ Z_n(y) \right] = E_i * Z_n^{'}(b_i) + F_i * Z_n(b_i) \nonumber \\
    & U_i\left[ Z_n(y) \right] = D_i
\end{align}
де $i = \overline{0, 1}$, $b_0 = b$, $b_1 = 0$,
\begin{equation*}
    E_0 = \begin{pmatrix}
        1 & 0 \\
        0 & 2G + \lambda
    \end{pmatrix}, \quad
    F_0 = \begin{pmatrix}
        0 & -\alpha_n \\
        \alpha_n \lambda & 0
    \end{pmatrix}, \quad
\end{equation*}
\begin{equation*}
    E_1 = \begin{pmatrix}
        1 & 0 \\
        0 & 0
    \end{pmatrix}, \quad
    F_1 = \begin{pmatrix}
        0 & -\alpha_n \\
        0 & 1
    \end{pmatrix}, \quad
\end{equation*}
\begin{equation*}
    D_0 = \begin{pmatrix}
        0 \\
        -p_n
    \end{pmatrix}, \quad
    D_1 = \begin{pmatrix}
        0 \\
        0
    \end{pmatrix}, \quad
\end{equation*}

Для знаходження розв'язку задачі у просторі трансформант, знайдемо фундаментальну матрицю рівняння (\ref{transf_mat_static_1}).
Шукати її будем у наступному вигляді \cite{gantmaher}:
\begin{equation}
    Y(y) = \frac{1}{2\pi i} \oint_C e^{sy} M^{-1}(s)ds
\end{equation}
Де $M(s)$ - характерестична матриця рівняння (\ref{transf_mat_static_1}), а $C$ - замкнений контур який містить усі особливі точки. Яку будемо шукати з наступної умовни
\begin{equation}
    L_2\left[ e^{sy}*I \right] = e^{sy} * M(s), \quad I = \begin{pmatrix} 1 & 0 \\ 0 & 1 \end{pmatrix}
\end{equation}
\begin{align*}
    &L_2\left[ e^{sy}*I \right] = e^{sy} \left( s^2A * I + s B*I + C*I \right) = \\
    &=e^{sy} \left( \begin{pmatrix}
        s^2 & 0 \\
        0 & s^2 (1 + \mu_0)
    \end{pmatrix} + \begin{pmatrix}
        0 & -\alpha_n \mu_0 s\\
        \alpha_n \mu_0 s & 0
    \end{pmatrix} + \begin{pmatrix}
        -\alpha_n^2(1 + \mu_0) & 0 \\
        0 & -\alpha_n^2
    \end{pmatrix} \right) =  \\
    &=e^{sy} \begin{pmatrix}
        s^2 -\alpha_n^2(1 + \mu_0) & -\alpha_n \mu_0 s \\
        \alpha_n \mu_0 s & s^2 (1 + \mu_0) -\alpha_n^2
     \end{pmatrix} \Rightarrow
\end{align*}

\begin{equation}
    M(s) = \begin{pmatrix}
        s^2 -\alpha_n^2(1 + \mu_0) & -\alpha_n \mu_0 s \\
        \alpha_n \mu_0 s & s^2 (1 + \mu_0) -\alpha_n^2
     \end{pmatrix}
\end{equation}

Знайдемо тепер $M^{-1}(s) = \frac{\widetilde{M(s)}}{det[M(s)]}$.
\begin{equation}
    \widetilde{M(s)} = \begin{pmatrix}
        s^2 (1 + \mu_0) -\alpha_n^2 & \alpha_n \mu_0 s \\
        -\alpha_n \mu_0 s & s^2 -\alpha_n^2(1 + \mu_0)
     \end{pmatrix}
\end{equation}
\begin{align}
    &det[M(s)] = \begin{vmatrix}
        s^2 - \alpha_n^2 - \alpha_n^2\mu_0 & -\alpha_n \mu_0 s \\
        \alpha_n \mu_0 s & s^2 (1 + \mu_0) -\alpha_n^2
     \end{vmatrix} = \nonumber \\
    &=(1+\mu_0)(s - \alpha_n)^2(s + \alpha_n)^2
\end{align}
Де $\alpha_n$, $-\alpha_n$, корені $det[M(s)]=0$, детальне знаходження яких наведено в (\nameref{ap_B}).

Враховучи це, тепер знайдемо значення фундаментальної матрицю за допомогою теореми про лишки:
\begin{align*}
    &\frac{1}{2\pi i} \oint_C e^{sy} M^{-1}(s)ds = \frac{2 \pi i}{2 \pi i (1 + \mu_0)} \sum_{i=1}^{2} Res\left[ e^{sy} \frac{\widetilde{M(s)}}{det[M(s)]} \right] = \\
    & = \frac{1}{(1 + \mu_0)} \left(Y_0(y) + Y_1(y) \right)
\end{align*}
Знайдемо $Y_0(y)$:
\begin{align}
    &Y_0(y) =  \frac{\partial}{\partial s} \left( \frac{e^{sy}}{(s+\alpha_n)^2} \widetilde{M(s)} \right) \Big|_{s=\alpha_n} = \nonumber \\
    &=\frac{e^{\alpha_n y}}{4\alpha_n} \begin{pmatrix}
    \alpha_n \mu_0 y + 2 + \mu_0 & \alpha_n \mu_0 y \\
    -\alpha_n \mu_0 y & -\alpha_n \mu_0 y + 2 + \mu_0
    \end{pmatrix}
\end{align}
Знайдемо $Y_1(y)$:
\begin{align}
    &Y_1(y) = \frac{\partial}{\partial s} \left(\frac{e^{sy}}{(s-\alpha_n)^2} \widetilde{M(s)} \right) \Big|_{s=-\alpha_n} = \nonumber \\
    =&\frac{e^{-\alpha_n y}}{4\alpha_n} \begin{pmatrix}
    \alpha_n \mu_0 y - 2 - \mu_0 & -\alpha_n \mu_0 y \\
    \alpha_n \mu_0 y & -\alpha_n \mu_0 y - 2 - \mu_0
    \end{pmatrix}
\end{align}

Таким чином можна записати розв'язок задачі у просторі трансформант:
\begin{equation}
    Z_n(y) = \frac{1}{1 + \mu_0} \left( Y_0(y) * \begin{pmatrix} c_1 \\ c_2 \end{pmatrix} +  Y_1(y) * \begin{pmatrix} c_3 \\ c_4 \end{pmatrix}  \right)
\end{equation}
Залишилось знайти невідомі коєфіцієнти $c_1$, $c_2$, $c_3$, $c_4$, використовуючи граничні умови (\ref{transf_bound_mat_static_1}).
Покрокове знаходження коєфіцієнтів наведено у (\nameref{ap_E}).
Таким чином можна записати розв'зок у просторі трансформант:
\begin{align}\label{transf_sol_u_static_1}
    &u_n(y) = \frac{e^{\alpha_n y}}{4 \alpha_n (1 + \mu_0)} \left[c_1 (\alpha_n \mu_0 y + 2 + \mu_0) + c_2 (\alpha_n \mu_0 y) \right] + \nonumber \\
    &\quad + \frac{e^{-\alpha_n y}}{4 \alpha_n (1 + \mu_0)} \left[c_3 (\alpha_n \mu_0 y - 2 - \mu_0) + c_4 (-\alpha_n \mu_0 y)\right]
\end{align}
\begin{align}\label{transf_sol_v_static_1}
    &v_n(y) = \frac{e^{\alpha_n y}}{4 \alpha_n (1 + \mu_0)} \left[c_1 (-\alpha_n \mu_0 y) + c_2 (-\alpha_n \mu_0 y + 2 + \mu_0) \right] + \nonumber \\
    &\quad + \frac{e^{-\alpha_n y}}{4 \alpha_n (1 + \mu_0)} \left[c_3 (\alpha_n \mu_0 y) + c_4 (-\alpha_n \mu_0 y - 2 - \mu_0)\right]
\end{align}

Викорустовуючи обернене інтегральне перетворення Фур'є до розв'язку задачі у просторі трансформант
(\ref{transf_sol_u_static_1}), (\ref{transf_sol_v_static_1}), отримаємо фінальний розв'язок задачі
\begin{equation}
    u(x,y) = \frac{2}{a} \sum_{n=1}^{\infty} u_n(y) sin(\alpha_n x), \quad \alpha_n = \frac{\pi n}{a}
\end{equation}
\begin{equation}
    v(x,y) = \frac{v_0(y)}{a} + \frac{2}{a} \sum_{n=1}^{\infty} v_n(y) cos(\alpha_n x), \quad \alpha_n = \frac{\pi n}{a}
\end{equation}

Останній крок це знаходження $v_0(y)$ у випадку коли $n=0$, $\alpha_n =0$.
Для цього повернемся до другого рівняння (\ref{transf_static_1}), та запишем його для цього випадку:
\begin{equation}\label{transf_v_0_static_1}
    (1 + \mu_0) v_n^{''}(y) = 0
\end{equation}
Та граничні умови:
\begin{equation}\label{transf_bound_v_0_static_1}
    \begin{cases}
        (2G + \lambda)v_0^{'}(y)|_{y=b} = -p_0 \\
        v_0(y)|_{y=0} = 0
    \end{cases}
\end{equation}
Де $p_0 = \int_{0}^{a}p(x)dx$

Розв'язок рівняння (\ref{transf_v_0_static_1}):
\begin{equation}
    v_0(y) = c_1 + c_2 y
\end{equation}
Застовоючи граничні умови (\ref{transf_bound_v_0_static_1}) для знаходження коєфіцієнтів $c_1$, $c_2$, отримаємо розв'язок задачі задачі:
\begin{equation}
    v_0(y) = \frac{-p_0}{(2G + \lambda)}y
\end{equation}
Тепер остаточний розв'зок задачі можна записати у вигляді:
\begin{equation}
    \begin{cases}
        u(x,y) = \frac{2}{a} \sum_{n=1}^{\infty} u_n(y) sin(\alpha_n x), \quad \alpha_n = \frac{\pi n}{a} \\
        v(x,y) = \frac{-p_0}{(2G + \lambda)a}y + \frac{2}{a} \sum_{n=1}^{\infty} v_n(y) cos(\alpha_n x), \quad \alpha_n = \frac{\pi n}{a}
    \end{cases}
\end{equation}

\subsubsection{Чисельні розрахунки}
Наведені чисельні експеренти розглядаються для сталі ($E=200$ ГПА, $\mu=0.25$).

Розглянута прямокунта область $0 \le x \le 10$, $0 \le y \le 15$, при функції навантаження $p(x)=(x-2.5)^2$.
На малюнках (Рис: \ref{static_1_u_1}), (Рис: \ref{static_1_v_1}), (Рис: \ref{static_1_sigma_x_1}), (Рис: \ref{static_1_sigma_y_1})
представлені функіі переміщень $u(x,y)$, $v(x,y)$ та напружень $\sigma_x(x,y)$, $\sigma_y(x,y)$ відповідно.
\begin{figure}[h!]
    \begin{center}
        \includegraphics[width=0.49\textwidth, scale=1]{images/results/static_1/function_u_1.png}
        \includegraphics[width=0.49\textwidth, scale=1]{images/results/static_1/function_u_2.png}
        \caption{Функція $u(x, y)$}\label{static_1_u_1}
    \end{center}
\end{figure}
\newpage
\begin{figure}[h!]
    \begin{center}
        \includegraphics[width=0.49\textwidth, scale=1]{images/results/static_1/function_v_1.png}
        \includegraphics[width=0.49\textwidth, scale=1]{images/results/static_1/function_v_2.png}
        \caption{Функція $v(x, y)$}\label{static_1_v_1}
    \end{center}
\end{figure}
\begin{figure}[h!]
    \begin{center}
        \includegraphics[width=0.49\textwidth, scale=1]{images/results/static_1/function_sigma_x_1.png}
        \includegraphics[width=0.49\textwidth, scale=1]{images/results/static_1/function_sigma_x_2.png}
        \caption{Функція $\sigma_x(x, y)$}\label{static_1_sigma_x_1}
    \end{center}
\end{figure}
\begin{figure}[h!]
    \begin{center}
        \includegraphics[width=0.49\textwidth, scale=1]{images/results/static_1/function_sigma_y_1.png}
        \includegraphics[width=0.49\textwidth, scale=1]{images/results/static_1/function_sigma_y_2.png}
        \caption{Функція $\sigma_y(x, y)$}\label{static_1_sigma_y_1}
    \end{center}
\end{figure}
