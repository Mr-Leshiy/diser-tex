\textbf{Актуальність роботи.} 
Прямокутна пружна область є однім з найбільш простих об'єктів для аналізу і моделювання у механіці пружного тіла.
Для багатьох застосувань прямокутна форма може бути використана як апроксімація більш складних об'єктів, наприклад:
в інженерних розрахунках прямокутні пластини використовують для моделювання деталів конструкцій з більш складними формами,
такими як пластини з отворами та вирізами. Прямокутна форма може бути застосована для різних типів задач, а саме:
для моделювання пружного деформування у твердих тілах, дослідження розподілу напружень та деформацій у матеріалах,
а також для розв'язання задач пов'язаних з пружністю у біологічних та геологічних системах.
Саме тому, завдяки широкому спектру застосувань, розробка аналітичних методів розв'язання задач для прямокутної пружної області залишається актуальною задачею.

% абзац хороший, но не сюда

% У даній роботі запропоновано використання аналітичного методу, що базується на методі інтегральних перетворень.
% Цей підхід дозволяє побудувати аналітичний розв'язок вихідної крайової задачі у просторі трансформант
% відносно трансформант шуканих переміщень. Для спрощення обчислень побудовано матрицю-функцію Гріна,
% яку зображено як комбінацію фундаментальних базісних матриць.
% Цей підхід дозволив отримати аналітичні подання напружень та переміщень прямокутної області та встановити важливі закономірності іх розподілу.

Аналіз літератури виявив достатню кількість мішаних задач пружності для прямокутника, 
які розв'язані за допомогою аналітичних та числових методів.
Незважаючи на це, питання поведінки механічних характеристик у кутових точках,
питання встановлення якісної поведінки хвильових полів усередині прямокутної області за умов динамічного навантаження залишається відкритим.
Цим обумовлено актуальність запропонованного дослідження, що полягає у розробці нового аналітичного підходу до розв'язання мішаних задач для прямокутної області.

\textbf{Мета і задачі дослідження.}
Метою цього дослідження є розробка нового підходу до розв'язання плоских мішаних задач теорії пружності для прямокутної області,
що надає можливість встановити важливі особливості розподілу напружень та переміщень.

Для досягнення поставленої мети необхідно виконати наступні завдання:
\begin{enumerate}
    \item розвиток методики, яка використовує застосування методу інтегральних перетворень разом з методоми розв'язання векторних крайових задач теорії пружності та використання матриці-фукції Гріна. 
    \item побудова аналітичного розв'язку задачі для прямокутної області, що піддається впливу зовнішнього статичного навантаження за умови виконання різних граничних умов на її бокових торцях.
    \item розв'язання динамічної задачі пружності для прямокутної області з метою встановлення закономірностей розподілу хвильових полів та динамічних напружень.
\end{enumerate}

\textit{\textbf{Об’єктом дослідження є}}
пружна прямокутна область під впливом зовнішнього навантаження різної природи (статичного та динамічного).

\textit{\textbf{Предметом дослідження є}}
закономірності зміни напружено-деформованого стану та хвильового поля прямокутної області в залежності від видів навантаження та крайових умов.

\textbf{Методи дослідження.}
У дисертаційній роботі розв'язання динамічних та статичних задач теорії пружності для прямокутної області було проведено методом інтегральних перетворень,
який застосовано безпосередньо до рівнянь рівноваги. 
Для розв'язання векторної крайової задачі у просторі трансформант побудовано матричну функцію Гріна.
Отримані в роботі сінгулярні інтегральні рівняння розв'язані за допомогою методу ортогональних многочленів,
з метою урахування реальної особливості невідомою функції на кінцях інтервалів інтегрування.

\textbf{Обґрунтованість та достовірність отриманих результатів} забезпечується:
використання точних математичних формулювань задач у лінійній механіці суцільного тіла та механіці руйнування;
використання перевірених і строгих аналітичних методів для отримання розв'язків сформульованих задач;
фізичною інтерпретацією результатів розрахунків задач.
Отримані результати збігаються з відомими результатами теоретичних досліджень.

\textbf{Наукова новизна} отриманих результатів полягає в наступному:
\begin{itemize}
    \item вперше застосовано нову методику розв'язання динамічних та статичних задач теорії пружності для прямокутної області,
    що ґрунтується на безпосередньому перетворенні рівнянь Ламе. Цей підхід дозволив отримати аналітичні подання для полів переміщень та напружень;
    \item побудовано матричну функцію Гріна, що дозволило звести вихідні задачі для прямокутної області до розв'язання сінгулярних інтегральних рівнянь.
    Встановлено нові особливості залежності полів переміщень та напружень від параметрів навантаження та крайових умов на торцях прямокутної області;
    \item отримано аналітичні розв'язки динамічної задачі пружності для прямокутної області та досліджено залежність хвильових полів від типу навантаження та геометричних розмірів області
\end{itemize}

\textbf{Особистий внесок здобувача.}
Основні результати дисертаційної роботи отримано здобувачем самостійно.
У роботах у співавторстві \cite{pozhylenkov_1, pozhylenkov_2, pozhylenkov_3, pozhylenkov_4, pozhylenkov_5, pozhylenkov_6},
науковому керівнику належить постановка задач, вибір методики їх розв’язання.
Дисертантом проведено огляд літератури, виконано усі математичні перетворення при побудові розв’язків,
здійснено програмну реалізацію та проведено аналіз отриманих результатів.

\textbf{Апробація результатів дисертації.}

Результати досліджень, які були включені до дисертаційної роботи, були представлені та обговорені на міжнародних наукових конференціях різного рівня:
конференція  <<Актуальные вопросы и перспективы развития транспортного и строительного комплексов>> (Гомель, 2018),
Х Міжнародна наукова конференція <<Математичні проблеми механіки неоднорідних структур>> (Львів, 2019),
25-th international conference <<Engineering Mechanics 2019>> (Svratka, 2019),
<<1st Virtual European Conference on Fracture>> (Italy, 2020),
<<26th International Conference on Fracture and Structural Integrity>> (Turin, 2021),
<<10th International Conference on Wave Mechanics and Vibrations>> (Lisbon, 2022).

У повному обсязі робота доповідалась на
\begin{itemize}
    \item науковому семінарі «Мішані задачі математичної фізики» кафедри методів
    математичної фізики Одеського національного університету імені І.І. Мечникова під керівництвом к.ф.-м.н., доц. Ю.С. Процерова.
\end{itemize}

\textbf{Публікації.}
Основні наукові положення дисертаційного дослідження відображено у 6 публікаціях з яких:
дві статті \cite{pozhylenkov_2,pozhylenkov_3} опубліковано у провідних фахових виданнях України, що входять у перелік ДАК України,
статті \cite{pozhylenkov_1,pozhylenkov_4,pozhylenkov_5,pozhylenkov_6} прореферовано у міжнародній наукометричній базі Scopus.

% Новый раздел
\textbf{Структура і обсяг дисертації.}
Робота складається зі вступу, 4 розділів, висновків, списку використаної літератури, що включає ?? найменування. Загальний обсяг дисертації становить ?? сторінок, із них ?? сторінок основного тексту. Робота містить ?? рисунків та ?? таблицю.

У \textit{вступі} аргументовано актуальність теми десертації; 
поставлено мету та завдання дослідження;
наведено методи розв'язання поставлених задач;
розкрито новизну і достовірність отриманих результатів, їх практичне та теоритичне значення;
представлено відомості про апробацію роботи, публікації та особоистий внесок здобувача.

У \textit{першому розділі} проведено огляд наукових робіт, що мають мають відношення до розв'язання плоских мішаних задач теорії пружності для прямокутної області.
Проаналізовано праці та внесок автора у вивчення наукової проблеми, на яку спрямована дана робота.
Показано, що таматика данної дисертаційної роботи є актуальною.

У \textit{другому розділі} сформульовано загальну постановку плоских мішаних динамічних задач теорії пружності для прямокутної області,
продемонстровано загальну методику побудови розв'язку.
Наведено побудова точного розв’язку крайової задачі у просторі завдяки використанню інтегрального скінченного sin- та cos-перетворення Фур'є.
Загальний розв’язок векторного однорідного рівняння побудовано за допомогою фундаментальної матричної системи розв’язків для відповідного однорідного матричного рівняння, що отримана методами контурного інтегрування.
Для отримання часткового розв’язку векторного неоднорідного рівняння побудовано матрицю-функцію Гріна, яку зображено як комбінацію фундаментальних базісних матриц.
Приведено схему розв’язання отриманного сінгулярного інтегрального рівняння методом ортогональних поліномів.

У \textit{третьому розділі} розглянуто динамічну та статичну задачу теорії пружності за умов ідеального контакту на бічних гранях.
Розв'язок поставленної задачі побудовано за вищеописаною методикою.
Задачу зведено до одновимірної крайової задачі у просторі трансформант з однорідними рівнянями рівноваги за допомогою використанню інтегрального скінченного sin- та cos-перетворення Фур'є.
Одновімірну задачу переформульовано у матрично-векторному виді, розв'зок якої знайдено за допомогою фундаментальної матричної системи розв’язків.
Застосування оберненного перетворення Фур'є завершує побудову вихідної крайової задачі.
Наведено аналіз напруженого стану прямокутної області за різних типів навантаження та різних геометричних розмірів області.

У \textit{четвертому розділі} розглянуто динамічну та статичну задачу теорії пружності за умов другої основної задачі теорії пружності на бічних гранях.
Розв'язок поставленної задачі побудовано за вищеописаною методикою.
Задачу зведено до одновимірної крайової задачі у просторі трансформант за допомогою використанню інтегрального скінченного sin- та cos-перетворення Фур'є.
Одновімірну задачу переформульовано у матрично-векторному виді, розв'зок якої знайдено як суперпозицію загального розв'язку однорідного векторного рівняння та частинного розв'язку неоднорідного рівняння.
Загальний та частинний розв'язки знайденні за методикою яка описана вище.
Розв'язано отримане сінгулярне інтегральне рівняння за допомогою метода ортогональних поліномів.
Наведено аналіз напруженого стану прямокутної області за різних типів навантаження та різних геометричних розмірів області.

У \textit{висновках}
сформульовано отримані результати та наведено основні якісні залежності хвильового поля навантажень та напружень
від геометричних параметрів прямокутної області, хакрактеру та частоти навантаження.

\textbf{Зв’язок роботи з науковими програмами, планами, темами.}
Дисертаційна робота виконана в рамках держбюджетних тем Одеського національного університету імені І. І. Мечникова
«Статичні та динамічні задачі для тіл канонічної форми з дефектами»
(2021-2024 рр., реєстраційний номер 0121U111664).

\textbf{Теоретичне і практичне значення одержаних результатів.} 
Запропонована методика для розв'язання мішаних динамічних задач теорії пружності для прямокутної області має теоретичне значення для подальшого розвитку математичних методів вирішення плоских задач теорії пружності.
Отримані результати стали складовою частиною курса "Теорія пружності" та були використанні студентами, які навчаються за спеціальністю "Прикладна математика", під час виконання магістерських робіт.
Отримані результати також можуть знайти застосування в геомеханіці, будівництві конструкцій, вивченні міцності елементів транспортних засобів та визначенні їх безпечності та в інших сферах.
