\textbf{Актуальність роботи.} 
Прямокутна пружна область є однією з найбільш простих для аналізу і моделювання у механіці пружного тіла.
Але для багатьох застосувань прямокутна форма може бути використана як апроксімація більш складних об'єктів, наприклад:
в інженерних розрахунках прямокутні пластини використовують для моделювання деталів конструкцій з більш складними формами,
такими як пластини з отворами та вирізами. Прямокутна форма може бути застосована для різних типів задач, а саме:
для моделювання пружного деформування у твердих тілах, дослідженню розподілу напружень та деформації у матеріалах,
а також для розв'язання задач пов'язанних з пружністю і біологічних та геологічних системах.
Саме тому розробка математичних аналітичних методів розв'язання задач для прямокутної пружної області залишаеться актуальної за своєю простотою,
універсальністю, практичністю на широкому спектрі застосувань, незважаючи на те, що існують більш складні та реалістичні моделі.

У даній роботі запропоновано використання інтегральних перетворень безпосередньо до рівнянь руху.
Цей підхід дозволяє побудувати аналітичний розв'язок вихідної крайової задачі у просторі трансформант
відносно шуканих переміщень. Для спрощення обчислень побудовано матрицю-функцію Гріна,
яку зображено як комбінацію фундаментальних базісних матриць.
Цей підхід був використано під час розв'язання динамічних та статичних мішаних задач теорії пружності для прямокутної області,
яка є модельним об'єктом для вивчення напружено-деформованого стану пружних матеріалів.

Незважаючи на те, що згідно з аналізом літератури, було встановлено досить широке коло дослідженних мішаних задач теорії пружності,
залишається наявність невирішених питань, що пов'язані з поведінкою напружень у кутових точках та їх розподілам у прямокутній області.
Залишається необхідним отримати загальну якісну картину поведінки хвильових полів, за умов динамічного навантаження.
Тому розробка нового аналітичного підходу до розв'язання плоских мішаних задач для прямокутної області залишається актуальною задачою.сті для прямокутної області.

\textbf{Зв’язок роботи з науковими програмами, планами, темами.}
Дисертаційна робота виконана в рамках держбюджетних тем Одеського національного університету імені І. І. Мечникова
«Статичні та динамічні задачі для тіл канонічної форми з дефектами»
(2021-2024 рр., реєстраційний номер 0121U111664).

\textbf{Мета і задачі дослідження.}
Головною метою цього дослідження є визначення взаємозв'язку між напружено-деформованим станом прямокутної області та різними крайовими умовами на його торцях.

Для досягнення поставленої мети необхідно виконати наступні завдання:
\begin{enumerate}
    \item розробка методики розв'язання плоских задач для прямокутної області, яка використовує безпосередні перетворення рівнянь рівноваги,
    спрямована на отримання розв'язку поставленої задачі для реальних механічних характеристик;
    \item побудові аналітичного розв'язку для задачі прямокутної області, яка піддається зовнішньому навантаженню різної конфігурації.
    Цей розв'язок отримується шляхом перетворення початкової задачі до одновимірної крайової задачі та використання матричної функції Гріна для її розв'язання.
    Основною метою є встановлення якісних та кількісних закономірностей для полів переміщень та напружень в цій задачі;
\end{enumerate}

\textit{\textbf{Об’єктом дослідження є}}
пружна прямокутна область під впливом зовнішнього навантаження різної природи (статичного та динамічного).

\textit{\textbf{Предметом дослідження є}}
закономірності зміни напружено-деформований стану та хвильового поля прямокутної області в залежності від видів навантаження.

\textbf{Методи дослідження.}
У даній дисертаційній роботі було розглянуто розв'язання мішаних динамічних та статичних задач теорії пружності для прямокутної області з використанням методу інтегральних перетворень,
що безпосередньо застосовується до рівнянь рівноваги. Цей підхід дозволяє зведення вихідної задачі до одновимірної крайової задачі у просторі трансформант,
відносно невідомих трансформант переміщень. Для розв'язання цієї задачі використовується матричне диференціальне числення та матриці-функції Гріна.
Задачу можна звести до одного сингулярного інтегрального рівняння. Для розв'язання якого застосовується спеціальний аналітично-числовий метод, який дозволяє враховувати рухомі та нерухомі особливості у ядрі інтегрального рівняння.

\textbf{Обґрунтованість та достовірність отриманих результатів} забезпечується:
використання точних математичних формулювань задач у лінійній механіці суцільного тіла та механіці руйнування;
використання перевірених і строгих аналітичних методів для отримання розв'язків сформульованих задач;
фізичною інтерпретацією результатів розрахунків задач.
Отримані результати збігаються з відомими результатами теоретичних досліджень.

\textbf{Наукова новизна} отриманих результатів полягає в наступному:
\begin{itemize}
    \item вперше була застосована нова методика розв'язання мішаних динамічних плоских задач теорії пружності для прямокутної області,
    що ґрунтується на безпосередньому перетворенні рівнянь рівноваги. Цей підхід дозволяє отримати аналітичні вирази для шуканих механічних характеристик;
    \item за допомогою побудови матриці-функції Гріна у формі комбінації фундаментальних базісних матриць та використання методу матричного диференціального числення,
    були отримані аналітичні розв'язки для мішаних плоских задач. В цих розв'язках задача була зведена до одного сингулярного інтегрального рівняння, що залежить від невідомого стрибка переміщень.
    Були встановлені особливості залежності полів переміщень та напружень від параметрів навантаження на короткому торці прямокутної області;
\end{itemize}

\textbf{Теоретичне і практичне значення одержаних результатів.} 
Запропонована методика для розв'язання мішаних динамічних задач теорії пружності для скінченної прямокутної області має важливе теоретичне значення для подальшого розвитку математичних методів вирішення плоских задач теорії пружності.
Отримані результати стали складовою частиною курсів "Теорія пружності" та "Додаткові глави методів математичної фізики" і були використані студентами, які навчаються за спеціальністю "Прикладна математика", у написанні їх магістерських і дипломних робіт.
Отримані результати також можуть знайти застосування в геомеханіці, будівництві конструкцій, вивченні міцності елементів транспортних засобів та визначенні їх безпечності та в інших сферах.

\textbf{Особистий внесок здобувача.}
Основні результати дисертаційної роботи отримано здобувачем самостійно.
У роботах у співавторстві \cite{pozhylenkov_1, pozhylenkov_2, pozhylenkov_3, pozhylenkov_4, pozhylenkov_5, pozhylenkov_6},
науковому керівнику належить постановка задач, вибір методики їх розв’язання.
Дисертантом проведено огляд літератури, виконано усі математичні перетворення при побудові розв’язків,
здійснено програмну реалізацію та проведено аналіз отриманих результатів.

\textbf{Апробація результатів дисертації.}

Результати досліджень, які були включені до дисертаційної роботи, були представлені і обговорені на різних міжнародних наукових конференціях:
\begin{itemize}
    \item конференція  <<Актуальные вопросы и перспективы развития транспортного и строительного комплексов>> (Білорусь, Гомель, 2018);
    \item Х Міжнародна наукова конференція <<Математичні проблеми механіки неоднорідних структур>> (Львів, 2019);
    \item 25-th international conference <<Engineering Mechanics 2019>> (Czech Republic, Svratka, 2019);
    \item <<1st Virtual European Conference on Fracture>> (Italy, 2020);
    \item <<26th International Conference on Fracture and Structural Integrity>> (Italy, Turin, 2021);
    \item <<10th International Conference on Wave Mechanics and Vibrations>> (Portugal, Lisbon, 2022)
\end{itemize}

\textbf{Публікації.}
Основні наукові положення дисертаційного дослідження відображено у 6 публікаціях,
дві статті \cite{pozhylenkov_2,pozhylenkov_3} опубліковано у провідних фахових виданнях України, що входить у перелік ДАК України,
статті \cite{pozhylenkov_1,pozhylenkov_4,pozhylenkov_5,pozhylenkov_6} прореферовано у міжнародній наукометричній базі Scopus.

\textbf{Структура і обсяг дисертації.}
???
Приблизний текст (Робота складається зі вступу, 5 розділів, висновків, списку використаної літератури, що включає 173 найменування. Загальний обсяг дисертації становить 160 сторінок, із них 119 сторінок основного тексту. Робота містить 68 рисунків та 1 таблицю.)

\textbf{Подяка.}
Автор висловлює глибоку подяку своєму першому вчителеві професору та науковому керівнику Н. Д. Вайсфельд, який сприяв виникненню його інтересу до математичних проблем механіки та визначив напрям наукових досліджень.
Дисертант висловлює щиру вдячність кандидату фізико-математичних наук, доценту Ю. С. Процерову за цінні наукові поради, що допомогли успішному проведенню досліджень.