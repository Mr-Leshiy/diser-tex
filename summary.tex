В дисертаційній роботі досліджено напружений стан та хвильові поля плоскої прямокутної області
під дією статичного та динамічного навантажень. За результатами дослідженя:
\begin{enumerate}
    \item Отримано методику аналітичного розв'язання задач теорії пружності для прямокутної області,
    що базується на застосувані методу інтегральних перетворень безпосередньо до рівнянь Ламе.
    Це дозволяє уникнути використання допоміжних функцій та сформулювати розв'язок у термінах механічних характеристик.

    \item Встановлено закономірності зміни напруженого стану прямокутної області в залежності від різних типів навантажень та різних типів граничних умов,
    які задано по її боковим торцях 

    \item Побудовано хвильові поля прямокутної пружної області та встановлено власні частоти тіла в залежності від типу динамічного навантаження на бокових торцях та геометричних розмірів області.
\end{enumerate}

Ці результати дозволили встановити такі особливості поведінки полів переміщень та напружень:
\begin{enumerate}
    \item 
\end{enumerate}