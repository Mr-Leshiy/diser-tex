У даному розділі досліджено плоскі статична та динамічна задачі теорії пружності для прямокутної області
за умов другої основної задачі теорії пружності на бічних гранях.

% Вихідні задачі зведено до одновимірної задачі у просторі трансформант за допомогою інтегрального перетворення Фур'є.
% Розв'зок одновимірної задачі у просторі трансформант знайдено як суперпозиція загального розв'язку векторного однорідного рівняння
% та часткового розв'язку неоднорідного рівняння.
% Загальний розв'язок векторного однорідного рівняння знайдено за допомогою методу матричного диференціального числення.
% Частковий розв'язок неоднорідного рівняння знайдено за допомогою матричної функції Гріна.
% Остаточне подання для полів переміщень та напружень наведено за допомогою оберненного перетворення Фур'є.

% Проведено чисельний аналіз отриманих функцій переміщень та напружень для різних розмірів прямокутної області та різних видів навантаження.

Результати розділу опубліковані в \cite{pozhylenkov_4}, \cite{pozhylenkov_6}, а також доповідалась на конференції \cite{conf_3}, \cite{conf_5}.