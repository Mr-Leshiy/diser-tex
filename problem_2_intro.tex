У даному розділі досліджено плоска статична задача теорії пружності для прямокутної області,
за умов другої основної задачі теорії пружності на бічних гранях.

Вихідна задача зведена до одновимірної задачі у просторі трансформант за допомогою інтегрального перетворення Фур'є.
Отримана крайова задача розв'язана точно за допомогою методу матрично диференціального числення,
фундаментальний розв'язок представлений як інтеграл по замкненому контору, який в свою чергу, був знайденний за допомогою теоремі про лишки.
Побудована матриця-функція Гріна як комбінація фундаментальних базисних розв'язків задачі у просторі трансформант.
Остаточний вигляд для функцій переміщеннь та напружень отриман шляхом оберненого перетворення Фур'є.
Побудовано та розв'язано сінгульрне інтегральне рівняння відносно невідомої функції шляхом викорстання методу ортагональних многочленів, та зведення рівнняння до бескінечної алгебричної системи,
яка в подальшому була розв'язана методом редукціі \cite{popov_3}.

Проведено чисельний аналіз отриманих функцій переміщень та напружень для різних розмірів прямокутної області та різних видів навантаження.

Результати розділу опубліковані в \cite{pozhylenkov_4}, \cite{pozhylenkov_6}, а також доповідалась на конференції \cite{conf_3}, \cite{conf_5}.