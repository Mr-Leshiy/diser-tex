У даному розділі наведено опис аналітичного апарату, який використовується для розв'язання мішаних задач теорії пружності для прямокутної області.
Цей підхід базується на результати раніше проведених досліджень, зокрема робіт \cite{popov_1} і \cite{popov_2}.
Розглянута методика розв'язання мішаних плоских задач ґрунтується на застосуванні інтегральних перетворень безпосередньо до системи рівнянь рівноваги Ламе та крайових умов.
Це дозволяє зводити вихідну задачу до векторної одновимірної крайової задачі.
Векторна одновимірна крайова задача точно розв'язується за допомогою матричного диференційного числення та матричної функції Гріна.

\subsection{Зведення вихідної крайової задачі до одновимірної векторної крайової задачі}
Розглядається пружна прямокутна область, яка займає область, що у декартовій системі координат описується співвідношенням $0 \le x \le a$, $0 \le y \le b$.

До прямокутної області на грані $y=b$ додане нормальне навантаження
\begin{equation}
    \sigma_y(x, y, t) |_{y=b} = -p(x, t), \quad  \tau_{xy}(x,y,t) |_{y=b} =0, \quad 0 \le x \le a
\end{equation}
де $p(x, t)$ відома функція.
На нижній грані виконуються наступні умови
\begin{equation}
    v(x,y,t) |_{y=0}, \quad \tau_{xy}(x,y,t) |_{y=0} =0
\end{equation}
На бічних гранях $x=0$ та $x=a$ граничні умови запишемо у формі
\begin{equation}\label{gen_bound_gen}
    U_1[f(x,y,t)]=0, \quad U_2[f(x,y,t)]=0 , \quad 0 \le y \le b
\end{equation}
Де 
\begin{align*}
    &U_1[f(x,y,t)]=\left[\alpha_1f(x,y,t) + \beta_1 \frac{\partial f(x,y,t)}{\partial x} \right]|_{x=0} \\
    &U_2[f(x,y,t)]=\left[\alpha_2f(x,y,t) + \beta_2 \frac{\partial f(x,y,t)}{\partial x} \right]|_{x=a} \\
\end{align*}
граничні функціонали у загальному виді (для кожної конкретної задачі вони будуть деталізовані), $f(x,y,t)=(u(x,y,t), v(x,y,t))^T$ - вектор переміщеннь.

Розглядаються наступні рівняння рівноваги Ламе:
\begin{equation}
    \begin{cases}
        \frac{\partial^2 u(x,y,t)}{\partial x^2} + \frac{\partial^2 u(x,y,t)}{\partial y^2} + \mu_0 (\frac{\partial^2 u(x,y,t)}{\partial x^2} + \frac{\partial^2 v(x,y,t)}{\partial x\partial y}) = \frac{1}{c_1^2} \frac{\partial^2 u(x,y,t)}{\partial t^2} \\
        \frac{\partial^2 v(x,y,t)}{\partial x^2} + \frac{\partial^2 v(x,y,t)}{\partial y^2} + \mu_0 (\frac{\partial^2 u(x,y,t)}{\partial x \partial y} + \frac{\partial^2 v(x,y,t)}{\partial y^2}) = \frac{1}{c_2^2} \frac{\partial^2 v(x,y,t)}{\partial t^2} \\
    \end{cases}
\end{equation}

Будемо розглядати випадок гармонічних коливань, тому можемо предствавити функції у наступному вигляді:
\begin{equation}
    u(x,y,t) = u(x,y) e^{i \omega t}, \quad v(x,y,t) = v(x,y) e^{i \omega t}, \quad p(x,t) = p(x) e^{i \omega t}
\end{equation}
Таким чином отримаємо наступні рівняння рівноваги:
\begin{equation}\label{lame_gen}
    \begin{cases}
        \frac{\partial^2 u(x,y)}{\partial x^2} + \frac{\partial^2 u(x,y)}{\partial y^2} + \mu_0 (\frac{\partial^2 u(x,y)}{\partial x^2} + \frac{\partial^2 v(x,y)}{\partial x\partial y}) = -\frac{\omega^2}{c_1^2}  u(x,y) \\
        \frac{\partial^2 v(x,y)}{\partial x^2} + \frac{\partial^2 v(x,y)}{\partial y^2} + \mu_0 (\frac{\partial^2 u(x,y)}{\partial x \partial y} + \frac{\partial^2 v(x,y)}{\partial y^2}) = -\frac{\omega^2}{c_2^2} v(x,y) \\
    \end{cases}
\end{equation}
Та граничні умови:
\begin{equation}\label{bound_gen}
    \begin{cases}
        \sigma_y(x, y) |_{y=b} = -p(x, t), \quad  \tau_{xy}(x,y) |_{y=b} =0 \\
        v(x,y) |_{y=0}, \quad \tau_{xy}(x,y) |_{y=0} =0 \\
        U_1[f(x,y)]=0, \quad U_2[f(x,y)]=0
    \end{cases}
\end{equation}

Введемо невідомі функції $\chi_1(y) = u(0, y)$, $\chi_2(y) = v(0, y)$, $\chi_3(y) = u(a, y)$, $\chi_4(y) = v(a, y)$.
Враховучи умову (\ref{gen_bound_gen}), отримаємо, що 
$\frac{\partial u(0, y)}{\partial x}=-\frac{\alpha_1}{\beta_1} \chi_1(y)$,
$\frac{\partial v(0, y)}{\partial x}=-\frac{\alpha_1}{\beta_1} \chi_2(y)$,
$\frac{\partial u(a, y)}{\partial x}=-\frac{\alpha_2}{\beta_2} \chi_3(y)$,
$\frac{\partial v(a, y)}{\partial x}=-\frac{\alpha_2}{\beta_2} \chi_4(y)$.
Отже умова (\ref{gen_bound_gen}) виконується автоматично.

\subsection{Зведеня задачі до одновимірної у просторі трансформант}
Для того, щоб звести задачу до одновимірної задачі, використаєм інтегральне перетворення Фур'є по змінній $x$ у до рівнянь (\ref{lame_1}) наступному вигляді:
\begin{equation}
    \begin{pmatrix}
        u_n(y) \\
        v_n(y)
    \end{pmatrix} = \int_{0}^{a} 
    \begin{pmatrix}
        u(x,y) sin(\alpha_n x) \\
        v(x,y) cos(\alpha_n x)
    \end{pmatrix} dx, \quad \alpha_n = \frac{\pi n}{a}, n=\overline{1, \infty}
\end{equation}

Для цього помножим перше та друге рівняння (\ref{lame_gen}) на $sin(\alpha_n x)$ та $cos(\alpha_n x)$ відповідно та проінтегруєм по змінній $x$ на інтервалі $0 \le x \le a$.
Покрокове інтегрування рівняння (\ref{lame_gen}) наведено у (\nameref{ap_A}).
Отримана система рівнянь задачі у просторі трансформант:
\begin{equation}\label{transf_gen}
    \begin{cases}
        u_n^{''}(y) - \alpha_n \mu_0 v_n^{'}(y) + (-\alpha_n^2 - -\alpha_n^2 \mu_0 + \frac{\omega^2}{c_1^2}) u_n(y) = 0 \\
        (1 + \mu_0) v_n^{''}(y) + \alpha_n \mu_0 u_n^{'}(y) + (- \alpha_n^2 + \frac{\omega^2}{c_2^2}) v_n(y) = 0 \\
    \end{cases}
\end{equation}