У даному розділі наведено опис аналітичного апарату, який використовується для розв'язання мішаних задач теорії пружності для прямокутної області.
Цей підхід базується на результати раніше проведених досліджень, зокрема робіт \cite{popov_1} і \cite{popov_2}.
Розглянута методика розв'язання мішаних плоских задач ґрунтується на застосуванні інтегральних перетворень безпосередньо до системи рівнянь рівноваги Ламе та крайових умов.
Це дозволяє зводити вихідну задачу до векторної одновимірної крайової задачі.
Векторна одновимірна крайова задача точно розв'язується за допомогою матричного диференційного числення та матричної функції Гріна.

\subsection{Постановка задачі}
Розглядається пружна прямокутна область, яка займає область, що у декартовій системі координат описується співвідношенням $0 \le x \le a$, $0 \le y \le b$.

До прямокутної області на грані $y=b$ додане нормальне навантаження
\begin{equation}
    \sigma_y(x, y, t) |_{y=b} = -p(x, t), \quad  \tau_{xy}(x,y,t) |_{y=b} =0, \quad 0 \le x \le a
\end{equation}
де $p(x, t)$ відома функція.
На нижній грані виконуються наступні умови
\begin{equation}
    v(x,y,t) |_{y=0}, \quad \tau_{xy}(x,y,t) |_{y=0} =0
\end{equation}
На бічних гранях $x=0$ та $x=a$ граничні умови запишемо у формі
\begin{equation}\label{gen_bound_gen}
    U_1[f(x,y,t)]=0, \quad U_2[f(x,y,t)]=0 , \quad 0 \le y \le b
\end{equation}
Де 
\begin{align*}
    &U_1[f(x,y,t)]=\left[\alpha_1f(x,y,t) + \beta_1 \frac{\partial f(x,y,t)}{\partial x} \right]|_{x=0} \\
    &U_2[f(x,y,t)]=\left[\alpha_2f(x,y,t) + \beta_2 \frac{\partial f(x,y,t)}{\partial x} \right]|_{x=a} \\
\end{align*}
граничні функціонали у загальному виді (для кожної конкретної задачі вони будуть деталізовані), $f(x,y,t)=(u(x,y,t), v(x,y,t))^T$ - вектор переміщеннь.

Розглядаються наступні рівняння рівноваги Ламе:
\begin{equation}
    \begin{cases}
        \frac{\partial^2 u(x,y,t)}{\partial x^2} + \frac{\partial^2 u(x,y,t)}{\partial y^2} + \mu_0 (\frac{\partial^2 u(x,y,t)}{\partial x^2} + \frac{\partial^2 v(x,y,t)}{\partial x\partial y}) = \frac{1}{c_1^2} \frac{\partial^2 u(x,y,t)}{\partial t^2} \\
        \frac{\partial^2 v(x,y,t)}{\partial x^2} + \frac{\partial^2 v(x,y,t)}{\partial y^2} + \mu_0 (\frac{\partial^2 u(x,y,t)}{\partial x \partial y} + \frac{\partial^2 v(x,y,t)}{\partial y^2}) = \frac{1}{c_2^2} \frac{\partial^2 v(x,y,t)}{\partial t^2} \\
    \end{cases}
\end{equation}

Будемо розглядати випадок гармонічних коливань, тому можемо предствавити функції у наступному вигляді:
\begin{equation}
    u(x,y,t) = u(x,y) e^{i \omega t}, \quad v(x,y,t) = v(x,y) e^{i \omega t}, \quad p(x,t) = p(x) e^{i \omega t}
\end{equation}
Таким чином отримаємо наступні рівняння рівноваги:
\begin{equation}\label{lame_gen}
    \begin{cases}
        \frac{\partial^2 u(x,y)}{\partial x^2} + \frac{\partial^2 u(x,y)}{\partial y^2} + \mu_0 (\frac{\partial^2 u(x,y)}{\partial x^2} + \frac{\partial^2 v(x,y)}{\partial x\partial y}) = -\frac{\omega^2}{c_1^2}  u(x,y) \\
        \frac{\partial^2 v(x,y)}{\partial x^2} + \frac{\partial^2 v(x,y)}{\partial y^2} + \mu_0 (\frac{\partial^2 u(x,y)}{\partial x \partial y} + \frac{\partial^2 v(x,y)}{\partial y^2}) = -\frac{\omega^2}{c_2^2} v(x,y) \\
    \end{cases}
\end{equation}
Та граничні умови:
\begin{equation}\label{bound_gen}
    \begin{cases}
        \sigma_y(x, y) |_{y=b} = -p(x), \quad  \tau_{xy}(x,y) |_{y=b} =0 \\
        v(x,y) |_{y=0}, \quad \tau_{xy}(x,y) |_{y=0} =0 \\
        U_1[f(x,y)]=0, \quad U_2[f(x,y)]=0
    \end{cases}
\end{equation}

Введемо невідомі функції $\chi_1(y) = u(0, y)$, $\chi_2(y) = v(0, y)$, $\chi_3(y) = u(a, y)$, $\chi_4(y) = v(a, y)$.
Враховучи умову \eqref{gen_bound_gen}, отримаємо, що 
$\frac{\partial u(0, y)}{\partial x}=-\frac{\alpha_1}{\beta_1} \chi_1(y)$,
$\frac{\partial v(0, y)}{\partial x}=-\frac{\alpha_1}{\beta_1} \chi_2(y)$,
$\frac{\partial u(a, y)}{\partial x}=-\frac{\alpha_2}{\beta_2} \chi_3(y)$,
$\frac{\partial v(a, y)}{\partial x}=-\frac{\alpha_2}{\beta_2} \chi_4(y)$.
Отже умова \eqref{gen_bound_gen} виконується автоматично.

\subsection{Зведеня задачі до одновимірної у просторі трансформант}
Для того, щоб звести задачу до одновимірної задачі, використаєм інтегральне перетворення Фур'є по змінній $x$ у до рівнянь \eqref{lame_gen} наступному вигляді:
\begin{equation}
    \begin{pmatrix}
        u_n(y) \\
        v_n(y)
    \end{pmatrix} = \int_{0}^{a} 
    \begin{pmatrix}
        u(x,y) sin(\alpha_n x) \\
        v(x,y) cos(\alpha_n x)
    \end{pmatrix} dx, \quad \alpha_n = \frac{\pi n}{a}, n=\overline{1, \infty}
\end{equation}

Для цього помножим перше та друге рівняння \eqref{lame_gen} на $sin(\alpha_n x)$ та $cos(\alpha_n x)$ відповідно та проінтегруєм по змінній $x$ на інтервалі $0 \le x \le a$.
Покрокове інтегрування рівняння \eqref{lame_gen} наведено у (\nameref{ap_A}).
Отримана система рівнянь задачі у просторі трансформант:
\begin{equation}\label{transf_gen}
    \begin{cases}
        u_n^{''}(y) - \alpha_n \mu_0 v_n^{'}(y) - (\alpha_n^2 + \alpha_n^2 \mu_0 - \frac{\omega^2}{c_1^2}) u_n(y) = \\
        = \alpha_n(1 + \mu_0)(\chi_3(y) cos(\alpha_n a) - \chi_1(y)) \\
        \\
        (1 + \mu_0) v_n^{''}(y) + \alpha_n \mu_0 u_n^{'}(y) - (\alpha_n^2 - \frac{\omega^2}{c_2^2}) v_n(y) = \\
        = (\frac{\alpha_2}{\beta_2}\chi_4(y) cos(\alpha_n a) - \frac{\alpha_1}{\beta_1}\chi_2(y)) - \mu_0 (\chi_3^{'}(y) cos(\alpha_n a) -\chi_1^{'}(y))
    \end{cases}
\end{equation}

Застосовуючи інтегральне перетворення до граничних умов,
отримаємо наступні умови задачі у просторі трансформант
\begin{equation}\label{transf_bound_gen}
    \begin{cases}
        \left( (2G + \lambda)v_n^{'}(y) + \alpha_n \lambda u_n(y) \right)|_{y=b} = -p_n \\
        \left(u_n^{'}(y) - \alpha_n v_n(y)  \right)|_{y=b} = 0 \\
        v_n(y)|_{y=0} = 0 \\
        \left(u_n^{'}(y) - \alpha_n v_n(y)  \right)|_{y=0} = 0
    \end{cases}
\end{equation}
Де $p_n = \int_{0}^{a} p(x) cos(\alpha_n x) dx$

\subsection{Зведення задачі у просторі трансформант до матрично-векторної форми}
Для того щоб розв'язати задачу у простосторі трансформант, перепишмо її у матрично-векторній формі.
Рівняння рівноваги \eqref{transf_gen} запишемо у наступному вигляді:
\begin{align}\label{transf_mat_gen}
    &L_2\left[ Z_n(y) \right] = A * Z_n^{''}(y) + B * Z_n^{'}(y) + C * Z_n(y) \nonumber \\
    &L_2\left[ Z_n(y) \right] = F_n(y)
\end{align}
Де
\begin{equation*}
    A = \begin{pmatrix}
        1 & 0 \\
        0 & 1 + \mu_0
    \end{pmatrix}, \quad
    B = \begin{pmatrix}
        0 & -\alpha_n \mu_0 \\
        \alpha_n \mu_0 & 0
    \end{pmatrix}
\end{equation*}
\begin{equation*}
    C = \begin{pmatrix}
        -\alpha_n^2 -\alpha_n^2 \mu_0 + \frac{\omega^2}{c_1^2} & 0 \\
        0 & -\alpha_n^2 + \frac{\omega^2}{c_2^2}
    \end{pmatrix}, \quad
    Z_n(y) = \begin{pmatrix}
        u_n(y) \\
        v_n(y)
    \end{pmatrix}
\end{equation*}
\begin{equation*}
    F_n(y) = \begin{pmatrix}
        \alpha_n(1 + \mu_0)(\chi_3(y) cos(\alpha_n a) - \chi_1(y)) \\
        (\frac{\alpha_2}{\beta_2}\chi_4(y) cos(\alpha_n a) - \frac{\alpha_1}{\beta_1}\chi_2(y)) - \mu_0 (\chi_3^{'}(y) cos(\alpha_n a) -\chi_1^{'}(y))
    \end{pmatrix}
\end{equation*}
Граничні умови \eqref{transf_bound_gen} запишемо у наступному вигляді:
\begin{align}\label{transf_bound_mat_gen}
    &U_i\left[ Z_n(y) \right] = E_i * Z_n^{'}(b_i) + F_i * Z_n(b_i) \nonumber \\
    &U_i\left[ Z_n(y) \right] = D_i
\end{align}
Де $i = \overline{0, 1}$, $b_0 = b$, $b_1 = 0$,
\begin{equation*}
    E_0 = \begin{pmatrix}
        1 & 0 \\
        0 & 2G + \lambda
    \end{pmatrix}, \quad
    F_0 = \begin{pmatrix}
        0 & -\alpha_n \\
        \alpha_n \lambda & 0
    \end{pmatrix}, \quad
\end{equation*}
\begin{equation*}
    E_1 = \begin{pmatrix}
        1 & 0 \\
        0 & 0
    \end{pmatrix}, \quad
    F_1 = \begin{pmatrix}
        0 & -\alpha_n \\
        0 & 1
    \end{pmatrix}, \quad
\end{equation*}
\begin{equation*}
    D_0 = \begin{pmatrix}
        0 \\
        -p_n
    \end{pmatrix}, \quad
    D_1 = \begin{pmatrix}
        0 \\
        0
    \end{pmatrix}, \quad
\end{equation*}

Для знаходження розв'язку задачі у просторі трансформант, знайдем фундаментальну матрицю рівняння \eqref{transf_mat_gen}.
Шукати її будем у наступному вигляді:
\begin{equation}
    Y(y) = \frac{1}{2\pi i} \oint_C e^{sy} M^{-1}(s)ds
\end{equation}
Де $M(s)$ - характерестична матриця рівняння \eqref{transf_mat_gen}, а $C$ - замкнений контур який містить усі особливі точки. Яку будемо шукати з наступної умовни
\begin{equation}
    L_2\left[ e^{sy}*I \right] = e^{sy} * M(s), \quad I = \begin{pmatrix} 1 & 0 \\ 0 & 1 \end{pmatrix}
\end{equation}
\begin{align*}
    &L_2\left[ e^{sy}*I \right] = e^{sy} \left( s^2A * I + s B*I + C*I \right) = \\
    &=e^{sy} \begin{pmatrix}
        s^2 - \alpha_n^2 - \alpha_n^2\mu_0 + \frac{\omega^2}{c_1^2} & -\alpha_n \mu_0 s \\
        \alpha_n \mu_0 s & s^2 (1 + \mu_0) -\alpha_n^2 + \frac{\omega^2}{c_1^2}
     \end{pmatrix} =>
\end{align*}

\begin{equation}
    M(s) = \begin{pmatrix}
        s^2 - \alpha_n^2 - \alpha_n^2\mu_0 + \frac{\omega^2}{c_1^2} & -\alpha_n \mu_0 s \\
        \alpha_n \mu_0 s & s^2 (1 + \mu_0) -\alpha_n^2 + \frac{\omega^2}{c_2^2}
     \end{pmatrix}
\end{equation}

Знайдемо тепер $M^{-1}(s) = \frac{\widetilde{M(s)}}{det[M(s)]}$.
\begin{equation}
    \widetilde{M(s)} = \begin{pmatrix}
        s^2 (1 + \mu_0) -\alpha_n^2 + \frac{\omega^2}{c_2^2} & \alpha_n \mu_0 s \\
        -\alpha_n \mu_0 s & s^2 - \alpha_n^2 - \alpha_n^2\mu_0 + \frac{\omega^2}{c_1^2}
     \end{pmatrix}
\end{equation}
\begin{align}
    &det[M(s)] = (s^2 (1 + \mu_0) -\alpha_n^2 + \frac{\omega^2}{c_2^2})(s^2 - \alpha_n^2 - \alpha_n^2\mu_0 + \frac{\omega^2}{c_1^2}) + (\alpha_n \mu_0 s)^2 = \nonumber \\
    &=(1+\mu_0)(s - a_1)(s + a_1)(s - a_2)(s + a_2)
\end{align}
Де $a_1$, $a_2$:
\begin{align*}
    a_1 = \sqrt{\frac{b_1}{b_3} - \omega \sqrt{\frac{b_2}{b_3}}}, \\
    a_2 = \sqrt{\frac{b_1}{b_3} + \omega \sqrt{\frac{b_2}{b_3}}} \\
\end{align*}
\begin{align*}
    &b_1 = 2 \alpha_n^2 c_1^2 c_2^2 \mu_0 + 2 \alpha_n^2 c_1^2 c_2^2 - c_1^2 \omega^2 - c_2^2 \mu_0 \omega^2 - c_2^2 \omega^2, \\
    &b_2 = 4\alpha_n^2 c_1^4 c_2^2 \mu_0^2 + 4 \alpha_n^2 c_1^4 c_2^2 \mu_0 - 4 \alpha_n^2 c_1^2 c_2^4 \mu_0^2  - \\
    &- 4 \alpha_n^2 c_1^2 c_2^4 \mu_0 + c_1^4 \omega^2 - 2 c_1^2 c_2^2 \mu_0 \omega^2 - 2 c_1^2 c_2^2 \omega^2 + \\ 
    &+ c_2^4 \mu_0^2 \omega^2 + c_2^4 \omega^2, \\
    &b_3 = 2 c_1^2 c_2^2 \mu_0 + 2 c_1^2 c_2^2
\end{align*}

Враховучи це, тепер знайдемо значення фундаментальної матрицю за допомогою теореми про лишки:
\begin{align*}
    &\frac{1}{2\pi i} \oint_C e^{sy} M^{-1}(s)ds = \frac{2 \pi i}{2 \pi i (1 + \mu_0)} \sum_{i=1}^{2} Res\left[ e^{sy} \frac{\widetilde{M(s)}}{det[M(s)]} \right] = \\
    & = \frac{1}{(1 + \mu_0)} \left(Y_0(y) + Y_1(y) + Y_2(y) + Y_3(y) \right)
\end{align*}
Знайдем $Y_0(y)$:
\begin{align}
    &Y_0(y) =  \left( \frac{e^{sy}}{(s+a_1)(s - a_2)(s + a_2)} \widetilde{M(s)} \right) \Big|_{s=a_1} = \nonumber \\
    &=\frac{e^{a_1 y}}{2a_1 (a_1^2 - a_2^2)} \begin{pmatrix}
        a_1^2 (1 + \mu_0) -\alpha_n^2 + \frac{\omega^2}{c_2^2} & \alpha_n \mu_0 a_1 \\
        -\alpha_n \mu_0 a_1 & a_1^2 - \alpha_n^2 - \alpha_n^2\mu_0 + \frac{\omega^2}{c_1^2}
    \end{pmatrix}
\end{align}
Знайдем $Y_1(y)$:
\begin{align}
    &Y_1(y) =  \left( \frac{e^{sy}}{(s-a_1)(s - a_2)(s + a_2)} \widetilde{M(s)} \right) \Big|_{s=-a_1} = \nonumber \\
    &=-\frac{e^{-a_1 y}}{2a_1 (a_1^2 - a_2^2)} \begin{pmatrix}
        a_1^2 (1 + \mu_0) -\alpha_n^2 + \frac{\omega^2}{c_2^2} & -\alpha_n \mu_0 a_1 \\
        \alpha_n \mu_0 a_1 & a_1^2 - \alpha_n^2 - \alpha_n^2\mu_0 + \frac{\omega^2}{c_1^2}
    \end{pmatrix}
\end{align}
Знайдем $Y_2(y)$:
\begin{align}
    &Y_2(y) =  \left( \frac{e^{sy}}{(s+a_2)(s - a_1)(s + a_1)} \widetilde{M(s)} \right) \Big|_{s=a_2} = \nonumber \\
    &=\frac{e^{a_2 y}}{2a_2 (a_2^2 - a_1^2)} \begin{pmatrix}
        a_2^2 (1 + \mu_0) -\alpha_n^2 + \frac{\omega^2}{c_2^2} & \alpha_n \mu_0 a_2 \\
        -\alpha_n \mu_0 a_2 & a_2^2 - \alpha_n^2 - \alpha_n^2\mu_0 + \frac{\omega^2}{c_1^2}
    \end{pmatrix}
\end{align}
Знайдем $Y_3(y)$:
\begin{align}
    &Y_3(y) =  \left( \frac{e^{sy}}{(s-a_2)(s - a_1)(s + a_1)} \widetilde{M(s)} \right) \Big|_{s=-a_2} = \nonumber \\
    &=-\frac{e^{-a_2 y}}{2a_2 (a_2^2 - a_1^2)} \begin{pmatrix}
        a_2^2 (1 + \mu_0) -\alpha_n^2 + \frac{\omega^2}{c_2^2} & -\alpha_n \mu_0 a_2 \\
        \alpha_n \mu_0 a_2 & a_2^2 - \alpha_n^2 - \alpha_n^2\mu_0 + \frac{\omega^2}{c_1^2}
    \end{pmatrix}
\end{align}

Знайдем тепер фундамельні бизисні матриці $\Psi_0(y)$, $\Psi_1(y)$, шукать їх будем у наступному вигляді:
\begin{equation}
    \Psi_i(y) = \frac{1}{1 + \mu_0} \left( Y_0(y) + Y_1(y) \right) * C_1^i + \frac{1}{1 + \mu_0} \left( Y_2(y) + Y_3(y) \right) * C_2^i
\end{equation}

Залишилось знайти невідомі матриці коєфіцієнтів $C_1^0$, $C_2^0$, $C_1^1$, $C_2^1$ використовуючи граничні умови \eqref{transf_bound_mat_gen}.
Покрокове знаходження яких наведено у (\nameref{ap_B}).

Таким чином матрицю Гріна можемо записати у вигляді:
\begin{equation}
    G(y,\xi) = 
    \begin{cases}
        \Psi_0(y) * \Psi_1(\xi), \quad 0 \le y < \xi \\
        \Psi_1(y) * \Psi_0(\xi), \quad \xi < y \le b
    \end{cases}
\end{equation}

Для данної матриці Гріна виконано усі властивості, зокрема виконані однорідні граничні умови \eqref{transf_bound_mat_gen}
та однорідні рівняння рівноваги у просторі трансформант \eqref{transf_mat_gen}:
\begin{equation*}
    L_2\left[  G(y, \xi) \right] = 0
\end{equation*}
\begin{equation*}
    U_0\left[ G(y, \xi) \right] = 0, \quad  U_1\left[ G(y, \xi) \right],
\end{equation*}

Таким чином ми можемо записати розв'язок крайової задачі у просторі трансформант:
\begin{equation}
    Z_n(y) = \int_0^b G(y,\xi) F_n(\xi) d\xi + \Psi_0(y) * D_0 + \Psi_1(y) * D_1
\end{equation}

Введемо наступні позначення $G(y, \xi) = \begin{pmatrix}
    g_1(y,\xi) & g_2(y,\xi) \\
    g_3(y,\xi) & g_4(y,\xi)
\end{pmatrix}$, $F_n(y) = \begin{pmatrix}
    f_n^1(y) \\
    f_n^2(y)
\end{pmatrix}$, $\Psi_i(y) = \begin{pmatrix}
    \psi_i^1(y) & \psi_i^2(y) \\
    \psi_i^3(y) & \psi_i^4(y)
\end{pmatrix}$, $i=0,1$. Враховуючи це, шукані функціі перемішень у просторі трансформант можна записати у наступному вигляді
\begin{align}\label{transf_sol_u_gen}
    &u_n(y) = \int_0^b \left[g_1(y, \xi)f_n^1(\xi) + g_2(y, \xi)f_n^2(\xi) \right]d\xi - \psi_0^2(y) p_n
\end{align}
\begin{align}\label{transf_sol_v_gen}
    &v_n(y) = \int_0^b \left[g_3(y, \xi)f_n^1(\xi) + g_4(y, \xi)f_n^2(\xi) \right]d\xi - \psi_0^4(y) p_n
\end{align}

\subsection{Фінальний розв'язок задачі}
Викорустовуючи обернене інтегральне перетворення Фур'є до розв'язку задачі у просторі трансформант
(\ref{transf_sol_u_gen}), (\ref{transf_sol_v_gen}), отримаємо фінальний розв'язок задачі
\begin{equation}
    u(x,y) = \frac{2}{a} \sum_{n=1}^{\infty} u_n(y) sin(\alpha_n x), \quad \alpha_n = \frac{\pi n}{a}
\end{equation}
\begin{equation}
    v(x,y) = \frac{v_0(y)}{a} + \frac{2}{a} \sum_{n=1}^{\infty} v_n(y) cos(\alpha_n x), \quad \alpha_n = \frac{\pi n}{a}
\end{equation}

Залишилось знайти невідомі функції $\chi_1(y)$, $\chi_2(y)$, $\chi_3(y)$, $\chi_4(y)$, для цього треба побудувати систему інтегральних рівнянь,
використовуючи граничну умову $\sigma_y(x, y) |_{y=b} = -p(x)$. В залежності від умов бічних гранях розв’язання задачі зводиться
