У даному розділі наведено опис аналітичного апарату, який використовується для розв'язання мішаних задач теорії пружності для прямокутної області.
Цей підхід базується на результати раніше проведених досліджень, зокрема робіт \cite{popov_1} і \cite{popov_2}.
Розглянута методика розв'язання мішаних плоских задач ґрунтується на застосуванні інтегральних перетворень безпосередньо до системи рівнянь рівноваги Ламе та крайових умов.
Це дозволяє зводити вихідну задачу до векторної одновимірної крайової задачі.
Векторна одновимірна крайова задача точно розв'язується за допомогою матричного диференційного числення та матричної функції Гріна.

\subsection{Зведення вихідної крайової задачі до одновимірної векторної крайової задачі}
Розглядається пружна прямокутна область, яка займає область, що у декартовій системі координат описується співвідношенням $0 \le x \le a$, $0 \le y \le b$.

До прямокутної області на грані $y=b$ додане нормальне навантаження
\begin{equation}
    \sigma_y(x, y, t) |_{y=b} = -p(x, t), \quad  \tau_{xy}(x,y,t) |_{y=b} =0, \quad 0 \le x \le a
\end{equation}
де $p(x, t)$ відома функція.
На бічних гранях $x=0$ та $x=a$ граничні умови запишемо у формі
\begin{equation}
    U_0[f(x,y,t)]=0, \quad U_1[f(x,y,t)]=0 , \quad 0 \le y \le b
\end{equation}
Де 
\begin{align*}
    &U_0[f(x,y,t)]=\left[\alpha_0f(x,y,t) + \beta_0 \frac{\partial f(x,y,t)}{\partial x} \right]|_{x=0} \\
    &U_1[f(x,y,t)]=\left[\alpha_1f(x,y,t) + \beta_1 \frac{\partial f(x,y,t)}{\partial x} \right]|_{x=a} \\
\end{align*}
граничні функціонали у загальному виді (для кожної конкретної задачі вони будуть деталізовані), $f(x,y,t)=(u(x,y,t), v(x,y,t))^T$ - вектор переміщеннь