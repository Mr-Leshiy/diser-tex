
Дослідження напружено-деформованого стану пружних тіл почало активно розвиватися у XIX столітті і залишається актуальним до нашого часу.
Це пояснюється широким спектром застосування в різноманітних інженерних галузях.
Класична лінійна теорія пружності є основою для більшості міцностних розрахунків в техніці.
Під час експлуатації будівель та інших конструкцій вони піддаються механічним, температурним та іншим впливам.
Тому при проектуванні необхідно враховувати міцність таких конструкцій. Характеристики міцності виробів можна отримати шляхом аналізу напружено-деформованого стану їх пружних моделей.
Одним з таких моделей є скінченна прямокутна область.
Тому актуальною проблемою є розробка аналітично-числових методів для дослідження її напружено-деформованого стану.