
Дослідження властивостей різних прирьодніх матеріалів цікавило людство, ще з далеких часів,
в основному для вирішення проблем будівництва.
У XIX столітті почало активно розвиватись дослідження напружено-деформованого стану пружних тіл та залишається актуальним в наші часи.
Це обгрунтовується великим попитом з боку інженерних галузь.
Під час експлуатації різноманітних виробів, від автомобілів до літаків та різних будівельних конструкцій, вони піддаються різним механічним впливам.
Тому при використанні різних матеріалів важливо враховувати їх міцність та пружні властивості.
Які можливо отримати шляхом аналізу напружено-деформованого стану математичних моделей.
Одним з таких моделей є пружна прямокутна область.
Тому розробка математичних методів для дослідження напружено-деформованого стану лишається актуальною проблемою.

Класичні загальні підходи до знаходження аналітичного розв'язку задач теорії пружності розглянуто у монографіях 