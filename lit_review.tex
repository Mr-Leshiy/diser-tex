
Дослідження властивостей різних прирьодніх матеріалів цікавило людство, ще з далеких часів,
в основному для вирішення проблем будівництва.
У XIX столітті почало активно розвиватись дослідження напружено-деформованого стану пружних тіл та залишається актуальним в наші часи.
Це обгрунтовується великим попитом з боку інженерних галузь.
Під час експлуатації різноманітних виробів, від автомобілів до літаків та різних будівельних конструкцій, вони піддаються різним механічним впливам.
Тому при використанні різних матеріалів важливо враховувати їх міцність та пружні властивості.
Які можливо отримати шляхом аналізу напружено-деформованого стану математичних моделей.
Одним з таких моделей є пружна прямокутна область.
Тому розробка математичних методів для дослідження напружено-деформованого стану лишається актуальною проблемою.

Класичні загальні підходи до знаходження аналітичного розв'язку задач теорії пружності наведено у монографіях
В. Новацького, В. Т. Грінченка, В. В. Мелешка, В. М. Вігака, М. Й. Юзвяка, В. А. Ясинського, Ю. В. Токового, Г. Я. Попова \cite{novacki_1, meleshko_1, vihak_1, vihak_2, popov_1}.
Розвинення та вдосконаленя цих підходів досліджено у роботах Я. О. Жука, О. К. Остоса, Р. М. Кушніра, Л. А. Фільштінського, Ю. В. Шрамко \cite{zhuk_1,kushnir_1, filshtin_1}
Дослідженню та розв'язанню сингулярних інтегральних рівнянь розглянуто у працях М. Г. Крейна, З. Ю. Журавльової, В. Г. Попова \cite{kreyn_1, zhuravleva_1,popov_v_1}.

\subsection{Чисельні методи розв'язання задач теорії пружності}
Спочатку розглянемо праці присвячені чисельним методам розв'язання задач теорії пружності
для того, щоби порівняти їх ефективність з аналітичним підходом побудови розв'язку.

Однією за класичних робот у цій сфері є \cite{oden_1}, що є розвиненням ідеї запропонованної у \cite{babushka_1}.
Розглянуто методи скінченних елементів, що застосовані до задач теорії пружності з деякими обмеженнями на рух.
Особлива увага приділяється методам штрафів.
Наведено обговорення умов, необхідних для того, щоби методи штрафів забезпечували стійкість та збіжність методів скінченних елементів.
Зокрема, розглядається використання методів зі зменшеним інтегруванням і описані критерії для порядку правил інтегруванням, достатніх для отримання стійких та збіжних схем.
Наведено приклади застосування методів зі зменшеним інтегруванням та штрафів до задач інкомпресібельної еластичності та задач контакту.

Maurizio A. \cite{maurizio_1} представлено метод скінченних елементів для аналізу двовимірних задач теорії пружності.
Запропонована дискретизація базується на біквадратичній інтерполяції компонентів переміщення та
використовує переваги забезпечення безперервності між елементами для отримання сприятливого зменшення загальної кількості ступенів свободи.

У праці \cite{liew_1} пропонується вдосконалена схема методу найменших квадратів (IMLS).
Обговорено метод найменших квадратів (MLS), який може призводити до погано обумовленої системи рівнянь,
що призводить до неправильного отримання розв'язку.
В IMLS використовується ортогональна система функцій з ваговою функцією як базисна функція.
IMLS має вищу обчислювальну ефективність і точність, ніж класичий метод найменьших квадратів (MLS),
і не призводить до погано обумовлених систем рівнянь.
Шляхом поєднання методу граничних елементів і методу IMLS, був представлений безсітковий метод граничних елементів (BEFM).

% In this paper, we consider the linear elasticity problem based on the Hellinger-Reissner variational principle. An O(h2) order superclose property for the stress and displacement and a global superconvergence result of the displacement are established by employing a Cl ́ement interpolation, an integral identity and appropriate postprocessing techniques.
Dongyang Shi, Minghao Li \cite{dong_1} розглянуто лінійна задача пружності на основі варіаційного принципу Геллінгера-Рейснера.
\textcolor{orange}{За допомогою інтерполяції Клемана, інтегральної тотожності та відповідних технік післяобробки,
встановлено властивості суперзбіжності $O(h^2)$ напружень та переміщень.}

Головчан В. Т. \cite{golovchan_1} побудував алгоритм вирішення плоских задач теорії пружності для прямокутної області.
Алгоритм ґрунтується на комплекснозначному представленні загального розв'язку диференціальних рівнянь
та використанні поліномів Лагранжа для задоволення крайових умов.
Стверджується, що це, можливо, найпростіший спосіб розв'язання крайових задач цього класу.

Як бачимо чисельні методи мають велику популярність у використанні та реалізації у багатьох чисельних програмних реалізаціях,
котрі доволі зручні у використанні в інженерній сфері.
Але якщо потрібно здійснити розрахунок напружень прямокутної області поблизу кутових точок,
то, як відомо, чисельні методи втрачають свою ефективність.

\subsection{Аналітичні методи розв'язання задач теорії пружності}

Проаналізуємо тепер аналітичні підходи для розв'язання мішаних задач для прямокутної області.

У праці \cite{shyam_1} розглянуто загальний випадок задачі прямокутного еластичного тіла у випадку плоскої деформації,
коли дві паралельні грані несуть нульове навантаження, а змішані умови задані на решту сторін.
Задача сформульована у вигляді системи двох інтегральних рівнянь Фредгольма другого роду.
Тип змішаних крайових умов може виникнути через стиснення гладкими штампами або через деякий періодичний розподіл системи тріщин у нескінченно довгій смузі.

М. Д. Коваленко \cite{kovalenko_1} отримано формули, що описують точний розв'язок крайової задачі теорії пружності для прямокутної області,
в якій дві протилежні (горизонтальні) грані є вільними, а напруження задані на інших двох сторонах (кінцях прямокутника).
Також наведені формули для напівсмуги. Розв'язки представлені у вигляді рядів за власними функціями Папковича-Фадле, коефіцієнти яких визначаються з простих формул.
Отримані формули залишаються такими ж для інших типів однорідних крайових умов, наприклад,
коли горизонтальні сторони прямокутника міцно закріплені, мають підкріплювальні ребра, які працюють на розтяг чи згинання тощо.

Bantsuri R. \cite{bantsuri_1} досліджено мішану задачу теорії пружності для прямокутника, який послаблено отворами,
коли дотичні напруження на зовнішній границі дорівнюють нулю, а нормальні зсуви постійні,
тоді як дотичні напруження на отворах дорівнюють нулю, а нормальні напруження постійні.
Знайдено границі отворів коли умови дотичного нормального напруження дорівнюють константам.
Використовуючи методи теорії аналітичних функцій, розв'язок знаходиться у квадратурах.

У дослідженні \cite{kashtalyan_1} розглянуто тривимірна задача пружності для прямокутної прастини,
яка піддається поперечному навантаженню.
Модуль Юнга пластини вважається змінним експоненційно по товщині, а коефіцієнт Пуассона припускається сталим.
Використано загальний розв'язок Плевако рівнянь рівноваги для неоднорідних ізотропних середовищ.

Роботу \cite{vihak_2} присвячено встановленю необхідних умов існування розв'язку для плоскої задачі теорії пружності для прямокутної області
з заданими зовнішніми крайовими та об'ємними силами.
Побудовано елементарні розв'язки, що задовольняють принципам Сен-Венана та принципу суперпозиції.

\subsection{Висновки до другого розділу}

Як проаналізовано в цьому розділі, два основних підходи використовуються для задач теорії пружності: чисельний та аналітичний.
Вибір між ними обумовлюється поданням граничних умов.

У данній дисертаційній роботі наведена методика, запропонована Г. Я. Поповим \cite{popov_4},
в якій застосування інтегральних перетворень безпосередньо до рівнянь рівноваги та крайових умов
дозволило звести вихідну задачу до одновимірної задачі у просторі трансформант.
Яку в подальшому зведено та розв'язано у векторній формі \cite{popov_5}.
Розв'язок отримного сінгулярного інтегрального рівняння побудовано за допомогою методу ортогональних поліномів \cite{popov_3}.

Згідно з аналізом наукової літератури,
не дивлячись на значну кількість проведених досліджень у напрямку мішаних задач теорії пружності для прямокую області,
залишається багато не вирішених питань, зокрема знаходження аналітичного розв'язку динамічних задач теорії пружності для прямокутника.
Цим і доводиться актуальність данної теми дослідження.