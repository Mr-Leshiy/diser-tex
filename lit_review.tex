
Дослідження властивостей різних прирьодніх матеріалів цікавило людство, ще з далеких часів,
в основному для вирішення проблем будівництва.
У XIX столітті почало активно розвиватись дослідження напружено-деформованого стану пружних тіл та залишається актуальним в наші часи.
Це обгрунтовується великим попитом з боку інженерних галузь.
Під час експлуатації різноманітних виробів, від автомобілів до літаків та різних будівельних конструкцій, вони піддаються різним механічним впливам.
Тому при використанні різних матеріалів важливо враховувати їх міцність та пружні властивості.
Які можливо отримати шляхом аналізу напружено-деформованого стану математичних моделей.
Одним з таких моделей є пружна прямокутна область.
Тому розробка математичних методів для дослідження напружено-деформованого стану лишається актуальною проблемою.

Класичні загальні підходи до знаходження аналітичного розв'язку задач теорії пружності наведено у монографіях
В. Новацького, В. Т. Грінченка, В. В. Мелешка, В. М. Вігака, М. Й. Юзвяка, В. А. Ясинського, Ю. В. Токового, Г. Я. Попова \cite{novacki_1, meleshko_1, vihak_1, vihak_2, popov_1}.
Розвинення та вдосконаленя цих підходів досліджено у роботах Я. О. Жука, О. К. Остоса, Р. М. Кушніра, Л. А. Фільштінського, Ю. В. Шрамко \cite{zhuk_1,kushnir_1, filshtin_1}
Дослідженню та розв'язанню сингулярних інтегральних рівнянь розглянуто у працях М. Г. Крейна, З. Ю. Журавльової, В. Г. Попова \cite{kreyn_1, zhuravleva_1,popov_v_1}.

\subsection{Чисельні методи розв'язання задач теорії пружності}
Спочатку розглянемо праці присвячені чисельним методам розв'язання задач теорії пружності
для того, щоби порівняти їх ефективність з аналітичним підходом побудови розв'язку.

Однією за класичних робот у цій сфері є \cite{oden_1}, що є розвиненням ідеї запропонованної у \cite{babushka_1}.
Розглянуто методи скінченних елементів, що застосовані до задач теорії пружності з деякими обмеженнями на рух.
Особлива увага приділяється методам штрафів.
Наведено обговорення умов, необхідних для того, щоби методи штрафів забезпечували стійкість та збіжність методів скінченних елементів.
Зокрема, розглядається використання методів зі зменшеним інтегруванням і описані критерії для порядку правил інтегруванням, достатніх для отримання стійких та збіжних схем.
Наведено приклади застосування методів зі зменшеним інтегруванням та штрафів до задач інкомпресібельної еластичності та задач контакту.

Maurizio A. \cite{maurizio_1} представлено метод скінченних елементів для аналізу двовимірних задач теорії пружності.
Запропонована дискретизація базується на біквадратичній інтерполяції компонентів переміщення та
використовує переваги забезпечення безперервності між елементами для отримання сприятливого зменшення загальної кількості ступенів свободи.

У праці \cite{liew_1} пропонується вдосконалена схема методу найменших квадратів (IMLS).
Обговорено метод найменших квадратів (MLS), який може призводити до погано обумовленої системи рівнянь,
що призводить до неправильного отримання розв'язку.
В IMLS використовується ортогональна система функцій з ваговою функцією як базисна функція.
IMLS має вищу обчислювальну ефективність і точність, ніж класичий метод найменьших квадратів (MLS),
і не призводить до погано обумовлених систем рівнянь.
Шляхом поєднання методу граничних елементів і методу IMLS, був представлений безсітковий метод граничних елементів (BEFM).

% In this paper, we consider the linear elasticity problem based on the Hellinger-Reissner variational principle. An O(h2) order superclose property for the stress and displacement and a global superconvergence result of the displacement are established by employing a Cl ́ement interpolation, an integral identity and appropriate postprocessing techniques.
Dongyang Shi, Minghao Li \cite{dong_1} розглянуто лінійна задача пружності на основі варіаційного принципу Геллінгера-Рейснера.
\textcolor{orange}{За допомогою інтерполяції Клемана, інтегральної тотожності та відповідних технік післяобробки,
встановлено властивості суперзбіжності $O(h^2)$ напружень та переміщень.}

Головчан В. Т. \cite{golovchan_1} побудував алгоритм вирішення плоских задач теорії пружності для прямокутної області.
Алгоритм ґрунтується на комплекснозначному представленні загального розв'язку диференціальних рівнянь
та використанні поліномів Лагранжа для задоволення крайових умов.
Стверджується, що це, можливо, найпростіший спосіб розв'язання крайових задач цього класу.

\subsection{Аналітичні методи розв'язання задач теорії пружності}


У праці \cite{shyam_1} розглянуто загальний випадок задачі прямокутного еластичного тіла у випадку плоскій деформації,
коли дві паралельні грані несуть нульове навантаження, а змішані умови задані на решту сторін.
Задача сформульована у вигляді системи двох інтегральних рівнянь Фредгольма другого роду.
Тип змішаних крайових умов може виникнути через стиснення гладкими штампами або через деякий періодичний розподіл системи тріщин у нескінченно довгій смузі.