
Дослідження властивостей різних прирьодніх матеріалів цікавило людство, ще з далеких часів,
в основному для вирішення проблем будівництва.
У XIX столітті почало активно розвиватись дослідження напружено-деформованого стану пружних тіл та залишається актуальним в наші часи.
Це обгрунтовується великим попитом з боку інженерних галузь.
Під час експлуатації різноманітних виробів, від автомобілів до літаків та різних будівельних конструкцій, вони піддаються різним механічним впливам.
Тому при використанні різних матеріалів важливо враховувати їх міцність та пружні властивості.
Які можливо отримати шляхом аналізу напружено-деформованого стану математичних моделей.
Одним з таких моделей є пружна прямокутна область.
Тому розробка математичних методів для дослідження напружено-деформованого стану лишається актуальною проблемою.

Класичні загальні підходи до знаходження аналітичного розв'язку задач теорії пружності наведено у монографіях
В. Новацького, В. Т. Грінченка, В. В. Мелешка, В. М. Вігака, М. Й. Юзвяка, В. А. Ясинського, Ю. В. Токового, Г. Я. Попова \cite{novacki_1, meleshko_1, vihak_1, vihak_2, popov_1}.
Розвинення та вдосконаленя цих підходів досліджено у роботах Я. О. Жука, О. К. Остоса, Р. М. Кушніра, Л. А. Фільштінського, Ю. В. Шрамко \cite{zhuk_1,kushnir_1, filshtin_1}
Дослідженню та розв'язанню сингулярних інтегральних рівнянь розглянуто у працях М. Г. Крейна, З. Ю. Журавльової, В. Г. Попова \cite{kreyn_1, zhuravleva_1,popov_v_1}.

\subsection{Чисельні методи розв'язання задач теорії пружності}
Спочатку розглянемо праці присвячені чисельним методам розв'язання задач теорії пружності
для того, щоби порівняти їх ефективність з аналітичним підходом побудови розв'язку.



\subsection{Аналітичні методи розв'язання задач теорії пружності}