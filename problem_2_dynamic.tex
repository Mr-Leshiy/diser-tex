

\subsection{Постановка задачі}
Розглядається пружна така сама прямокутна область як і в попередній задачі
$0 \le x \le a$, $0 \le y \le b$.

До прямокутної області на грані $y=b$ додане нормальне навантаження
\begin{equation}
    \sigma_y(x, y, t) |_{y=b} = -p(x, t), \quad  \tau_{xy}(x,y,t) |_{y=b} =0
\end{equation}
де $p(x,t)$ відома функція.
На бічних гранях виконується умова ідеального контакту
\begin{equation}
    u(x,y,t) |_{x=0}, \quad \tau_{xy}(x,y,t) |_{x=0} =0
\end{equation}
\begin{equation}
    u(x,y,t) |_{x=a}, \quad \tau_{xy}(x,y,t) |_{x=a} =0
\end{equation}
На нижній грані виконуються наступні умови
\begin{equation}
    v(x,y,t) |_{y=0}, \quad \tau_{xy}(x,y,t) |_{y=0} =0
\end{equation}
Розглядаються наступні рівняння рівноваги Ламе:
\begin{equation}
    \begin{cases}
        \frac{\partial^2 u(x,y,t)}{\partial x^2} + \frac{\partial^2 u(x,y,t)}{\partial y^2} + \mu_0 (\frac{\partial^2 u(x,y,t)}{\partial x^2} + \frac{\partial^2 v(x,y,t)}{\partial x\partial y}) = \frac{1}{c_1^2} \frac{\partial^2 u(x,y,t)}{\partial t^2} \\
        \frac{\partial^2 v(x,y,t)}{\partial x^2} + \frac{\partial^2 v(x,y,t)}{\partial y^2} + \mu_0 (\frac{\partial^2 u(x,y,t)}{\partial x \partial y} + \frac{\partial^2 v(x,y,t)}{\partial y^2}) = \frac{1}{c_2^2} \frac{\partial^2 v(x,y,t)}{\partial t^2} \\
    \end{cases}
\end{equation}

Будемо розглядати випадок гармонічних коливань, тому можемо предствавити функції у наступному вигляді:
\begin{equation}
    u(x,y,t) = u(x,y) e^{i \omega t}, \quad v(x,y,t) = v(x,y) e^{i \omega t}, \quad p(x,t) = p(x) e^{i \omega t}
\end{equation}
Таким чином отримаємо наступні рівняння рівноваги:
\begin{equation}\label{lame_2}
    \begin{cases}
        \frac{\partial^2 u(x,y)}{\partial x^2} + \frac{\partial^2 u(x,y)}{\partial y^2} + \mu_0 (\frac{\partial^2 u(x,y)}{\partial x^2} + \frac{\partial^2 v(x,y)}{\partial x\partial y}) = -\frac{\omega^2}{c_1^2}  u(x,y) \\
        \frac{\partial^2 v(x,y)}{\partial x^2} + \frac{\partial^2 v(x,y)}{\partial y^2} + \mu_0 (\frac{\partial^2 u(x,y)}{\partial x \partial y} + \frac{\partial^2 v(x,y)}{\partial y^2}) = -\frac{\omega^2}{c_2^2} v(x,y) \\
    \end{cases}
\end{equation}
Та граничні умови:
\begin{equation}\label{bound_2}
    \begin{cases}
        \sigma_y(x, y) |_{y=b} = -p(x, t), \quad  \tau_{xy}(x,y) |_{y=b} =0 \\
        u(x,y) |_{x=0}, \quad \tau_{xy}(x,y) |_{x=0} =0 \\
        u(x,y) |_{x=a}, \quad \tau_{xy}(x,y) |_{x=a} =0 \\
        v(x,y) |_{y=0}, \quad \tau_{xy}(x,y) |_{y=0} =0
    \end{cases}
\end{equation}

\subsection{Зведеня задачі до одновимірної у просторі трансформант}
Для того, щоб звести задачу до одновимірної задачі, використаєм інтегральне перетворення Фур'є по змінній $x$ у до рівнянь (\ref{lame_1}) наступному вигляді:
\begin{equation}
    \begin{pmatrix}
        u_n(y) \\
        v_n(y)
    \end{pmatrix} = \int_{0}^{a} 
    \begin{pmatrix}
        u(x,y) sin(\alpha_n x) \\
        v(x,y) cos(\alpha_n x)
    \end{pmatrix} dx, \quad \alpha_n = \frac{\pi n}{a}, n=\overline{1, \infty}
\end{equation}

Для цього помножим перше та друге рівняння (\ref{lame_2}) на $sin(\alpha_n x)$ та $cos(\alpha_n x)$ відповідно та проінтегруєм по змінній $x$ на інтервалі $0 \le x \le a$.
Покрокове інтегрування рівняння (\ref{lame_2}) наведено у (\nameref{ap_A_1}).
Отримана система рівнянь задачі у просторі трансформант:
\begin{equation}\label{transf_2}
    \begin{cases}
        u_n^{''}(y) - \alpha_n \mu_0 v_n^{'}(y) + (-\alpha_n^2 - -\alpha_n^2 \mu_0 + \frac{\omega^2}{c_1^2}) u_n(y) = 0 \\
        (1 + \mu_0) v_n^{''}(y) + \alpha_n \mu_0 u_n^{'}(y) + (- \alpha_n^2 + \frac{\omega^2}{c_2^2}) v_n(y) = 0 \\
    \end{cases}
\end{equation}

Застосовуючи інтегральне перетворення до граничних умов,
отримаємо наступні умови задачі у просторі трансформант
\begin{equation}\label{transf_bound_2}
    \begin{cases}
        \left( (2G + \lambda)v_n^{'}(y) + \alpha_n \lambda u_n(y) \right)|_{y=b} = -p_n \\
        \left(u_n^{'}(y) - \alpha_n v_n(y)  \right)|_{y=b} = 0 \\
        v_n(y)|_{y=0} = 0 \\
        \left(u_n^{'}(y) - \alpha_n v_n(y)  \right)|_{y=0} = 0
    \end{cases}
\end{equation}
Де $p_n = \int_{0}^{a} p(x) cos(\alpha_n x) dx$

\subsection{Зведення задачі у просторі трансформант до матрично-векторної форми}
Для того щоб розв'язати задачу у простосторі трансформант, перепишмо її у матрично-векторній формі.
Рівняння рівноваги (\ref{transf_2}) запишемо у наступному вигляді:
\begin{align}\label{transf_mat_2}
    &L_2\left[ Z_n(y) \right] = A * Z_n^{''}(y) + B * Z_n^{'}(y) + C * Z_n(y) \nonumber \\
    & L_2\left[ Z_n(y) \right] = 0
\end{align}
Де
\begin{equation*}
    A = \begin{pmatrix}
        1 & 0 \\
        0 & 1 + \mu_0
    \end{pmatrix}, \quad
    B = \begin{pmatrix}
        0 & -\alpha_n \mu_0 \\
        \alpha_n \mu_0 & 0
    \end{pmatrix}
\end{equation*}
\begin{equation*}
    C = \begin{pmatrix}
        -\alpha_n^2 -\alpha_n^2 \mu_0 + \frac{\omega^2}{c_1^2} & 0 \\
        0 & -\alpha_n^2 + \frac{\omega^2}{c_2^2}
    \end{pmatrix}, \quad
    Z_n(y) = \begin{pmatrix}
        u_n(y) \\
        v_n(y)
    \end{pmatrix}
\end{equation*}
Граничні умови (\ref{transf_bound_2}) запишемо у наступному вигляді:
\begin{align}\label{transf_bound_mat_2}
    &U_i\left[ Z_n(y) \right] = E_i * Z_n^{'}(b_i) + F_i * Z_n(b_i) \nonumber \\
    & U_i\left[ Z_n(y) \right] = D_i
\end{align}
Де $i = \overline{0, 1}$, $b_0 = b$, $b_1 = 0$,
\begin{equation*}
    E_0 = \begin{pmatrix}
        1 & 0 \\
        0 & 2G + \lambda
    \end{pmatrix}, \quad
    F_0 = \begin{pmatrix}
        0 & -\alpha_n \\
        \alpha_n \lambda & 0
    \end{pmatrix}, \quad
\end{equation*}
\begin{equation*}
    E_1 = \begin{pmatrix}
        1 & 0 \\
        0 & 0
    \end{pmatrix}, \quad
    F_1 = \begin{pmatrix}
        0 & -\alpha_n \\
        0 & 1
    \end{pmatrix}, \quad
\end{equation*}
\begin{equation*}
    D_0 = \begin{pmatrix}
        0 \\
        -p_n
    \end{pmatrix}, \quad
    D_1 = \begin{pmatrix}
        0 \\
        0
    \end{pmatrix}, \quad
\end{equation*}

Для знаходження розв'язку задачі у просторі трансформант, знайдем фундаментальну матрицю рівняння (\ref{transf_mat_2}).
Шукати її будем у наступному вигляді:
\begin{equation}
    Y(y) = \frac{1}{2\pi i} \oint_C e^{sy} M^{-1}(s)ds
\end{equation}
Де $M(s)$ - характерестична матриця рівняння (\ref{transf_mat_2}), а $C$ - замкнений контур який містить усі особливі точки. Яку будемо шукати з наступної умовни
\begin{equation}
    L_2\left[ e^{sy}*I \right] = e^{sy} * M(s), \quad I = \begin{pmatrix} 1 & 0 \\ 0 & 1 \end{pmatrix}
\end{equation}
\begin{align*}
    &L_2\left[ e^{sy}*I \right] = e^{sy} \left( s^2A * I + s B*I + C*I \right) = \\
    &=e^{sy} \begin{pmatrix}
        s^2 - \alpha_n^2 - \alpha_n^2\mu_0 + \frac{\omega^2}{c_1^2} & -\alpha_n \mu_0 s \\
        \alpha_n \mu_0 s & s^2 (1 + \mu_0) -\alpha_n^2 + \frac{\omega^2}{c_1^2}
     \end{pmatrix} =>
\end{align*}

\begin{equation}
    M(s) = \begin{pmatrix}
        s^2 - \alpha_n^2 - \alpha_n^2\mu_0 + \frac{\omega^2}{c_1^2} & -\alpha_n \mu_0 s \\
        \alpha_n \mu_0 s & s^2 (1 + \mu_0) -\alpha_n^2 + \frac{\omega^2}{c_2^2}
     \end{pmatrix}
\end{equation}

Знайдемо тепер $M^{-1}(s) = \frac{\widetilde{M(s)}}{det[M(s)]}$.
\begin{equation}
    \widetilde{M(s)} = \begin{pmatrix}
        s^2 (1 + \mu_0) -\alpha_n^2 + \frac{\omega^2}{c_2^2} & \alpha_n \mu_0 s \\
        -\alpha_n \mu_0 s & s^2 - \alpha_n^2 - \alpha_n^2\mu_0 + \frac{\omega^2}{c_1^2}
     \end{pmatrix}
\end{equation}
\begin{align}
    &det[M(s)] = (s^2 (1 + \mu_0) -\alpha_n^2 + \frac{\omega^2}{c_2^2})(s^2 - \alpha_n^2 - \alpha_n^2\mu_0 + \frac{\omega^2}{c_1^2}) + (\alpha_n \mu_0 s)^2 = \nonumber \\
    &=(1+\mu_0)(s - a_1)(s + a_1)(s - a_2)(s + a_2)
\end{align}
Де $a_1$, $a_2$:
\begin{align*}
    a_1 = \sqrt{\frac{b_1}{b_3} - \omega \sqrt{\frac{b_2}{b_3}}}, \\
    a_2 = \sqrt{\frac{b_1}{b_3} + \omega \sqrt{\frac{b_2}{b_3}}} \\
\end{align*}
\begin{align*}
    &b_1 = 2 \alpha_n^2 c_1^2 c_2^2 \mu_0 + 2 \alpha_n^2 c_1^2 c_2^2 - c_1^2 \omega^2 - c_2^2 \mu_0 \omega^2 - c_2^2 \omega^2, \\
    &b_2 = 4\alpha_n^2 c_1^4 c_2^2 \mu_0^2 + 4 \alpha_n^2 c_1^4 c_2^2 \mu_0 - 4 \alpha_n^2 c_1^2 c_2^4 \mu_0^2  - \\
    &- 4 \alpha_n^2 c_1^2 c_2^4 \mu_0 + c_1^4 \omega^2 - 2 c_1^2 c_2^2 \mu_0 \omega^2 - 2 c_1^2 c_2^2 \omega^2 + \\ 
    &+ c_2^4 \mu_0^2 \omega^2 + c_2^4 \omega^2, \\
    &b_3 = 2 c_1^2 c_2^2 \mu_0 + 2 c_1^2 c_2^2
\end{align*}

Враховучи це, тепер знайдемо значення фундаментальної матрицю за допомогою теореми про лишки:
\begin{align*}
    &\frac{1}{2\pi i} \oint_C e^{sy} M^{-1}(s)ds = \frac{2 \pi i}{2 \pi i (1 + \mu_0)} \sum_{i=1}^{2} Res\left[ e^{sy} \frac{\widetilde{M(s)}}{det[M(s)]} \right] = \\
    & = \frac{1}{(1 + \mu_0)} \left(Y_0(y) + Y_1(y) + Y_2(y) + Y_3(y) \right)
\end{align*}
Знайдем $Y_0(y)$:
\begin{align}
    &Y_0(y) =  \left( \frac{e^{sy}}{(s+a_1)(s - a_2)(s + a_2)} \widetilde{M(s)} \right) \Big|_{s=a_1} = \nonumber \\
    &=\frac{e^{a_1 y}}{2a_1 (a_1^2 - a_2^2)} \begin{pmatrix}
        a_1^2 (1 + \mu_0) -\alpha_n^2 + \frac{\omega^2}{c_2^2} & \alpha_n \mu_0 a_1 \\
        -\alpha_n \mu_0 a_1 & a_1^2 - \alpha_n^2 - \alpha_n^2\mu_0 + \frac{\omega^2}{c_1^2}
    \end{pmatrix}
\end{align}
Знайдем $Y_1(y)$:
\begin{align}
    &Y_1(y) =  \left( \frac{e^{sy}}{(s-a_1)(s - a_2)(s + a_2)} \widetilde{M(s)} \right) \Big|_{s=-a_1} = \nonumber \\
    &=-\frac{e^{-a_1 y}}{2a_1 (a_1^2 - a_2^2)} \begin{pmatrix}
        a_1^2 (1 + \mu_0) -\alpha_n^2 + \frac{\omega^2}{c_2^2} & -\alpha_n \mu_0 a_1 \\
        \alpha_n \mu_0 a_1 & a_1^2 - \alpha_n^2 - \alpha_n^2\mu_0 + \frac{\omega^2}{c_1^2}
    \end{pmatrix}
\end{align}
Знайдем $Y_2(y)$:
\begin{align}
    &Y_2(y) =  \left( \frac{e^{sy}}{(s+a_2)(s - a_1)(s + a_1)} \widetilde{M(s)} \right) \Big|_{s=a_2} = \nonumber \\
    &=\frac{e^{a_2 y}}{2a_2 (a_2^2 - a_1^2)} \begin{pmatrix}
        a_2^2 (1 + \mu_0) -\alpha_n^2 + \frac{\omega^2}{c_2^2} & \alpha_n \mu_0 a_2 \\
        -\alpha_n \mu_0 a_2 & a_2^2 - \alpha_n^2 - \alpha_n^2\mu_0 + \frac{\omega^2}{c_1^2}
    \end{pmatrix}
\end{align}
Знайдем $Y_3(y)$:
\begin{align}
    &Y_3(y) =  \left( \frac{e^{sy}}{(s-a_2)(s - a_1)(s + a_1)} \widetilde{M(s)} \right) \Big|_{s=-a_2} = \nonumber \\
    &=-\frac{e^{-a_2 y}}{2a_2 (a_2^2 - a_1^2)} \begin{pmatrix}
        a_2^2 (1 + \mu_0) -\alpha_n^2 + \frac{\omega^2}{c_2^2} & -\alpha_n \mu_0 a_2 \\
        \alpha_n \mu_0 a_2 & a_2^2 - \alpha_n^2 - \alpha_n^2\mu_0 + \frac{\omega^2}{c_1^2}
    \end{pmatrix}
\end{align}

Таким чином ми можемо записати розв'язок задачі у просторі трансформант:
\begin{equation}
    Z_n(y) = \frac{1}{1 + \mu_0} \left( Y_0(y) +  Y_1(y)  \right) * \begin{pmatrix} c_1 \\ c_2 \end{pmatrix} +  \frac{1}{1 + \mu_0} \left( Y_2(y) +  Y_3(y) \right) * \begin{pmatrix} c_3 \\ c_4 \end{pmatrix}
\end{equation}
Залишилось знайти невідомі коєфіцієнти $c_1$, $c_2$, $c_3$, $c_4$, використовуючи граничні умови (\ref{transf_bound_mat_2}).
Покрокове знаходження коєфіцієнтів наведено у (\nameref{ap_C_1}).
