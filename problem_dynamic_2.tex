\subsubsection{Постановка задачі}
Розглядається пружне прямокутне тіло, яке займає область,
що описується у декартовій системі координат співвідношеннями $-a \le x \le a$, $0 \le y \le b$.

На грані $y=b$ додано нормальне навантаження
\begin{equation}\label{bound_1_dynamic_2}
    \widetilde{\sigma_y}(x, y, t) |_{y=b} = -\widetilde{p}(x, t), \quad  \widetilde{\tau_{xy}}(x,y,t) |_{y=b} =0,
\end{equation}
де $\widetilde{p}(x,t)$ відома функція.

На бічних гранях виконано умови другої основної задачі теорії пружності
\begin{equation}\label{bound_2_dynamic_2}
    \widetilde{u}(x,y,t) |_{x=\pm a} = 0, \quad  \widetilde{v}(x,y,t) |_{x=\pm a} =0.
\end{equation}
Умови ідеального контакту вважаються виконаними по нижній грані
\begin{equation}\label{bound_3_dynamic_2}
    \widetilde{v}(x,y,t) |_{y=0} = 0, \quad \widetilde{\tau_{xy}}(x,y,t) |_{y=0} =0.
\end{equation}
Потрібно відшукати розв'язок рівнянь Ламе
\begin{equation}
    \begin{cases}
        \frac{\partial^2  \widetilde{u}(x,y,t)}{\partial x^2} + \frac{\partial^2  \widetilde{u}(x,y,t)}{\partial y^2} + \mu_0 (\frac{\partial^2  \widetilde{u}(x,y,t)}{\partial x^2} + \frac{\partial^2  \widetilde{v}(x,y,t)}{\partial x\partial y}) = \frac{1}{c_1^2} \frac{\partial^2  \widetilde{u}(x,y,t)}{\partial t^2}, \\
        \frac{\partial^2  \widetilde{v}(x,y,t)}{\partial x^2} + \frac{\partial^2  \widetilde{v}(x,y,t)}{\partial y^2} + \mu_0 (\frac{\partial^2  \widetilde{u}(x,y,t)}{\partial x \partial y} + \frac{\partial^2  \widetilde{v}(x,y,t)}{\partial y^2}) = \frac{1}{c_2^2} \frac{\partial^2  \widetilde{v}(x,y,t)}{\partial t^2}, \\
    \end{cases}
\end{equation}
за умови виконання крайових умов \eqref{bound_1_dynamic_2} - \eqref{bound_3_dynamic_2}.

Тут і далі постановку задачі розглянуто у випадку гармонічних коливань
\begin{equation}\label{garm_dynamic_2}
    \widetilde{u}(x,y,t) = u(x,y) e^{i \omega t}, \quad  \widetilde{v}(x,y,t) = v(x,y) e^{i \omega t}, \quad  \widetilde{p}(x,t) = p(x) e^{i \omega t}
\end{equation}
З урахуванням подання переміщеннь \eqref{garm_dynamic_2} рівняння Ламе переформульовано
\begin{equation}\label{lame_dynamic_2}
    \begin{cases}
        \frac{\partial^2 u(x,y)}{\partial x^2} + \frac{\partial^2 u(x,y)}{\partial y^2} + \mu_0 (\frac{\partial^2 u(x,y)}{\partial x^2} + \frac{\partial^2 v(x,y)}{\partial x\partial y}) = -\frac{\omega^2}{c_1^2}  u(x,y) \\
        \frac{\partial^2 v(x,y)}{\partial x^2} + \frac{\partial^2 v(x,y)}{\partial y^2} + \mu_0 (\frac{\partial^2 u(x,y)}{\partial x \partial y} + \frac{\partial^2 v(x,y)}{\partial y^2}) = -\frac{\omega^2}{c_2^2} v(x,y) \\
    \end{cases}
\end{equation}

Відповідно до цього переформульовано крайові умови з урахуванням симетрії задачі
\begin{equation}\label{bound_dynamic_2}
    \begin{cases}
        \sigma_y(x, y) |_{y=b} = -p(x), \quad  \tau_{xy}(x,y) |_{y=b} =0, \\
        v(x,y) |_{y=0} = 0, \quad \tau_{xy}(x,y) |_{y=0} = 0, \\
        u(x,y) |_{x=0} = 0, \quad \tau_{xy}(x,y) |_{x=0} = 0, \\
        u(x,y) |_{x=a} = 0, \quad v(x,y) |_{x=a} = 0.
    \end{cases}
\end{equation}

\subsubsection{Розв'язання векторної крайової задачі у просторі трансформант}
Після застосування перетворення Фур'є \eqref{int_trans_static_2} та інтегрування за частинами обох рівнянь \eqref{lame_dynamic_2} отримано систему звичайних диференціальних рівнянь у просторі трансформант
\begin{equation}\label{transf_dynamic_2}
    \begin{cases}
        u_n^{''}(y) - \alpha_n \mu_0 v_n^{'}(y) + (-\alpha_n^2 -\alpha_n^2 \mu_0 + \frac{\omega^2}{c_1^2}) u_n(y) = -(1 + \mu_0)sin(\alpha_n a) f(y), \\
        (1 + \mu_0) v_n^{''}(y) + \alpha_n \mu_0 u_n^{'}(y) + (- \alpha_n^2 + \frac{\omega^2}{c_2^2}) v_n(y) = 0,\\
    \end{cases}
\end{equation}
де $f(y) = \frac{\partial u(x,y)}{\partial x}|_{x=a}$ - невідома функція.

Інтегральне перетворення застосовано відповідно до крайових умов \eqref{bound_dynamic_2}
\begin{equation}\label{transf_bound_dynamic_2}
    \begin{cases}
        \left( (2G + \lambda)v_n^{'}(y) + \alpha_n \lambda u_n(y) \right)|_{y=b} = -p_n, \\
        \left(u_n^{'}(y) - \alpha_n v_n(y)  \right)|_{y=b} = 0, \\
        v_n(y)|_{y=0} = 0, \\
        \left(u_n^{'}(y) - \alpha_n v_n(y)  \right)|_{y=0} = 0,
    \end{cases}
\end{equation}
де $p_n = \int_{0}^{a} p(x) cos(\alpha_n x) dx$.

Далі сформульовано векторну крайову задачу у просторі трансформант:
\begin{equation}\label{transf_mat_dynamic_2}
    L_2\left[ Z_n(y) \right] = F_n(y),
\end{equation}
\begin{equation}
    L_2\left[ Z_n(y) \right] = A * Z_n^{''}(y) + B * Z_n^{'}(y) + C * Z_n(y),
\end{equation}
де
\begin{equation*}
    A = \begin{pmatrix}
        1 & 0 \\
        0 & 1 + \mu_0
    \end{pmatrix}, \quad
    B = \begin{pmatrix}
        0 & -\alpha_n \mu_0 \\
        \alpha_n \mu_0 & 0
    \end{pmatrix},
\end{equation*}
\begin{equation*}
    \centering
    C = \begin{pmatrix}
        -\alpha_n^2 -\alpha_n^2 \mu_0 + \frac{\omega^2}{c_1^2} & 0 \\
        0 & -\alpha_n^2 + \frac{\omega^2}{c_2^2}
    \end{pmatrix},
\end{equation*}
\begin{equation*}
    Z_n(y) = \begin{pmatrix}
        u_n(y) \\
        v_n(y)
    \end{pmatrix}, \quad 
    F_n(y) = \begin{pmatrix}
        -(1 + \mu_0)sin(\alpha_n a) f(y) \\
        0
    \end{pmatrix}
\end{equation*}
Граничні умови (\ref{transf_bound_dynamic_2}) записано за допомогою крайових функціоналів:
\begin{equation}\label{transf_bound_mat_dynamic_2}
    U_i\left[ Z_n(y) \right] = D_i,
\end{equation}
\begin{equation}
    U_i\left[ Z_n(y) \right] = E_i * Z_n^{'}(b_i) + F_i * Z_n(b_i),
\end{equation}
де $i = \overline{0, 1}$, $b_0 = b$, $b_1 = 0$.
\begin{equation*}
    E_0 = \begin{pmatrix}
        1 & 0 \\
        0 & 2G + \lambda
    \end{pmatrix}, \quad
    F_0 = \begin{pmatrix}
        0 & -\alpha_n \\
        \alpha_n \lambda & 0
    \end{pmatrix}, \quad
\end{equation*}
\begin{equation*}
    E_1 = \begin{pmatrix}
        1 & 0 \\
        0 & 0
    \end{pmatrix}, \quad
    F_1 = \begin{pmatrix}
        0 & -\alpha_n \\
        0 & 1
    \end{pmatrix}, \quad
\end{equation*}
\begin{equation*}
    D_0 = \begin{pmatrix}
        0 \\
        -p_n
    \end{pmatrix}, \quad
    D_1 = \begin{pmatrix}
        0 \\
        0
    \end{pmatrix}, \quad
\end{equation*}

За схемою, що детально викладено у попередньому параграфі (4.1.2) отримано матрицю $M(s)$:

% Розв'язок векторної одновимірної крайової задачі знайдено як суперпозиція загального розв'язку векторного однорідного рівняння $Z_n^0(y)$
% та часткового розв'язку неоднорідного рівняння $Z_n^1(y)$.
% \begin{equation}
%     Z_n(y) = Z_n^0(y) + Z_n^1(y)
% \end{equation}

% Ці розв'язки знайдено за допомогою апарату матричного диференціального числення \cite{popov_4}, \cite{gantmaher},
% для чого побудовано фундаментальну матричну систему розв'язків та фундаментальну базисну систему розв'язків відповідної матричної крайової задачі.

% Потрібно відшукати розв'язок однорідного векторного рівняння \eqref{transf_mat_dynamic_2}.
% З цією метою спочатку розшукуємо розв'язок однорідного матричного рівняння.
% Її шукатимемо у наступному поданні \cite{gantmaher}:
% \begin{equation}
%     Y(y) = \frac{1}{2\pi i} \oint_C e^{sy} M^{-1}(s)ds,
% \end{equation}
% де $M(s)$ - характеристична матриця рівняння (\ref{transf_mat_dynamic_2}), $C$ - замкнений контур, який містить усі особливі точки підінтегрального виразу.
% Матрицю $M(s)$ шукатимемо за схемою
% \begin{equation}
%     L_2\left[ e^{sy}*I \right] = e^{sy} * M(s), \quad I = \begin{pmatrix} 1 & 0 \\ 0 & 1 \end{pmatrix}.
% \end{equation}
\begin{equation}
    M(s) = \begin{pmatrix}
        s^2 - \alpha_n^2 - \alpha_n^2\mu_0 + \frac{\omega^2}{c_1^2} & -\alpha_n \mu_0 s \\
        \alpha_n \mu_0 s & s^2 (1 + \mu_0) -\alpha_n^2 + \frac{\omega^2}{c_2^2}
     \end{pmatrix}.
\end{equation}

Знайдено обернену матрицю $M^{-1}(s)$, яку побудовано у наступній формі $M^{-1}(s) = \frac{\widetilde{M(s)}}{det[M(s)]}$, де $\widetilde{M(s)}$ - транспонована матриця алгебричних доповнень,
$det[M(s)]$ - детермінант матриці
\begin{equation}
    \widetilde{M(s)} = \begin{pmatrix}
        s^2 (1 + \mu_0) -\alpha_n^2 + \frac{\omega^2}{c_2^2} & \alpha_n \mu_0 s \\
        -\alpha_n \mu_0 s & s^2 - \alpha_n^2 - \alpha_n^2\mu_0 + \frac{\omega^2}{c_1^2}
     \end{pmatrix}
\end{equation}
\begin{align}
    &det[M(s)] = \begin{vmatrix}
        s^2 - \alpha_n^2 - \alpha_n^2\mu_0 + \frac{\omega^2}{c_1^2} & -\alpha_n \mu_0 s \\
        \alpha_n \mu_0 s & s^2 (1 + \mu_0) -\alpha_n^2 + \frac{\omega^2}{c_2^2}
     \end{vmatrix} = \nonumber \\
    &=(s - s_1)(s + s_1)(s - s_2)(s + s_2),
\end{align}
де $\pm s_i$, $i=\overline{1, 2}$ - чотири прості корені рівняння $det[M(s)]=0$ (детальне знаходження яких наведено у Додатку B).

За допомогою теореми про лишки побудовано фундаментальну матричну систему розв'язків:
\begin{align*}
    &\frac{1}{2\pi i} \oint_C e^{sy} M^{-1}(s)ds = \frac{2 \pi i}{2 \pi i} \sum_{i=1}^{4} Res\left[ e^{sy} \frac{\widetilde{M(s)}}{det[M(s)]} \right] = \\
    & = Y_0(y) + Y_1(y) + Y_2(y) + Y_3(y),
\end{align*}
де
\begin{align}
    &Y_0(y) =  \left( \frac{e^{sy}}{(s+s_1)(s - s_2)(s + s_2)} \widetilde{M(s)} \right) \Big|_{s=s_1} = \nonumber \\
    &=\frac{e^{s_1 y}}{2s_1 (s_1^2 - s_2^2)} \begin{pmatrix}
        s_1^2 (1 + \mu_0) -\alpha_n^2 + \frac{\omega^2}{c_2^2} & \alpha_n \mu_0 s_1 \\
        -\alpha_n \mu_0 s_1 & s_1^2 - \alpha_n^2 - \alpha_n^2\mu_0 + \frac{\omega^2}{c_1^2}
    \end{pmatrix}
\end{align}
\begin{align}
    &Y_1(y) =  \left( \frac{e^{sy}}{(s-s_1)(s - s_2)(s + s_2)} \widetilde{M(s)} \right) \Big|_{s=-s_1} = \nonumber \\
    &=-\frac{e^{-s_1 y}}{2s_1 (s_1^2 - s_2^2)} \begin{pmatrix}
        s_1^2 (1 + \mu_0) -\alpha_n^2 + \frac{\omega^2}{c_2^2} & -\alpha_n \mu_0 s_1 \\
        \alpha_n \mu_0 s_1 & s_1^2 - \alpha_n^2 - \alpha_n^2\mu_0 + \frac{\omega^2}{c_1^2}
    \end{pmatrix}
\end{align}
\begin{align}
    &Y_2(y) =  \left( \frac{e^{sy}}{(s+s_2)(s - s_1)(s + s_1)} \widetilde{M(s)} \right) \Big|_{s=s_2} = \nonumber \\
    &=\frac{e^{s_2 y}}{2s_2 (s_2^2 - s_1^2)} \begin{pmatrix}
        s_2^2 (1 + \mu_0) -\alpha_n^2 + \frac{\omega^2}{c_2^2} & \alpha_n \mu_0 s_2 \\
        -\alpha_n \mu_0 s_2 & s_2^2 - \alpha_n^2 - \alpha_n^2\mu_0 + \frac{\omega^2}{c_1^2}
    \end{pmatrix}
\end{align}
\begin{align}
    &Y_3(y) =  \left( \frac{e^{sy}}{(s-s_2)(s - s_1)(s + s_1)} \widetilde{M(s)} \right) \Big|_{s=-s_2} = \nonumber \\
    &=-\frac{e^{-s_2 y}}{2s_2 (s_2^2 - s_1^2)} \begin{pmatrix}
        s_2^2 (1 + \mu_0) -\alpha_n^2 + \frac{\omega^2}{c_2^2} & -\alpha_n \mu_0 s_2 \\
        \alpha_n \mu_0 s_2 & s_2^2 - \alpha_n^2 - \alpha_n^2\mu_0 + \frac{\omega^2}{c_1^2}
    \end{pmatrix}
\end{align}

З метою відшукати розв'язок неоднорідної векторної крайової задачі побудовано матричну функцію Гріна,
для чого знайдено фундаментальну базисну систему розв'язків, що задовольняють матричну крайову задачу:
\begin{align}\label{psi_probl_dynamic_2}
    &L_2\left[ \Psi_i(y) \right] = 0, \nonumber \\
    &U_i\left[ \Psi_j(y) \right] = \delta_{j,i}, \quad j= \overline{0, 1}, i= \overline{0, 1},
\end{align}
де $\delta_{j,i}$ - символ Кронекера, а функції  $\Psi_0(y)$, $\Psi_1(y)$ розшукано у формі
\begin{equation}\label{psi_dynamic_2}
    \Psi_i(y) = \left( Y_0(y) + Y_1(y) \right) * C_1^i + \left( Y_2(y) + Y_3(y) \right) * C_2^i.
\end{equation}
У поданні \eqref{psi_dynamic_2} потрібно знайти невідомі матриці коефіцієнтів $C_i^j$, $i=\overline{1, 2}$, $i=\overline{0,  1}$,
що зроблено за допомогою граничних умов \eqref{transf_bound_mat_dynamic_2} (дивись Додаток С).

Матрицю Гріна побудовано аналогічно методики викладеної у параграфі (4.1.2),
де $\Psi_0(y)$, $\Psi_1(y)$ визначено формулами \eqref{psi_dynamic_2}
\begin{equation}
    G(y,\xi) = 
    \begin{cases}
        \Psi_0(y) * \Psi_1(\xi), \quad 0 \le y < \xi \\
        \Psi_1(y) * \Psi_0(\xi), \quad \xi < y \le b
    \end{cases}
\end{equation}

% Для данної матриці Гріна виконано усі властивості, зокрема виконані однорідні граничні умови \eqref{transf_bound_mat_dynamic_2}
% та однорідні рівняння рівноваги у просторі трансформант \eqref{transf_mat_dynamic_2}:
% \begin{equation*}
%     L_2\left[  G(y, \xi) \right] = 0
% \end{equation*}
% \begin{equation*}
%     U_0\left[ G(y, \xi) \right] = 0, \quad  U_1\left[ G(y, \xi) \right] = 0,
% \end{equation*}

% C этого места

Остаточний розв'язок містить невідому функцію, яку згодом відшукано

Шукані функціі перемішень у просторі трансформант можна записати у наступному вигляді
\begin{align}\label{transf_sol_u_dynamic_2}
    &u_n(y) = -\int_0^b g_2(y, \xi)f(\xi) cos(\alpha_n a) d\xi - \psi_0^2(y) p_n
\end{align}
\begin{align}\label{transf_sol_v_dynamic_2}
    &v_n(y) = -\int_0^b g_4(y, \xi)f(\xi) cos(\alpha_n a) d\xi - \psi_0^4(y) p_n
\end{align}

Викорустовуючи обернене інтегральне перетворення Фур'є до розв'язку задачі у просторі трансформант
(\ref{transf_sol_u_dynamic_2}), (\ref{transf_sol_u_dynamic_2}), отримаємо фінальний розв'язок задачі
\begin{equation}
    u(x,y) = \frac{2}{a} \sum_{n=1}^{\infty} u_n(y) sin(\alpha_n x), \quad \alpha_n = \frac{\pi n}{a}
\end{equation}
\begin{equation}
    v(x,y) = \frac{v_0(y)}{a} + \frac{2}{a} \sum_{n=1}^{\infty} v_n(y) cos(\alpha_n x), \quad \alpha_n = \frac{\pi n}{a}
\end{equation}

Знайдем тепер $v_0(y)$ розглянувши задачу у просторі трансформант \eqref{transf_dynamic_2}, \eqref{transf_bound_dynamic_2} при $n=0$, $\alpha_n = 0$.
Детальний розв'язок якої наведено в (\nameref{ap_D}). Тоді остаточний розв'язок $v(x,y)$ буде мати вигляд
\begin{align}
    &v(x,y) = -\frac{1}{a(1+\mu_0)} \int_{0}^{b}g(y,\xi) f(\xi) d\xi - \psi_0(y) \frac{p_0}{a(2G + \lambda)} \nonumber \\
    &- \frac{2}{a} \sum_{n=1}^{\infty} \left( \int_0^b \left[g_4(y, \xi) cos(\alpha_n a) f(\xi) \right]d\xi + \psi_0^4(y) p_n  \right) cos(\alpha_n x)
\end{align}

\subsubsection{Розв'зання сингулярного інтегрального рівняння задачі}
Залишилось знайти невідому функцію $f(y)$ для якої побудовано сінгулярне інтегральне рівнняння за допомогою граничної умови $\sigma_y(x, y) |_{y=b} = -p(x)$.
\begin{equation}\label{int_eq_dynamic_2}
    \frac{1}{\pi} \int_{-1}^{1} \left( a_2(t) ln\left[ \frac{1}{\lvert x - t \rvert} \right] + a_3(t, x) \right) f(t) dt = a_1(x),
\end{equation}
де
% нужно поправить
\begin{align*}
    &a_1(x) = a p(x) - 2(2G + \lambda) \frac{\partial}{\partial y} \sum_{n=1}^{\infty} \psi_0^{4}(y) p_n cos(\alpha_n x)|_{y=b} - \nonumber \\
    &- 2\lambda \frac{\partial}{\partial x} \sum_{n=1}^{\infty}\psi_0^2(y) p_n sin(\alpha_n x)|_{y=b} - \psi_0^{'}(b) p_0
\end{align*}
\begin{equation*}
    a_2(t) = \frac{2}{\pi} \lim_{n \rightarrow \infty}\left[ \frac{\partial \widetilde{g_4(y, \xi)}}{\partial y} + \lambda \widetilde{g_2(y, \xi)} \right]|_{y=b}, 
\end{equation*}
\begin{align*}
    &a_3(t, x) = \sum_{n=1}^{N} cos(\alpha_n a) cos(\alpha_n x) \left[(2G + \lambda) \frac{\partial g_4(y, t)}{\partial y} + \alpha_n \lambda g_2(y, t) \right]|_{y=b} - \\
    & - a_2 \sum_{n=0}^{N} (-1)^n (2n + 1)^{-1} e^{-(2n + 1) \frac{\pi}{2a} (b - t)} cos((2n + 1) \frac{\pi}{2a} x)
\end{align*}

Інтегральне рівняння \eqref{int_eq_dynamic_2} розв'язується методом ортогональних поліномів \cite{popov_3}.
Згідно з методом невідому функцію $f(t)$ розвинуто у ряд за поліномами Чебишева першого роду
\begin{equation}\label{unk_fun_dynamic_2}
    f(t) = \frac{1}{a_2(t)} \frac{1}{\sqrt{1 - t^2}} \sum_{k=0}^{\infty} \varphi_k T_{k}(t),
\end{equation}
де $\varphi_k$ - невідомі коєфіцієнти, $T_{k}(t)$ - поліном Чебишева першого роду.


Представлення невідомої функції \eqref{unk_fun_dynamic_2} підставлено у сінгулярне інтегральне рівняння \eqref{int_eq_dynamic_2}
та враховуючи спектральне співвідношення B.1.9 \cite{ortogonal}
\begin{equation}
    \frac{1}{\pi} \int_{-1}^{1} ln\left[ \frac{1}{\lvert x - t \rvert} \right] \frac{T_k(t)}{\sqrt{1 - t^2}} dt = \upsilon_k T_k(x),
    \begin{cases}
        \upsilon_0 = ln 2, \\
        \upsilon_k = k^{-1}, k \ge 1
    \end{cases}
\end{equation} 
отримано
\begin{equation}\label{int_eq_2_dynamic_2}
    \sum_{k=0}^{\infty}  \varphi_k \upsilon_k T_{k}( x ) + \sum_{k=0}^{\infty} \varphi_k \frac{1}{\pi} \int_{0}^{1} \frac{a_3(t, x)}{a_2(t)} \frac{T_{k}(t)}{\sqrt{1 - t^2}} dt = a_1(x)
\end{equation}

Реалізація стандартної схеми методу ортогональних поліномів приводить до розв'язання нескінченої системи алгебричних рівнянь відносно невідномих коєфіцієнтів $\varphi_k$, $k=\overline{0, \infty}$,
яка в подальшому розв'язується методом редукції.

\begin{equation}\label{int_system_dynamic_2}
    \frac{\phi_m \pi}{2m} + \sum_{k=0}^{\infty} \phi_k g_{k, m} = f_m,
\end{equation}
де $g_{k, m} = \frac{1}{\pi} \int_{-1}^{1} \frac{T_{m}(x)}{\sqrt{1 - x^2}} \int_{0}^{1} \frac{a_3(t, x )}{a_2(t)} \frac{T_{k}(t)}{\sqrt{1 - t^2}} dt dx$,
$f_m = \int_{-1}^{1} \frac{T_{m}(x) a_1(x)}{\sqrt{1 - x^2}} dx$ інтеграли відомих функцій

\subsubsection{Числові розрахунки та обговорення}
Наведені чисельні експеренти розглядаються для сталі ($E=200$ ГПА, $\mu=0.25$).

Розглянута прямокунта область $0 \le x \le 10$, $0 \le y \le 15$, при функції навантаження $p(x)=(x-2.5)^2$.
